
% === Pali ===

\cleartoverso
\chapter*{Māgaṇḍiya Sutta}

\verseref{1}
\dropCap{d}isvāna taṇhaṁ aratiṁ ragañca\\
nāhosi chando api methunasmiṁ\\
kimevidaṁ muttakarīsapuṇṇaṁ\\
pādāpi naṁ samphusituṁ na icche

\verseref{2}
\emph{etādisaṁ ce ratanaṁ na icchasi}\\
\emph{nāriṁ narindehi bahūhi patthitaṁ}\\
\emph{diṭṭhigataṁ sīlavataṁ nu jīvitaṁ}\\
\emph{bhavūpapattiñca vadesi kīdisaṁ}

\verseref{3}
idaṁ vadāmīti na tassa hoti\\
dhammesu niccheyya samuggahītaṁ\\
passañca diṭṭhīsu anuggahāya\\
ajjhattasantiṁ pacinaṁ adassaṁ

% === Slovenian ===

\cleartorecto
\chapter{Učenje Māgaṇḍiyi}

%\verseref{1}
\dropCap{S}poznal sem hrepenenje,\\
nezadovoljstvo in strast,\\
ne čutim več niti želja do spolnosti --\\
le kaj je to, polno urina in gnoja?\\
Tega se niti s svojo nogo ne bi hotel dotakniti.

%\verseref{2}
\emph{Če si ne želite zaklada, kot je ta,\\
ženske, ki je zaželena med mnogimi gospodi,\\
kakšno prepričanje, moralo, običaje, način življenja,\\
in o kakšnem posmrtnem bivanju govorite?}

%\verseref{3}
Tu ni ničesar, čemur bi lahko rekel:\\\vin »To razglašam«, Māgandiya,\\
teorije, ki so bile prisvojene med ideologijami.\\
Vendar, ko sem brez oklepanja na njih opazoval poglede,\\
sem z razumevanjem spoznal notranji mir.

% === Pali ===

\clearpage

\verseref{4}
\emph{vinicchayā yāni pakappitāni}\\
\emph{te ve munī brūsi anuggahāya}\\
\emph{ajjhattasantīti yametamatthaṁ}\\
\emph{kathaṁ nu dhīrehi paveditaṁ taṁ}

\verseref{5}
na diṭṭhiyā na sutiyā na ñāṇena\\
sīlabbatenāpi na suddhimāha\\
adiṭṭhiyā assutiyā añāṇā\\
asīlatā abbatā nopi tena\\
ete ca nissajja anuggahāya\\
santo anissāya bhavaṁ na jappe

\verseref{6}
\emph{no ce kira diṭṭhiyā na sutiyā na ñāṇena}\\
\emph{sīlabbatenāpi na suddhimāha}\\
\emph{adiṭṭhiyā assutiyā añāṇā}\\
\emph{asīlatā abbatā nopi tena}\\
\emph{maññāmahaṁ momuhameva dhammaṁ}\\
\emph{diṭṭhiyā eke paccenti suddhiṁ}

\verseref{7}
diṭṭhañca nissāya anupucchamāno\\
samuggahītesu pamohamāgā\\
ito ca nāddakkhi aṇumpi saññaṁ\\
tasmā tuvaṁ momuhato dahāsi

% === Slovenian ===

\clearpage

%\verseref{4}
\emph{Katerakoli teorija se je oblikovala,\\
resnično, Modrijan, govorite o tem brez oklepanja.\\
Ta »notranji mir«, karkoli že to pomeni,\\
kako je ta spoznan med modrijani?}

%\verseref{5}
Ne s tem, kar je bilo videno ali slišano,\\\vin niti ne z znanjem, Māgandiya,\\
ne z moralo in običaji; reče se, da čistost je;\\
tudi ne z odsotnostjo tega, kar je videno ali slišano,\\\vin niti ne z neznanjem,\\
brez morale in običajev -- tudi s tem ne.\\
Z opuščanjem tega, brez odvisnosti od česarkoli drugega,\\
miren človek, neodvisen, ne bo hrepenel po obstoju.

%\verseref{6}
\emph{Če praviš, da se o čistosti ne govori s tem,\\\vin kar je videno ali slišano,\\
niti ne z znanjem, ne z moralo in običaji;\\
tudi ne z odsotnostjo tega,\\\vin kar je videno ali slišano, niti ne z neznanjem,\\
ne brez morale in običajev -- tudi s tem ne;\\
potem si predstavljam, da je to res zmedena ideologija.\\
Nekateri se zanašajo na čistost s pogledom.}

%\verseref{7}
Zaradi takega spraševanja, ki je\\\vin odvisno od pogledov, Māgandiya,\\
si postal zbegan s tem, ko si sebi prisvojil domneve.\\
Zato ne vidiš niti najmanjšega smisla v vsem tem\\
in vse to imaš za zmedo.

% === Pali ===

\clearpage

\verseref{8}
samo visesī uda vā nihīno\\
yo maññatī so vivadetha tena\\
tīsu vidhāsu avikampamāno\\
samo visesīti na tassa hoti

\verseref{9}
saccanti so brāhmaṇo kiṁ vadeyya\\
musāti vā so vivadetha kena\\
yasmiṁ samaṁ visamaṁ vāpi natthi\\
sa kena vādaṁ paṭisaṁyujeyya

\verseref{10}
okaṁ pahāya aniketasārī\\
gāme akubbaṁ muni santhavāni\\
kāmehi ritto apurakkharāno\\
kathaṁ na viggayha janena kayirā

\verseref{11}
yehi vivitto vicareyya loke\\
na tāni uggayha vadeyya nāgo\\
jalambujaṁ kaṇḍakavārijaṁ yathā\\
jalena paṅkena canūpalittaṁ\\
evaṁ munī santivādo agiddho\\
kāme ca loke ca anūpalitto

% === Slovenian ===

\clearpage

%\verseref{8}
Kdor se ima za enakega, boljšega ali slabšega,\\
se bo boril glede na te oznake.\\
Tistemu, ki ne okleva med temi tremi razlikami,\\
se ne pojavljajo oznake »enak« ali »boljši«.

%\verseref{9}
Čemu bi lahko ta sveti človek pravil: »To je resnica«,\\
ali se prepiral: »To je napačno«?\\
Za takega ni niti enakosti niti neenakosti --\\
le s kom bi se lahko prepiral?

%\verseref{10}
Modrijan, ki je zapustil dom in je brez stalnega bivališča,\\
si ne gradi intime med ljudmi,\\
osvobodil se čutnih je užitkov,\\\vin ničesar si ne postavlja v ospredje,\\
za nič se ne zavzema in ne razpravlja z ljudmi.

%\verseref{11}
Od vseh stvari se je odrešil, med tem, ko še živi v tem svetu,\\
veliki človek se jih ne oklepa in se ne prepira.\\
Tako kot beli lokvanj, čigar pecelj raste iz vode\\
ni umazan z vodo in blatom,\\
tako tudi modrijan, govornik miru, človek brez pohlepa,\\
ni umazan s čutnostjo in svetom.

% === Pali ===

\clearpage

\verseref{12}
na vedagū diṭṭhiyā na mutiyā\\
sa mānameti na hi tammayo so\\
na kammunā nopi sutena neyyo\\
anūpanīto sa nivesanesu

\verseref{13}
saññāvirattassa na santi ganthā\\
paññāvimuttassa na santi mohā\\
saññañca diṭṭhiñca ye aggahesuṁ\\
te ghaṭṭayantā vicaranti loketi

% === Slovenian ===

\clearpage

%\verseref{12}
Tisti, ki je spoznal resnico v vidnem ali čutnem,\\
ni zato ošaben, saj se ne enači s tem.\\
Ne vodi ga, kar je bilo ustvarjeno ali naučeno,\\
in ne dela si zaključkov o tem, kar se je že uveljavilo.

%\verseref{13}
Tu ni vezi za tistega, ki je nenavezan na zaznave,\\
tu ni zmedenosti za tistega, ki je osvobojen z razumevanjem.\\
Toda tisti, ki se močno oklepajo pogledov in zaznav,\\
pohajkujejo po svetu in povzročajo probleme.

