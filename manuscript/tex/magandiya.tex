\cleartorecto
\chapterNote{Māgaṇḍiya Sutta}
\chapter{Učenje Māgaṇḍiyi}

%\verseref{1}
\dropCap{S}poznal sem hrepenenje,\\
nezadovoljstvo in strast,\\
ne čutim več niti želja do spolnosti --\\
le kaj je to, polno urina in gnoja?\\
Tega se niti s svojo nogo ne bi hotel dotakniti.

%\verseref{2}
\emph{Če si ne želite zaklada, kot je ta,\\
ženske, ki je zaželena med mnogimi gospodi,\\
kakšno prepričanje, moralo, običaje, način življenja,\\
in o kakšnem posmrtnem bivanju govorite?}

%\verseref{3}
Tu ni ničesar, čemur bi lahko rekel: »To razglašam«,\\
teorije, ki so bile prisvojene med idejami.\\
Toda, ko sem uvidel in se ne oklepal pogledov,\\
sem z razumevanjem spoznal notranji mir.

\clearpage

%\verseref{4}
\emph{Katerakoli teorija se je oblikovala,\\
resnično, Modrijan, govorite o tem brez oklepanja.\\
Ta »notranji mir«, karkoli že to pomeni,\\
kako je ta spoznan med modrijani?}

%\verseref{5}
Ne s tem, kar je bilo videno ali slišano,\\\vin niti ne z znanjem, Māgandiya,\\
ne z moralo in običaji; reče se, da čistost je;\\
tudi ne z odsotnostjo tega, kar je videno ali slišano,\\\vin niti ne z neznanjem,\\
brez morale in običajev -- tudi s tem ne.\\
Z opuščanjem tega, brez odvisnosti od česarkoli drugega,\\
miren človek, neodvisen, ne bo hrepenel po obstoju.

%\verseref{6}
\emph{Če praviš, da se o čistosti ne govori s tem,\\\vin kar je videno ali slišano,\\
niti ne z znanjem, ne z moralo in običaji;\\
tudi ne z odsotnostjo tega, kar je videno ali slišano,\\\vin niti ne z neznanjem,\\
ne brez morale in običajev -- tudi s tem ne;\\
potem si predstavljam, da je to res zmedena ideologija.\\
Nekateri se zanašajo na čistost s pogledom.}

%\verseref{7}
Zaradi takega spraševanja, ki je\\\vin odvisno od pogledov, Māgandiya,\\
si postal zbegan s tem, ko si sebi prisvojil domneve.\\
Zato ne vidiš niti najmanjšega smisla v vsem tem\\
in vse to imaš za zmedo.

\clearpage

%\verseref{8}
Kdor se ima za enakega, boljšega ali slabšega,\\
se bo boril glede na te oznake.\\
Tistemu, ki ne okleva med temi tremi razlikami,\\
se ne pojavljajo oznake »enak« ali »boljši«.

%\verseref{9}
Čemu bi lahko ta sveti človek pravil: »To je resnica«,\\
ali se prepiral: »To je napačno«?\\
Za takega ni niti enakosti niti neenakosti --\\
le s kom bi se lahko prepiral?

%\verseref{10}
Modrijan, ki je zapustil dom, tava kot brezdomec,\\
si ne gradi intime med ljudmi,\\
osvobodil se čutnih je užitkov,\\\vin ničesar si ne postavlja v ospredje,\\
za nič se ne zavzema in ne razpravlja z ljudmi.

%\verseref{11}
Od vseh stvari se je odrešil, med tem, ko še živi v tem svetu,\\
veliki človek se jih ne oklepa in se ne prepira.\\
Tako kot beli lokvanj, čigar pecelj raste iz vode\\
ni umazan z vodo in blatom,\\
tako tudi modrijan, govornik miru, človek brez pohlepa,\\
ni umazan s čutnostjo in svetom.

\clearpage

%\verseref{12}
Tisti, ki je spoznal resnico v vidnem ali čutnem,\\
ni zato ošaben, saj ni narejen iz tega.\\
Ne vodi ga, kar je bilo ustvarjeno ali naučeno,\\
ne vodi ga v nobeno bivališče.

%\verseref{13}
Tu ni vezi za tistega, ki je nenavezan na zaznave,\\
tu ni zmedenosti za tistega, ki je osvobojen z razumevanjem.\\
Toda tisti, ki se močno oklepajo pogledov in zaznav,\\
pohajkujejo naokrog in povzročajo probleme.

