
% === Pali ===

\cleartoverso
\chapter*{Sāriputta Sutta}

\verseref{1}
\dropCap{n}\emph{a me diṭṭho ito pubbe\\
na suto uda kassaci\\
evaṁ vagguvado satthā\\
tusitā gaṇimāgato}

\verseref{2}
\emph{sadevakassa lokassa\\
yathā dissati cakkhumā}\\
\emph{sabbaṁ tamaṁ vinodetvā\\
ekova ratimajjhagā}

\verseref{3}
\emph{taṁ buddhaṁ asitaṁ tādiṁ\\
akuhaṁ gaṇimāgataṁ}\\
\emph{bahūnamidha baddhānaṁ\\
atthi pañhena āgamaṁ}

\verseref{4}
\emph{bhikkhuno vijigucchato\\
bhajato rittamāsanaṁ}\\
\emph{rukkhamūlaṁ susānaṁ vā\\
pabbatānaṁ guhāsu vā}

% === Slovenian ===

\cleartorecto
\chapter{Učenje Sāriputti}

%\verseref{1}
\dropCap{N}\emph{ikoli poprej nisem videl,\\
niti od drugih slišal,\\
o učitelju, ki tako lepo govori,\\
ki je prišel sem v družbi zadovoljnih.}

%\verseref{2}
\emph{Kot svet s svojimi bogovi}\\
\emph{vidi tega Vidca,}\\
\emph{ki je razblinil vso temo,}\\
\emph{ki je sam zase prišel do zadovoljstva,}

%\verseref{3}
\emph{k temu Razsvetljenemu, neodvisnemu, ki je takšen kot je,}\\
\emph{nespletkarskemu, ki je prišel s to skupino,}\\
\emph{v imenu tistih tukaj, ki so še navezani,}\\
\emph{k temu sem pristopil z vprašanjem:}

%\verseref{4}
\emph{Za meniha, ki je potrt}\\
\emph{in biva v praznem bivališču}\\
\emph{ali ob vznožju drevesa ali na grobu}\\
\emph{ali v jamah ali v gorah}

% === Pali ===

\clearpage

\verseref{5}
\emph{uccāvacesu sayanesu\\
kīvanto tattha bheravā}\\
\emph{yehi bhikkhu na vedheyya\\
nigghose sayanāsane}

\verseref{6}
\emph{katī parissayā loke\\
gacchato agataṁ disaṁ}\\
\emph{ye bhikkhu abhisambhave\\
pantamhi sayanāsane}

\verseref{7}
\emph{kyāssa byappathayo assu\\
kyāssassu idha gocarā}\\
\emph{kāni sīlabbatānāssu\\
pahitattassa bhikkhuno}

\verseref{8}
\emph{kaṁ so sikkhaṁ samādāya\\
ekodi nipako sato}\\
\emph{kammāro rajatasseva\\
niddhame malamattano}

\verseref{9}
vijigucchamānassa yadidaṁ phāsu\\
rittāsanaṁ sayanaṁ sevato ce\\
sambodhikāmassa yathānudhammaṁ\\
taṁ te pavakkhāmi yathā pajānaṁ

% === Slovenian ===

\clearpage

%\verseref{5}
\emph{ali v drugih bivališčih,}\\
\emph{katero mero strahu mora menih prenesti,}\\
\emph{ne da bi se vznemiril}\\
\emph{v svojem tihem bivališču?}

%\verseref{6}
\emph{Koliko je tistih posvetnih nevarnosti}\\
\emph{za nekoga, ki gre proti neznanem kraju,}\\
\emph{ki bi jih moral menih zlahka preseči}\\
\emph{v svojem odmaknjenem bivališču?}

%\verseref{7}
\emph{Kakšen naj bi bil njegov govor?}\\
\emph{Kako naj v sebi žanje polja?}\\
\emph{Kakšna naj bodo morala in običaji}\\
\emph{za meniha, ki je osredotočen vase?}

%\verseref{8}
\emph{Kako naj vadi tisti,}\\
\emph{ki je zedinjen, bister, pozoren,}\\
\emph{da lahko odpihne stran svojo nečistost}\\
\emph{kot kovač, ki oblikuje srebro?}

%\verseref{9}
Kakšno naj bo udobje za potrtega, Sāriputta,\\
ki vadi v praznem bivališču,\\
ki si želi popolno razsvetljenje v skladu z Dhammo --\\
naj ti povem to, kar je v skladu z mojim razumevanjem.

% === Pali ===

\clearpage

\verseref{10}
pañcannaṁ dhīro bhayānaṁ na bhāye\\
bhikkhu sato sapariyantacārī\\
ḍaṁsādhipātānaṁ sarīsapānaṁ\\
manussaphassānaṁ catuppadānaṁ

\verseref{11}
paradhammikānampi na santaseyya\\
disvāpi tesaṁ bahubheravāni\\
athāparāni abhisambhaveyya\\
parissayāni kusalānuesī

\verseref{12}
ātaṅkaphassena khudāya phuṭṭho\\
sītaṁ athuṇhaṁ adhivāsayeyya\\
so tehi phuṭṭho bahudhā anoko\\
vīriyaṁ parakkammadaḷhaṁ kareyya

\verseref{13}
theyyaṁ na kāre na musā bhaṇeyya\\
mettāya phasse tasathāvarāni\\
yadāvilattaṁ manaso vijaññā\\
kaṇhassa pakkhoti vinodayeyya

\verseref{14}
kodhātimānassa vasaṁ na gacche\\
mūlampi tesaṁ palikhañña tiṭṭhe\\
athappiyaṁ vā pana appiyaṁ vā\\
addhābhavanto abhisambhaveyya

% === Slovenian ===

\clearpage

%\verseref{10}
Modrijan se ne bo bal petih neprijetnih stvari,\\
menih, ki je pozoren in živi v odrekanju:\\
pikov muh, komarjev in drugih žuželk,\\
stikov z ljudmi in tudi štirinožcev.

%\verseref{11}
Ne bo mu neprijetno niti med privrženci drugih idej,\\
čeprav je v njih videl veliko strahu.\\
Človek, ki je iskalec resnice,\\
bo zlahka presegel tudi te probleme:

%\verseref{12}
prenašal bo bolezen in lakoto,\\
mraz in vročino.\\
Ko bo v stiku z mnogimi stvarmi,\\
se bo uril v trdnosti svoje odločnosti.

%\verseref{13}
Ne bo se umazal s krajami, ne bo napačno govoril,\\
njegov dotik bo prijazen do trpečih in šibkih.\\
Vsako prepoznano motnjo v umu\\
bo pregnal z mislijo: »To je temna stran!«

%\verseref{14}
Ne bo padel pod vpliv jeze in zaničevanja,\\
izkopal bo njuno korenino.\\
Zlahka bo presegel vse: kar je prijetno\\
in tudi neprijetno -- on bo resnično spoštovan.

% === Pali ===

\clearpage

\verseref{15}
paññaṁ purakkhatvā kalyāṇapīti\\
vikkhambhaye tāni parissayāni\\
aratiṁ sahetha sayanamhi pante\\
caturo sahetha paridevadhamme

\verseref{16}
kiṁsū asissāmi kuva vā asissaṁ\\
dukkhaṁ vata settha kvajja sessaṁ\\
ete vitakke paridevaneyye\\
vinayetha sekho aniketacārī

\verseref{17}
annañca laddhā vasanañca kāle\\
mattaṁ so jaññā idha tosanatthaṁ\\
so tesu gutto yatacāri gāme\\
rusitopi vācaṁ pharusaṁ na vajjā

\verseref{18}
okkhittacakkhu na ca pādalolo\\
jhānānuyutto bahujāgarassa\\
upekkhamārabbha samāhitatto\\
takkāsayaṁ kukkucciyūpachinde

\verseref{19}
cudito vacībhi satimābhinande\\
sabrahmacārīsu khilaṁ pabhinde\\
vācaṁ pamuñce kusalaṁ nātivelaṁ\\
janavādadhammāya na cetayeyya

% === Slovenian ===

\clearpage

%\verseref{15}
Ker daje prednost modrosti in se veseli pravičnosti,\\
bo premagal te probleme --\\
v svojem odmaknjenem bivališču bo kos nezadovoljstvu\\
in dvignil se bo nad štiri oblike trpljenja:

%\verseref{16}
»Kaj bom jedel?« ali »Kje bom jedel?«\\
»Zagotovo bom neprijetno spal.« »Kjer naj spim nocoj?«\\
Teh misli, ki povzročajo trpljenje,\\
se učenec, ki ni nikjer nastanjen, ne oprime.

%\verseref{17}
Ko dobiva hrano in oblačila ob pravem času ve,\\
da ga le zmernost vodi v zadovoljstvo.\\
Zavarovan glede teh stvari in obvladan gre po vasi,\\
ne da bi govoril ostro, četudi bi bil izzvan.

%\verseref{18}
S povešenimi očmi, brez želje po pohajkovanju\\
ter predan meditaciji je vedno buden.\\
Z začetno ravnodušnostjo in samoobvladanostjo,\\
bo prenehal s težnjo uma k razglabljanju in skrbem.

%\verseref{19}
Ko je okaran, v pozornosti se veseli,\\
saj je prekinil trmo do kolegov v svetem življenju.\\
Dovoli si le govor, ki je dober in ne predolg\\
in ne posveča pozornosti priljubljenim govoricam.

% === Pali ===

\clearpage

\verseref{20}
athāparaṁ pañca rajāni loke\\
yesaṁ satīmā vinayāya sikkhe\\
rūpesu saddesu atho rasesu\\
gandhesu phassesu sahetha rāgaṁ

\verseref{21}
etesu dhammesu vineyya chandaṁ\\
bhikkhu satimā suvimuttacitto\\
kālena so sammā dhammaṁ parivīmaṁsamāno\\
ekodibhūto vihane tamaṁ soti

% === Slovenian ===

\clearpage

%\verseref{20}
Poleg vsega tega obstaja v svetu pet onesnaženosti,\\
ki se jim mora pozoren človek z vadbo odpovedati:\\
kos naj bo strastem oblike, zvoka,\\
tudi okusa, vonja in dotika.

%\verseref{21}
Menih s pozornostjo in s popolnoma osvobojenim srcem,\\
zavrne vsako željo po teh čutnih objektih.\\
Pravočasno, pravilno in temeljito preiskuje Dhammo,\\
z zedinjenim umom naj izniči vso to temo.

