\cleartorecto
\chapterNote{Sāriputta Sutta}
\chapter{Učenje Sāriputti}

%\verseref{1}
\dropCap{N}\emph{ikoli poprej nisem videl,\\
niti od drugih slišal,\\
o učitelju, ki tako lepo govori,\\
ki je prišel sem v družbi zadovoljnih.}

%\verseref{2}
\emph{Kot svet s svojimi božanstvi}\\
\emph{vidi tega Vidca,}\\
\emph{ki je razblinil vso temo,}\\
\emph{ki je sam zase prišel do zadovoljstva,}

%\verseref{3}
\emph{k temu Prebujenemu, neodvisnemu, ki je takšen kot je,}\\
\emph{nespletkarskemu, ki je prišel s to skupino,}\\
\emph{v imenu tistih tukaj, ki so še navezani,}\\
\emph{k temu sem pristopil z vprašanjem:}

%\verseref{4}
\emph{Za meniha, ki čuti odpor do sveta}\\
\emph{in biva v praznem bivališču}\\
\emph{ali ob vznožju drevesa ali na grobu}\\
\emph{ali v jamah ali v gorah}

\clearpage

%\verseref{5}
\emph{ali v takih in drugačnih bivališčih,}\\
\emph{katero mero strahu mora menih prenesti,}\\
\emph{ne da bi se vznemiril}\\
\emph{v svojem tihem bivališču?}

%\verseref{6}
\emph{Koliko je tistih posvetnih nevarnosti}\\
\emph{za nekoga, ki gre proti neznanem kraju,}\\
\emph{ki bi jih moral menih zlahka preseči}\\
\emph{v svojem odmaknjenem bivališču?}

%\verseref{7}
\emph{Kakšen naj bi bil njegov govor?}\\
\emph{Kako naj v sebi žanje polja?}\\
\emph{Kakšna naj bodo morala in običaji}\\
\emph{za meniha, ki je odločen?}

%\verseref{8}
\emph{Kako naj vadi tisti,}\\
\emph{ki je zedinjen, bister, pozoren,}\\
\emph{da lahko odpihne stran svojo nečistost}\\
\emph{kot kovač, ki oblikuje srebro?}

%\verseref{9}
Za tistega, ki čuti odpor, Sāriputta,\\
ki vadi v praznem bivališču,\\
ki si želi popolno prebujenje v skladu z Dhammo --\\
naj ti povem to, kar je v skladu z mojim razumevanjem.

\clearpage

%\verseref{10}
Modrijan se ne bo bal petih neprijetnih stvari,\\
menih, ki je pozoren in živi v odrekanju:\\
pikov muh, komarjev in drugih žuželk,\\
stikov z ljudmi in tudi štirinožcev.

%\verseref{11}
Ne bo mu neprijetno niti med privrženci drugih idej,\\
čeprav je v njih videl veliko strahu.\\
Človek, ki je iskalec resnice,\\
bo zlahka presegel tudi te probleme:

%\verseref{12}
prenašal bo bolezen in lakoto,\\
mraz in vročino.\\
Ko bo v stiku z mnogimi stvarmi,\\
se bo uril v trdnosti svoje odločnosti.

%\verseref{13}
Ne bo se umazal s krajami, ne bo napačno govoril,\\
njegov dotik bo prijazen do trpečih in šibkih.\\
Vsako prepoznano motnjo v umu\\
bo pregnal z mislijo: »To je temna stran!«

%\verseref{14}
Ne bo padel pod vpliv jeze in arogance,\\
izkopal bo njuno korenino.\\
Popolnoma bo presegel vse,\\
kar je ljubo in neljubo.

\clearpage

%\verseref{15}
Ker daje prednost modrosti in se veseli pravičnosti,\\
bo premagal te probleme --\\
v svojem odmaknjenem bivališču bo kos nezadovoljstvu\\
in dvignil se bo nad štiri oblike objokovanja:

%\verseref{16}
»Kaj bom jedel?« ali »Kje bom jedel?«\\
»Zagotovo bom neprijetno spal.« »Kjer naj spim nocoj?«\\
Teh misli, ki povzročajo objokovanja,\\
se učenec, ki ni nikjer nastanjen, ne oprime.

%\verseref{17}
Ko dobiva hrano in oblačila ob pravem času ve,\\
da ga le zmernost vodi v zadovoljstvo.\\
Zavarovan glede teh stvari in obvladan gre po vasi,\\
ne da bi govoril ostro, četudi bi bil izzvan.

%\verseref{18}
S povešenimi očmi, brez želje po pohajkovanju\\
ter predan meditaciji je vedno buden.\\
Z osnovano ravnodušnostjo in samoobvladanostjo,\\
bo prenehal s težnjo uma k razglabljanju in skrbem.

%\verseref{19}
Ko je okaran, v pozornosti jim prisluhne,\\
saj je prekinil trmo do kolegov v svetem življenju.\\
Dovoli si le govor, ki je dober in ne predolg\\
in ne posveča pozornosti priljubljenim govoricam.

\clearpage

%\verseref{20}
Poleg vsega tega obstaja v svetu pet onesnaženosti,\\
ki se jim mora pozoren človek z vadbo odpovedati:\\
kos naj bo strastem oblike, zvoka,\\
tudi okusa, vonja in dotika.

%\verseref{21}
Menih s pozornostjo in s popolnoma osvobojenim srcem,\\
zavrne vsako željo po teh čutnih objektih.\\
Pravočasno, pravilno in temeljito preiskuje Dhammo,\\
z zedinjenim umom naj izniči vso to temo.

