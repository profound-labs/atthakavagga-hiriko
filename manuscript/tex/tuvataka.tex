\cleartorecto
\chapterNote{Tuvaṭaka Sutta}
\chapter{Učenje o hitrosti}

%\verseref{1}
\dropCap{S}\emph{prašujem potomca sonca,\\
Njegovo svetost, o samoti in o stanju miru:\\
s kakšnim védenjem je lahko menih razbremenjen,\\
da se ne bi vezal na nič v tem svetu?}

%\verseref{2}
Popolnoma naj bi prenehal misliti: »Jaz sem«,\\
to celotno korenino uveljavljene obsedenosti.\\
Ne glede koliko hrepenenja je še v njem,\\
bo v popolni pozornosti vadil to razrešitev.

%\verseref{3}
Katerekoli ideje, ki jih direktno pozna,\\
ne glede, če te prihajajo od sebe ali od zunaj,\\
z njimi si ne bo utrjeval stališča,\\
saj krepostni temu ne bi rekli »mir«.

%\verseref{4}
Ne bo se imel za boljšega,\\
ne za slabšega niti ne za enakovrednega.\\
Čeprav dotaknjen z mnogimi stvarmi,\\
ne bo zagovarjal misli o sebi.

\clearpage

%\verseref{5}
Le v sebi lahko pride do miru,\\
menih ga ne bo iskal v zunanjem svetu.\\
Za tistega, ki je v miru v sebi,\\
ni ničesar za pridobiti, kaj šele za zavreči.

%\verseref{6}
Tako, kot je sredi morja\\
popolnoma mirno in ni nobenih valov,\\
tako tudi brez nihanja človek mirno biva --\\
tak menih si ne bo jemal časti.

%\verseref{7}
\emph{Čigar oči so odprte z odloženimi bremeni,}\\
\emph{ta je razložil Dhammo, kot jo je sam spoznal.}\\
\emph{Častiti gospod, povejte nam o poteku napredka,}\\
\emph{o etični dolžnosti in tudi o koncentraciji.}

%\verseref{8}
Menih ne dovoli, da so njegove oči nemirne,\\
svoja ušesa zapre pred družbenimi govoricami,\\
ni požrešen na okuse\\
in ničesar v svetu ne jemlje, kot »to je moje«.

%\verseref{9}
Kadarkoli menih čuti neprijetnost,\\
se ne predaja žalovanju.\\
Ne hrepeni po obstoju,\\
niti ni pretresen zaradi strahu.

\clearpage

%\verseref{10}
Hrano in pijačo,\\
živila in tudi oblačila --\\
vsa ta imetja si ne bi kopičil\\
niti se ne bo bal njihovega pomanjkanja.

%\verseref{11}
Je meditant, ki ne bo nemirno taval naokrog,\\
ki se vzdrži obžalovanih dejanj, je pazljiv.\\
Kjerkoli ima namen sedeti ali ležati,\\
menih naj živi v kraju z malo motenj.

%\verseref{12}
Naj ne spi predolgo,\\
marljiv v opreznosti naj vztraja v budnosti.\\
Tako bo zapustil vse: lenobo, iluzije, smeh, igre,\\
spolnost in vse njihove izpeljanke.

%\verseref{13}
Ritualno zdravljenje naj ne bo njegova praksa,\\
niti ne razlaga sanj, tolmačenje znakov in astrologije.\\
Učenec ne bo tolmačil živalskih krikov,\\
zdravil neplodnost ali druge bolezni.

%\verseref{14}
Menih naj ne bo pretresen zaradi kritik,\\
niti ne vzvišen zaradi pohval.\\
Pregnal bo celotno hrepenenje,\\
skupaj s strahom pred izgubo, jezo in žalitve.

\clearpage

%\verseref{15}
Menih naj ne kupuje in ne prodaja,\\
nobenih kritik naj ne raznaša,\\
naj ne bo nadloga med ljudmi\\
in laskajoč z željami po koristi.

%\verseref{16}
Menih naj se ne hvali,\\
niti naj ne izreka namigovalnih besed,\\
naj ne vadi v predrznem obnašanju,\\
in naj se izogne spornim govorom.

%\verseref{17}
Naj se ne predaja lažem;\\
tako s popolno pozornostjo ne bo zlorabljal zaupanja.\\
Hkrati naj nobenega ne prezira\\
zaradi njegovega življenja, razumevanja, morale ali običajev.

%\verseref{18}
Ko sliši mnogo besed in je izzvan\\
od mislecev in navadnih ljudi,\\
nazaj naj jim ne odgovarja ostro,\\
saj si dobri ne dela sovražnikov.

%\verseref{19}
Menih bo s takim razumevanjem Dhamme,\\
vadil v skladu s proučevanjem, s pozornostjo\\
in z razumevanjem stanja prenehanja kot miru,\\
tako ne bo zanemarjal Gotamovega učenja.

\clearpage

%\verseref{20}
On je resnično osvajalec neosvojenega,\\
sam je uvidel resnico, ki ne temelji na govoricah.\\
Zato se vedno spoštljivo pokloni\\
in pazljivo vadi po vodilu Blaženega.

