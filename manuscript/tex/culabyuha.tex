\cleartorecto
\chapterNote{Cūḷabyūha Sutta}
\chapter{Kratko učenje o slepi ulici}

%\verseref{1}
\dropCap{V}\emph{ztrajajoč v svojih pogledih,\\
»mojstri« razpravljajo o različnih argumentih:\\
»Le s tem znanjem lahko spoznaš resnico.\\
Če kdo to zanika, ta ni popoln.«}

%\verseref{2}
\emph{In s takimi argumenti razpravljajo}\\
\emph{in govorijo: »Drugi je bedak, ne mojster«.}\\
\emph{Toda kateri med argumenti je pravi?}\\
\emph{Seveda, vsi zase trdijo, da so mojstri.}

%\verseref{3}
Da je tisti, ki ne soglaša z idejami drugega\\
bedak z nižjim razumevanjem,\\
potemtakem so \emph{vsi} bedaki z nižjim razumevanjem,\\
saj vztrajajo v svojih pogledih.

%\verseref{4}
In tudi če so z lastnimi pogledi očiščeni,\\
izkušeni, pametni in jasno modri,\\
potem noben med njimi ni nižjega razumevanja,\\
saj so vsi njihovi pogledi sprejeti na isti način.

% === Slovenian ===

\clearpage

%\verseref{5}
Toda jaz ne trdim: »Tako je«,\\
kot si nasprotujejo bedaki.\\
A ker vsak vidi le svoj pogled kot resnično pravi,\\
vidijo drugega v drugem bedaka.

%\verseref{6}
\emph{Temu, kar nekateri pravijo, da je »pravilno«, »tako je«,\\
spet drugi pravijo, da je »zaman«, da je »napačno«\\
in v takšnem prepričanju se ti prepirajo.\\
Zakaj misleci ne razodenejo zgolj ene resnice?}

%\verseref{7}
Obstaja le ena resnica in ne mnogo njih.\\
Če bi človek to vedel, se seveda ne bi prepiral.\\
Toda misleci hvalijo svoje resnice,\\
zato ne morejo razodeti le ene.

%\verseref{8}
\emph{Zakaj razpravljajo o različnih resnicah,\\
in prepirljivo trdijo, da so mojstri?\\
Slišati je mnogo različnih resnic\\
ali pa samo sledijo svoji špekulaciji?}

%\verseref{9}
Mnogovrstne resnice v svetu niso trajne,\\
razen, ko si svet tako predstavlja.\\
In ko jim uspe izmisliti si novo špekulacijo,\\
govorijo le o dvojnosti: »resnici« in »laži«.

\clearpage

%\verseref{10}
Odvisnost od vsega tega, kar je videno, slišano, začuteno,\\
z moralo in običaji, on prezira.\\
Ko je prepričan v svojo teorijo, se posmehuje drugim\\
in pravi: »Ta je bedak in ne mojster.«

%\verseref{11}
Medtem ko drugega vidi kot »bedaka«,\\
o sebi govori, kot da je »mojster«.\\
In ko o sebi trdi, da je mojster,\\
s tem istočasno prezira druge.

%\verseref{12}
Le s svojimi pretiranimi pogledi je on »popoln«,\\
ter pijan od domišljavosti ima sebe za dovršenega.\\
Sam sebe se je v svoji glavi poveličil\\
in s tem tudi svoje poglede.

%\verseref{13}
Če je zaradi besed nekdo slabši,\\
potem je slabo tudi njegovo razumevanje.\\
A če sam zase modrijan spozna resnico,\\
potem ni med misleci nihče bedak.

%\verseref{14}
»Tisti, ki razpravljajo drugače od nas,\\
niso uspeli v čistosti in so nepopolni« --\\
le sektaši vsak po svoje tako razpravljajo,\\
resnično razvneti v strasti za svojimi pogledi.

\clearpage

%\verseref{15}
»Samo v tem je čistost,« trdijo;\\
pravijo, da čistosti ni v drugih idejah.\\
Le sektaši so vsak po svoje zakoreninjeni,\\
ko tako odločno razglašajo poglede svoje.

%\verseref{16}
Vendar, ko tako odločno razglašajo poglede svoje,\\
le katerega bi lahko jemali za bedaka?\\
On sam lahko izzove zgolj konflikt,\\
ko razglasi drugega z drugačnimi idejami za bedaka.

%\verseref{17}
Dokler se človek postavi za sodnika,\\
povzdiguje zgolj sebe in nadaljuje s prepiri v svetu.\\
A ko zapusti vsa sojenja,\\
več ne povzroča sporov v svetu.

