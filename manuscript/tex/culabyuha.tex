
% === Pali ===

\cleartoverso
\chapter*{Cūḷabyūha Sutta}

\verseref{1}
\dropCap{s}\emph{akaṁsakaṁdiṭṭhiparibbasānā\\
viggayha nānā kusalā vadanti\\
yo evaṁ jānāti sa vedi dhammaṁ\\
idaṁ paṭikkosamakevalī so}

\verseref{2}
\emph{evampi viggayha vivādayanti\\
bālo paro akkusaloti cāhu}\\
\emph{sacco nu vādo katamo imesaṁ\\
sabbeva hīme kusalā vadānā}

\verseref{3}
parassa ce dhammamanānujānaṁ\\
bālomako hoti nihīnapañño\\
sabbeva bālā sunihīnapaññā\\
sabbevime diṭṭhiparibbasānā

\verseref{4}
sandiṭṭhiyā ceva na vīvadātā\\
saṁsuddhapaññā kusalā mutīmā\\
na tesaṁ koci parihīnapañño\\
diṭṭhī hi tesampi tathā samattā

% === Slovenian ===

\cleartorecto
\chapter{Kratko učenje o slepi ulici}

%\verseref{1}
\dropCap{V}\emph{ztrajajoč v svojih pogledih,\\
»mojstri« razpravljajo o različnih argumentih:\\
»Le s tem znanjem lahko spoznaš resnico.\\
Če kdo to zanika, ta ni popoln.«}

%\verseref{2}
\emph{In s takimi argumenti razpravljajo}\\
\emph{in govorijo: »Drugi je bedak, ne mojster«.}\\
\emph{Toda kateri med argumenti je pravi?}\\
\emph{Seveda, vsi zase trdijo, da so mojstri.}

%\verseref{3}
Da je tisti, ki ne soglaša z idejami drugega\\
bedak z nižjim razumevanjem,\\
potemtakem so \emph{vsi} bedaki z nižjim razumevanjem,\\
saj vztrajajo v svojih pogledih.

%\verseref{4}
Toda če so vsi ti z lastnimi pogledi,\\\vin pa čeprav niso brezmadežni,\\
mojstri z uvidom in s čistim razumevanjem,\\
potem noben med njimi ni slabšega razumevanja,\\
saj prav vsi lahko postanejo popolni le z lastnimi pogledi.

% === Pali ===

\clearpage

\verseref{5}
na vāhametaṁ tathiyanti brūmi\\
yamāhu bālā mithu aññamaññaṁ\\
sakaṁ sakaṁ diṭṭhimakaṁsu saccaṁ\\
tasmā hi bāloti paraṁ dahanti

\verseref{6}
\emph{yamāhu saccaṁ tathiyanti eke\\
tamāhu aññe tucchaṁ musāti}\\
\emph{evampi vigayha vivādayanti\\
kasmā na ekaṁ samaṇā vadanti}

\verseref{7}
ekaṁ hi saccaṁ na dutīyamatthi\\
yasmiṁ pajāno vivade pajānaṁ\\
nānā te saccāni sayaṁ thunanti\\
tasmā na ekaṁ samaṇā vadanti

\verseref{8}
\emph{kasmā nu saccāni vadanti nānā\\
pavādiyāse kusalā vadānā}\\
\emph{saccāni sutāni bahūni nānā\\
udāhu te takkamanussaranti}

\verseref{9}
na heva saccāni bahūni nānā\\
aññatra saññāya niccāni loke\\
takkañca diṭṭhīsu pakappayitvā\\
saccaṁ musāti dvayadhammamāhu

% === Slovenian ===

\clearpage

%\verseref{5}
Toda jaz ne trdim: »Tako je«,\\
kot si nasprotujejo bedaki.\\
A ker vsak vidi le svoj pogled kot resnično pravi,\\
vidijo drugega v drugem bedaka.

%\verseref{6}
\emph{Temu, kar nekateri pravijo, da je »pravilno«, »tako je«,\\
spet drugi pravijo, da je »zaman«, da je »napačno«\\
in v takšnem prepričanju se ti prepirajo.\\
Zakaj misleci ne razodenejo zgolj ene resnice?}

%\verseref{7}
Obstaja le ena resnica in ne mnogo njih.\\
Če bi človek to vedel, se seveda ne bi prepiral.\\
Toda misleci razpravljajo o različnih resnicah,\\
zato ne razodenejo le ene.

%\verseref{8}
\emph{Zakaj razpravljajo o različnih resnicah,\\
in prepirljivo trdijo, da so mojstri?\\
Slišati je mnogo različnih resnic\\
ali pa samo sledijo drugi špekulaciji?}

%\verseref{9}
Mnogovrstne resnice v svetu niso trajne,\\
razen, ko si kaj predstavljamo.\\
In ko jim uspe izmisliti si novo špekulacijo,\\
govorijo le o dvojnosti: »resnici« in »laži«.

% === Pali ===

\clearpage

\verseref{10}
diṭṭhe sute sīlavate mute vā\\
ete ca nissāya vimānadassī\\
vinicchaye ṭhatvā pahassamāno\\
bālo paro akkusaloti cāha

\verseref{11}
yeneva bāloti paraṁ dahāti\\
tenātumānaṁ kusaloti cāha\\
sayamattanā so kusalāvadāno\\
aññaṁ vimāneti tadeva pāva

\verseref{12}
atisāradiṭṭhiyāva so samatto\\
mānena matto paripuṇṇamānī\\
sayameva sāmaṁ manasābhisitto\\
diṭṭhī hi sā tassa tathā samattā

\verseref{13}
parassa ce hi vacasā nihīno\\
tumo sahā hoti nihīnapañño\\
atha ce sayaṁ vedagū hoti dhīro\\
na koci bālo samaṇesu atthi

\verseref{14}
aññaṁ ito yābhivadanti dhammaṁ\\
aparaddhā suddhimakevalī te\\
evampi titthyā puthuso vadanti\\
sandiṭṭhirāgena hi tebhirattā

% === Slovenian ===

\clearpage

%\verseref{10}
Odvisnost od vsega tega, kar je videno, slišano, začuteno,\\
z moralo in običaji, on prezira.\\
Ko je prepričan v svojo teorijo, se posmehuje drugim\\
in pravi: »Ta je bedak in ne mojster.«

%\verseref{11}
Medtem ko drugega vidi kot »bedaka«,\\
o sebi govori, kot da je »mojster«.\\
In ko o sebi trdi, da je mojster,\\
s tem istočasno prezira druge.

%\verseref{12}
Le s svojimi pretiranimi pogledi je on »popoln«,\\
ter pijan od domišljavosti ima sebe za dovršenega.\\
Sam sebe je blagoslovil\\
in s tem tudi svoje poglede.

%\verseref{13}
Če besede delajo osebo slabo,\\
potem je slabo tudi njeno razumevanje.\\
A če sam zase modrijan spozna resnico,\\
potem ni med misleci nihče bedak.

%\verseref{14}
»Tisti, ki razpravljajo drugače od nas,\\
niso uspeli v čistosti in so nepopolni« --\\
le sektaši vsak po svoje tako razpravljajo,\\
resnično razvneti v strasti za svojimi pogledi.

% === Pali ===

\clearpage

\verseref{15}
idheva suddhiṁ iti vādayanti\\
nāññesu dhammesu visuddhimāhu\\
evampi titthyā puthuso niviṭṭhā\\
sakāyane tattha daḷhaṁ vadānā

\verseref{16}
sakāyane vāpi daḷhaṁ vadāno\\
kamettha bāloti paraṁ daheyya\\
sayaṁva so medhagamāvaheyya\\
paraṁ vadaṁ bālamasuddhidhammaṁ

\verseref{17}
vinicchaye ṭhatvā sayaṁ pamāya\\
uddhaṁ sa lokasmiṁ vivādameti\\
hitvāna sabbāni vinicchayāni\\
na medhagaṁ kubbati jantu loketi

% === Slovenian ===

\clearpage

%\verseref{15}
»Samo v tem je čistost,« trdijo;\\
pravijo, da čistosti ni v drugih idejah.\\
Le sektaši so vsak po svoje zakoreninjeni,\\
ko tako odločno razglašajo poglede svoje.

%\verseref{16}
Vendar, ko tako odločno razglašajo poglede svoje,\\
le katerega bi lahko jemali za bedaka?\\
On sam lahko izzove zgolj konflikt,\\
ko razglasi drugega z drugačnimi idejami za bedaka.

%\verseref{17}
Dokler se človek drži teorij,\\
povzdiguje zgolj sebe in nadaljuje s prepiri v svetu.\\
A ko zapusti vse te teorije,\\
več ne povzroča sporov v svetu.

