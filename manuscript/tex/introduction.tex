\chapter{Spremna beseda prevajalca}

\emph{Aṭṭhakavagga}, \emph{Zbirka osemverznih sutt}, je skupaj z
\emph{Pārāyanavaggo}, prepoznana kot najstarejša zbirka Buddhovih besed,
ki vsebuje najbolj temeljna učenja, ki so se širila v ustnem izročilu
med menihi v zgodnjem času Buddhovega življenja. \emph{Zbirka
osemverznih sutt} se imenuje zato, ker se \emph{aṭṭhake} (osem verzov)
pojavijo v drugi, tretji in četrti sutti in te sutte imajo tudi v sebi
ime \emph{aṭṭhaka}. Te so \emph{Guhaṭṭhaka}, \emph{Duṭṭhaṭṭhaka} in
\emph{Suddhaṭṭhaka}. Ta zbirka šestnajstih učenj (\emph{suttami}) je
zelo verjetno najprej obstajala kot samostojna zbirka in jo danes
najdemo v četrtem delu \emph{Sutta-nipāte}, delu \emph{Khuddaka-nikāye}
budističnega kanona. Zaradi te zgodovinske pomembnosti, bogate
filozofske miselnosti in ker so vsa avtentična Buddhova učenja vključena
v tej zbirki, se lahko \emph{Aṭṭhakavaggo} jemlje kot temelj celotnega
budističnega kanona.

\emph{Aṭṭhakavagga} je bila zelo poznano besedilo v času Buddhovega
življenja, vendar je danes precej pozabljeno. Tu Buddha uči o temeljnem
problemu, ki je prisoten v vsakem človeku (ne glede ali človek to vidi v
sebi ali ne) in o temeljnem konfliktu v družbi, ki formulira današnja
gibanja, miselnosti in s tem krize. Posebno dandanes družba potrebuje
novo prebuditev teh učenj v času, ko to staro učenje izgublja moč in ko
osebni, družbeni in religiozni pogledi in mnenja dobivajo večjo moč in
veljavnost, kot pa »naravno« modrost. \emph{Aṭṭhakavagga} je tu na voljo
za vse tiste, ki želijo pogledati znotraj problema sveta v katerem
živimo ter tudi na problem svoje eksistence in trpljenja in tako
opustiti nevednost in navezovanja, ki slepijo naša videnja.

\emph{Aṭṭhakavagga} je zagotovo zbirka z najbolj filozofsko tematiko.
To tudi pomeni, da je v njej Dhamma ne samo najbolj globoko opisana,
ampak da jo je verjetno tudi najtežje razumeti. To se je posebno
pokazalo pri pripravljanju tega prevoda. Upam, da prevod preveč ne
odstopa od originalnega pomena. Originalni tekst prihaja iz
\emph{Sutta Nipāta Pāli}, The Myanmar Pāli Tipitaka objavljeno v
Vipassana Research Institute in Digital Pali Reader (2012). Kljub temu,
da sem poskusil biti avtentičen do originalnega jezika, sem pogosto
moral iskati pomoč v angleških prevodih \emph{Aṭṭhakavagge}. Ti so:

\begin{packeditemize}
\item
  Bhikkhu Paññobhāsa: \emph{The Aṭṭhakavagga}, \emph{Pali, with English
  translation} -- izdaja (1) Path Press Publications, 2014; (2) Aruno
  Publications, 2014.
\item
  K.R. Norman: \emph{The Group of Discourses II} -- izdaja Pali Text
  Society, translation series no. 45, 1992.
\item
  Ñaṇādīpa Mahāthera: neobjavljeni zapiski iz knjige \emph{The Group of
  Discourses II}.
\item
  Bhikkhu Varado: \emph{The Group of Octads} -- neobjavljen prevod
  \emph{Aṭṭhakavagge}.
\item
  Bhikkhu Ṭhānissaro: prevod \emph{Aṭṭhakavagge}, objavljeno na \emph{www.accesstoinsight.org}
\end{packeditemize}

\looseness=-1
Za pomoč pri prevajanju se bi rad iz srca zahvalil Bojanu Božiču,
Andreji Grablovic, Barbari Kos in drugim, ki so prispevali svoje misli
in popravke. Brez njihove pomoči ne bi mogel končati tega prevoda.
Posebna zahvala gre tudi častitemu Gambhīru, ki je oblikovno pripravil
to lepo izdajo. Za vse napake, ki se bi morda pojavile v tej izdaji, sem
odgovoren jaz.

{\raggedleft
Bhikkhu Hiriko\\
Samostan Amaravati, Februar 2015
\par}

\clearpage

\section{Spremna beseda ob drugi izdaji}

Minilo je šest let od prve izdaje, ko je bilo natisnjenih le petdeset 
kopij. Poleg potrebe po ponatisu, sem v tej novi izdaji vključil tudi 
nekaj popravkov, ki so nastale po dodatni študiji tega palijskega dela.
Dodatna gradiva, ki so mi bila na voljo, so:

\begin{packeditemize}
\item
  Ñaṇādīpa Thera: \emph{The Silent Sages of Old}, \emph{Suttas from the Suttanipāta} -- izdaja Path Press Publications, 2018.
\item
  Bhikkhu Bodhi: \emph{The Suttanipāta}, \emph{An Ancient Collection of the Buddha's Discourses together with its Commentaries} -- izdaja Wisdom Publications, 2017.
\item
  N. A. Jayawikrama: \emph{A Critical Analysis of the Sutta Nipāta}, \emph{A series of articles from the Pali Buddhist Review} -- izdaja Wisdom Publications, 1976-7.
\end{packeditemize}

V tej izdaji sem tudi odmaknil besedilo v originalnem palijkemu jeziku,
da lahko omočimo bolj ekonomičen ponatis.

\bigskip

{\raggedleft
Bhikkhu Hiriko\\
Samostan Samanadipa, April 2021
\par}

