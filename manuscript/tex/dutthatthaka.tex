\cleartorecto
\chapterNote{Duṭṭhaṭṭhaka Sutta}
\chapter{Učenje o nepoštenosti}

%\verseref{1}
\dropCap{R}azpravljajo tisti z nepoštenimi nameni,\\
razpravljajo tudi tisti z iskrenimi nameni.\\
A modrijan se ne vpleta v nobeno razpravo,\\
on je človek, ki v nič ni vezan.

%\verseref{2}
Le kako naj nekdo preseže svoje poglede,\\
ko pa ga želje zapeljejo in tem nato sledi\\
ter si popolnost ustvarja kakor mu godi?\\
Kakor razume, tako tudi razpravlja.

\clearpage

%\verseref{3}
Če kdorkoli, čeprav ni vprašan,\\
govori drugim o svoji lastni morali in navadah\\
ter sam od sebe govori o sebi,\\
temu mojstri pravijo, da je na slabi poti.

%\verseref{4}
Toda umirjeni menih, ki dosegel je razbremenitev,\\
se ne hvali o morali svoji: »Jaz sem tak«.\\
Za tistega, za katerega ne obstajajo posvetne časti --\\
temu mojstri pravijo, da je na plemeniti poti.

%\verseref{5}
Kogarkoli ideje, ki so oblikovane, pogojene\\
in postavljene v ospredje, niso brezhibne.\\
Za tistega, ki vidi v tem korist za sebe,\\
je njegov mir odvisen od nestabilnosti.

%\verseref{6}
Res ni lahko opustiti utrjenih pogledov,\\
ki so le teorije prisvojene med idejami.\\
Človek med temi utrjenimi pogledi\\
zavrača in sprejema le različne ideje.

%\verseref{7}
Za osvobojenega človeka na tem svetu\\
ni izoblikovanih pogledov na tak ali drugačen obstoj.\\
Osvobojen človek je opustil prevaro in ponos --\\
saj le po čem bi se lahko ravnal, ko pa ni v nič zavezan?

%\verseref{8}
Če je nekdo zavezan idejam, je zavezan tudi v razprave.\\
A če ni zavezan, o čem in kako naj bi razpravljal?\\
Za njega ni ničesar pridobljenega niti zavrženega;\\
celo tukaj se je že otresel vseh pogledov.

