
% === Pali ===

\cleartoverso
\chapter*{Duṭṭhaṭṭhaka Sutta}

\verseref{1}
\dropCap{v}adanti ve duṭṭhamanāpi eke\\
athopi ve saccamanā vadanti\\
vādañca jātaṁ muni no upeti\\
tasmā munī natthi khilo kuhiñci

\verseref{2}
sakañhi diṭṭhiṁ kathamaccayeyya\\
chandānunīto ruciyā niviṭṭho\\
sayaṁ samattāni pakubbamāno\\
yathā hi jāneyya tathā vadeyya

\verseref{3}
yo attano sīlavatāni jantu\\
anānupuṭṭhova paresa pāvā\\
anariyadhammaṁ kusalā tamāhu\\
yo ātumānaṁ sayameva pāvā

\verseref{4}
santo ca bhikkhu abhinibbutatto\\
itihanti sīlesu akatthamāno\\
tamariyadhammaṁ kusalā vadanti\\
yassussadā natthi kuhiñci loke

% === Slovenian ===

\cleartorecto
\chapter{Osemverzno učenje o nepoštenosti}

%\verseref{1}
\dropCap{R}azpravljajo tisti z nepoštenimi nameni,\\
razpravljajo tudi tisti z iskrenimi nameni.\\
A modrijan se ne vpleta v nobeno razpravo,\\
on je človek, ki v nič ni vezan.

%\verseref{2}
Le kako naj nekdo preseže svoje poglede,\\
ko pa ga želje zapeljejo in tem nato sledi\\
ter si popolnost ustvarja kakor mu godi?\\
Kakor razume, tako tudi razpravlja.

%\verseref{3}
Če kdorkoli, čeprav ni vprašan,\\
govori drugim o svoji lastni morali in navadah\\
ter sam od sebe govori o sebi,\\
temu mojstri pravijo, da je na slabi poti.

%\verseref{4}
Toda umirjeni menih, ki dosegel je \emph{nibbāno},\\
se ne hvali o morali svoji: »Jaz sem tak«.\\
Za tistega, za katerega ne obstajajo posvetne časti --\\
temu mojstri pravijo, da je na plemeniti poti.

% === Pali ===

\clearpage

\verseref{5}
pakappitā saṅkhatā yassa dhammā\\
purakkhatā santi avīvadātā\\
yadattani passati ānisaṁsaṁ\\
taṁ nissito kuppapaṭiccasantiṁ

\verseref{6}
diṭṭhīnivesā na hi svātivattā\\
dhammesu niccheyya samuggahītaṁ\\
tasmā naro tesu nivesanesu\\
nirassatī ādiyatī ca dhammaṁ

\verseref{7}
dhonassa hi natthi kuhiñci loke\\
pakappitā diṭṭhi bhavābhavesu\\
māyañca mānañca pahāya dhono\\
sa kena gaccheyya anūpayo so

\verseref{8}
upayo hi dhammesu upeti vādaṁ\\
anūpayaṁ kena kathaṁ vadeyya\\
attā nirattā na hi tassa atthi\\
adhosi so diṭṭhimidheva sabbanti

% === Slovenian ===

\clearpage

%\verseref{5}
Kogarkoli ideje, ki so oblikovane, pogojene\\
in postavljene v ospredje, niso brezhibne.\\
Za tistega, ki vidi v tem korist za sebe,\\
je njegov mir odvisen od nestabilnosti.

%\verseref{6}
Res ni lahko opustiti utrjenih pogledov,\\
ki so le teorije prisvojene med idejami.\\
Človek med temi utrjenimi pogledi\\
zavrača in sprejema le različne ideje.

%\verseref{7}
Za osvobojenega človeka na tem svetu\\
ni izoblikovanih pogledov na tak ali drugačen obstoj.\\
Osvobojen človek je opustil prevaro in ponos --\\
saj le po čem bi se lahko ravnal, ko pa ni v nič zavezan?

%\verseref{8}
Brez dvoma, če je nekdo zavezan idejam,\\\vin je zavezan tudi v razprave.\\
A če ni zavezan, o čem in kako naj bi razpravljal?\\
Za njega ni ničesar pridobljenega niti zavrženega;\\
celo tukaj se je že otresel vseh pogledov.

