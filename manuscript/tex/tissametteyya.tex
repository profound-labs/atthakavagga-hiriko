
% === Pali ===

\cleartoverso

\vspace*{30mm}

\begin{verse}

\verseref{1}
\emph{methunamanuyuttassa}\\
\emph{vighātaṁ brūhi mārisa}\\
\emph{sutvāna tava sāsanaṁ}\\
\emph{viveke sikkhissāmase}

\verseref{2}
methunamanuyuttassa\\
mussatevāpi sāsanaṁ\\
micchā ca paṭipajjati\\
etaṁ tasmiṁ anāriyaṁ

\verseref{3}
eko pubbe caritvāna\\
methunaṁ yo nisevati\\
yānaṁ bhantaṁva taṁ loke\\
hīnamāhu puthujjanaṁ

\verseref{4}
yaso kitti ca yā pubbe\\
hāyatevāpi tassa sā\\
etampi disvā sikkhetha\\
methunaṁ vippahātave

\end{verse}

% === Slovenian ===

\chapter[Tissametteyya Sutta]{{kama-sutta-gray.png}{ucenje-2.png}}
\tocChapterNote{Učenje Tissu Matteyi}

\begin{verse}

%\vFirst
%\includegraphics[scale=0.3]{ce-smrtniku-gray.png}

%\verseref{1}
\emph{Za iskalca Dhamme, ki se predaja spolnosti,\\
povejte nam, o gospod, kako mu to škoduje?\\
Po poslušanju tvojega učenja,\\
bomo šli vaditi v samoto.}

%\verseref{2}
Tisti, ki se predaja spolnosti, Metteyya,\\
je pozabil vse učenje.\\
Tako tava po napačni poti --\\
in to je v njem nizkotno.

%\verseref{3}
Kdorkoli je prej živel sam zase\\
in se sedaj predaja spolnosti,\\
je kot vozilo brez nadzora --\\
takemu človeku se »slabič« v svetu pravi.

%\verseref{4}
In ne glede na to, kakšno slavo in ugled je osvojil,\\
vse je izgubljeno.\\
S tem razumevanjem naj sedaj vadi tako,\\
da popolnoma spolnost opusti.

\end{verse}

% === Pali ===

\clearpage
\begin{verse}

\verseref{5}
saṅkappehi pareto so\\
kapaṇo viya jhāyati\\
sutvā paresaṁ nigghosaṁ\\
maṅku hoti tathāvidho

\verseref{6}
atha satthāni kurute\\
paravādehi codito\\
esa khvassa mahāgedho\\
mosavajjaṁ pagāhati

\verseref{7}
paṇḍitoti samaññāto\\
ekacariyaṁ adhiṭṭhito\\
athāpi methune yutto\\
mandova parikissati

\verseref{8}
etamādīnavaṁ ñatvā\\
muni pubbāpare idha\\
ekacariyaṁ daḷhaṁ kayirā\\
na nisevetha methunaṁ

\verseref{9}
vivekaññeva sikkhetha\\
etaṁ ariyānamuttamaṁ\\
na tena seṭṭho maññetha\\
sa ve nibbānasantike

\verseref{10}
rittassa munino carato\\
kāmesu anapekkhino\\
oghatiṇṇassa pihayanti\\
kāmesu gadhitā pajāti


\end{verse}

% === Slovenian ===

\clearpage
\begin{verse}

%\verseref{5}
Zatiran z mislimi,\\
tava kot nesrečnik.\\
Ko sliši pridige drugih ljudi,\\
postane tak človek zmeden.

%\verseref{6}
In ko ga potem karajo,\\
si izmisli orožja obrambe.\\
Eno izmed njih je pohlep\\
in z njim drvi v laži.

%\verseref{7}
Prej poznan kot pameten,\\
odločen za samotno življenje,\\
a ko se preda spolnosti,\\
kot idiot potegnjen je v težave.

%\verseref{8}
Z razumevanjem teh slabih dejanj,\\
se modrijan slej ko prej,\\
odločen za svoje samotno življenje,\\
ne bo več vdajal spolnosti.

%\verseref{9}
Vadil bo v samoti --\\
to je odličnost plemenitih.\\
A zaradi tega ne bo bil vzvišen --\\
on resnično je \emph{nibbāni} blizu.

%\verseref{10}
Človeštvo, ki je zavezano čutnim užitkom,\\
zavida modrijanu, ki živi v praznini,\\
ki ne upa na čutne užitke\\
in ki je prečkal to poplavo.

\end{verse}

