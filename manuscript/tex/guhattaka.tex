\cleartorecto
\chapterNote{Guhaṭṭaka Sutta}
\chapter{Učenje o votlini}

%\verseref{1}
\dropCap{N}avezan človek je zelo skrit v votlini\\
in trdno pogreznjen v zbegano nevednost.\\
Takšen je daleč od samote,\\
saj se resnično ni lahko osvoboditi posvetne čutnosti.

%\verseref{2}
Osnovani v željah, vezani na užitke obstoja,\\
se ljudje s težavo osvobodijo, nihče jim ne more pomagati.\\
Z upanjem o tem, kaj sledi ali kaj je že minilo,\\
hrepeneči po sedanji čutnosti ali tisti prejšnji,

%\verseref{3}
so pohlepni, odvisni, zbegani v navezanosti,\\
skopi in tako utrjeni v krivi poti.\\
In ko zakrivijo neprijetnosti, objokujejo:\\
»Kaj bo z nami, ko bomo odšli od tu?«

%\verseref{4}
Zato naj človek vadi na tak način —\\
ko uvidi v svetu pota, ki so kriva,\\
naj jim zaradi tega ne sledi.\\
Modri pravijo, da življenje res je kratko.

\clearpage

%\verseref{5}
V svetu vidim to zbegano človeško raso,\\
ki se prepleta s hrepenenjem po obstoju.\\
Slabiči jočejo pred obrazom Smrti,\\
neosvobojeni hrepenenj po vseh vrst obstoja.

%\verseref{6}
Le poglej! Premetavajo se v tem, čemur pravijo »moje«,\\
so kot ribe v osušenem potoku z malo vode.\\
Ko to vidiš, le sledi poti kar »ni-moje«,\\
brez ustvarjanja navezanosti do obstoja.

%\verseref{7}
Z opuščenimi željami po obeh skrajnostih,\\
z jasnim razumevanjem sprejema, brez pohlepa po ničemer,\\
brez dejanj, ki bi si jih kasneje očital,\\
se modri ne oklepa ničesar, kar vidi in sliši.

%\verseref{8}
Modri, s popolnim razumevanjem zaznave,\\
bo prečkal to poplavo: ni pogreznjen v posedovanje,\\
z izpuljenim trnom in vedrim srcem,\\
si ne želi ne tega, ne drugega sveta.

