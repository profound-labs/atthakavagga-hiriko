
% === Pali ===

\cleartoverso
\chapter*{Guhaṭṭaka Sutta}

\verseref{1}
\dropCap{s}atto guhāyaṁ bahunābhichanno\\
tiṭṭhaṁ naro mohanasmiṁ pagāḷho\\
dūre vivekā hi tathāvidho so\\
kāmā hi loke na hi suppahāyā

\verseref{2}
icchānidānā bhavasātabaddhā\\
te duppamuñcā na hi aññamokkhā\\
pacchā pure vāpi apekkhamānā\\
ime va kāme purime va jappaṁ

\verseref{3}
kāmesu giddhā pasutā pamūḷhā\\
avadāniyā te visame niviṭṭhā\\
dukkhūpanītā paridevayanti\\
kiṁsū bhavissāma ito cutāse

\verseref{4}
tasmā hi sikkhetha idheva jantu\\
yaṁ kiñci jaññā visamanti loke\\
na tassa hetū visamaṁ careyya\\
appañhidaṁ jīvitamāhu dhīrā

% === Slovenian ===

\cleartorecto
\chapter{Osemverzno učenje o votlini}

%\verseref{1}
\dropCap{N}avezan človek je zelo skrit v votlini\\
in trdno pogreznjen v zbegano nevednost.\\
Takšen je daleč od samote,\\
saj se resnično ni lahko osvoboditi posvetne čutnosti.

%\verseref{2}
Osnovani v željah, vezani na užitke obstoja,\\
se ljudje s težavo osvobodijo, tudi medsebojnih vezi.\\
Z upanjem o tem, kaj sledi ali kaj je že minilo,\\
hrepeneči po sedanji čutnosti ali tisti prejšnji,

%\verseref{3}
so pohlepni, odvisni, zbegani v navezanosti,\\
skopi in tako utrjeni v krivi poti.\\
In ko zakrivijo neprijetnosti, objokujejo:\\
»Kaj bo z nami, ko bomo odšli od tu?«

%\verseref{4}
Zato naj človek vadi tukaj in zdaj —\\
ko uvidi v svetu pota, ki so kriva,\\
naj jim zaradi tega ne sledi.\\
Modri pravijo, da življenje res je kratko.

% === Pali ===

\clearpage

\verseref{5}
passāmi loke pariphandamānaṁ\\
pajaṁ imaṁ taṇhagataṁ bhavesu\\
hīnā narā maccumukhe lapanti\\
avītataṇhāse bhavābhavesu

\verseref{6}
mamāyite passatha phandamāne\\
maccheva appodake khīṇasote\\
etampi disvā amamo careyya\\
bhavesu āsattimakubbamāno

\verseref{7}
ubhosu antesu vineyya chandaṁ\\
phassaṁ pariññāya anānugiddho\\
yadattagarahī tadakubbamāno\\
na lippatī diṭṭhasutesu dhīro

\verseref{8}
saññaṁ pariññā vitareyya oghaṁ\\
pariggahesu muni nopalitto\\
abbūḷhasallo caramappamatto\\
nāsiṁsatī lokamimaṁ parañcāti

% === Slovenian ===

\clearpage

%\verseref{5}
V svetu vidim to človeško raso,\\
ki se ničvredno ukvarja s hrepenenjem po obstoju.\\
Slabiči jočejo v čeljustih Smrti,\\
saj jih hrepenenje po tem ali onem oddaljuje od osvoboditve.

%\verseref{6}
Le poglej! Premetavajo se v tem, čemur pravijo »moje«,\\
so kot ribe v osušenem potoku z malo vode.\\
Ko to vidiš, le sledi poti kar »ni-moje«,\\
brez ustvarjanja navezanosti do obstoja.

%\verseref{7}
Z opuščenimi željami po obeh skrajnostih,\\
z jasnim razumevanjem kontakta, brez pohlepa po ničemer,\\
brez aktivnosti, ki bi si jih kasneje očital,\\
se modri ne oklepa ničesar, kar vidi in sliši.

%\verseref{8}
Modri, s popolnim razumevanjem zaznave,\\
bo prečkal to poplavo: ni pogreznjen v posedovanje,\\
z izpuljenim trnom in vedrim srcem,\\
si ne želi ne tega, ne drugega sveta.

