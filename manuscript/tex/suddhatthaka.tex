\cleartorecto
\chapterNote{Suddhaṭṭhaka Sutta}
\chapter{Učenje o čistosti}

%\verseref{1}
\dropCap[»]{V}idim čistost,\\ univerzalnost, neškodljivost.\\
V svojih pogledih je človek popolnoma čist.«\\
S takim razumevanjem in\\\vin s spoznanjem »univerzalnosti«\\
se ta »videc čistosti« ne zanaša\\\vin na nič drugega, kot na znanje.

%\verseref{2}
Če je človek »čist« s svojimi pogledi in mislimi,\\
če je z znanjem opustil neprijetnost,\\
potem je »očiščen« z nečim posebnim.\\
Resnično, takšnega človeka izdaja njegov pogled.

%\verseref{3}
Sveti človek ne pravi, da je čistost nekaj pogojenega\\
ali kar je videno, slišano, občuteno ali kar\\\vin govori morala in običaj.\\
Tak človek ni umazan z zaslugami ali zlom,\\
za sabo je pustil vse pridobljeno in ničesar\\\vin novega si več ne ustvari.

\clearpage

%\verseref{4}
Ko ljudje opustijo, kar je bilo, se držijo tega, kar sledi.\\
Premaga jih navezanost, ker sledijo svoji strasti:\\
sprejemajo in opuščajo kakor opice,\\
ko spuščajo veje se že naslednje oprijemajo.

%\verseref{5}
Človek, ki je sprejel take ali drugačne običaje,\\
je kdaj potrt, kdaj vesel, ker se trdno oklepa zaznav.\\
Toda mojster, ki je s spoznanjem odkril Dhammo,\\
ni nikoli potrt ali vesel, ker je njegovo razumevanje jasno.

%\verseref{6}
On je v miru med vsemi idejami\\
in v vsem, kar je videno, slišano ali občuteno.\\
Okrog hodi bister in odprt --\\
le s čim na svetu bi se ga lahko merilo?

%\verseref{7}
Ti se ničesar ne oklepajo, ničesar ne dajejo v ospredje,\\
in ne razpravljajo: »To je absolutna čistost.«\\
Z razvozlanim vozlom lastništva,\\
so popolnoma brez posvetnih želja.

%\verseref{8}
Sveti človek je šel preko meja -- za njega ni ničesar tam,\\
kar si bi lahko prisvojil z znanjem in uvidom.\\
Brezstrasten do strasti, je ravnodušen do brezstrastja,\\
zanj ničesar univerzalnega ni prisvojeno.

