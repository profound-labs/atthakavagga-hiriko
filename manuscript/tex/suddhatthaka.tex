
% === Pali ===

\cleartoverso
\chapter*{Suddhaṭṭhaka Sutta}

\verseref{1}
\dropCap{p}assāmi suddhaṁ paramaṁ arogaṁ\\
diṭṭhena saṁsuddhi narassa hoti\\
evābhijānaṁ paramanti ñatvā\\
suddhānupassīti pacceti ñāṇaṁ

\verseref{2}
diṭṭhena ce suddhi narassa hoti\\
ñāṇena vā so pajahāti dukkhaṁ\\
aññena so sujjhati sopadhīko\\
diṭṭhī hi naṁ pāva tathā vadānaṁ

\verseref{3}
na brāhmaṇo aññato suddhimāha\\
diṭṭhe sute sīlavate mute vā\\
puññe ca pāpe ca anūpalitto\\
attañjaho nayidha pakubbamāno

% === Slovenian ===

\cleartorecto
\chapter{Osemverzno učenje o čistosti}

%\verseref{1}
\dropCap[»]{V}idim čistost,\\ univerzalnost, neškodljivost.\\
V svojih pogledih je človek popolnoma čist.«\\
S takim razumevanjem in\\\vin s spoznanjem »univerzalnosti«\\
se ta »videc čistosti« ne zanaša\\\vin na nič drugega, kot na znanje.

%\verseref{2}
Če je človek »čist« s svojimi pogledi in mislimi,\\
če je z znanjem opustil trpljenje,\\
potem je »očiščen« z nečim posebnim.\\
Resnično, takšnega človeka izdaja njegov pogled.

%\verseref{3}
Sveti človek ne pravi, da je čistost nekaj pogojenega\\
ali kar je videno, slišano, občuteno ali kar\\\vin govori morala in običaj.\\
Tak človek ni umazan z zaslugami ali zlom,\\
za sabo je pustil vse pridobljeno in ničesar\\\vin novega si več ne ustvari.

% === Pali ===

\clearpage

\verseref{4}
purimaṁ pahāya aparaṁ sitāse\\
ejānugā te na taranti saṅgaṁ\\
te uggahāyanti nirassajanti\\
kapīva sākhaṁ pamuñcaṁ gahāyaṁ

\verseref{5}
sayaṁ samādāya vatāni jantu\\
uccāvacaṁ gacchati saññasatto\\
vidvā ca vedehi samecca dhammaṁ\\
na uccāvacaṁ gacchati bhūripañño

\verseref{6}
sa sabbadhammesu visenibhūto\\
yaṁ kiñci diṭṭhaṁ va sutaṁ mutaṁ vā\\
tameva dassiṁ vivaṭaṁ carantaṁ\\
kenīdha lokasmi vikappayeyya

\verseref{7}
na kappayanti na purekkharonti\\
accantasuddhīti na te vadanti\\
ādānaganthaṁ gathitaṁ visajja\\
āsaṁ na kubbanti kuhiñci loke

\verseref{8}
sīmātigo brāhmaṇo tassa natthi\\
ñatvā va disvā va samuggahītaṁ\\
na rāgarāgī na virāgaratto\\
tassīdha natthī paramuggahītanti

% === Slovenian ===

\clearpage

%\verseref{4}
Ko ljudje opustijo, kar je bilo, se držijo tega, kar sledi.\\
Premaga jih navezanost, ker sledijo svoji strasti:\\
sprejemajo in opuščajo,\\
kakor opice, ko spuščajo veje se že naslednje oprijemajo.

%\verseref{5}
Človek, ki je sprejel religiozne običaje,\\
je kdaj potrt, kdaj vesel, ker se trdno oklepa zaznav.\\
Toda mojster, ki je s spoznanjem odkril Pot,\\
ni nikoli potrt ali vesel, ker je njegovo razumevanje jasno.

%\verseref{6}
On je v miru med vsemi idejami\\
in v vsem, kar je videno, slišano ali občuteno.\\
Okrog hodi bister in odprt --\\
le s čim na svetu bi se ga lahko sodilo?

%\verseref{7}
Ti se ničesar ne oklepajo, ničesar ne dajejo v ospredje,\\
in ne razpravljajo: »To je absolutna čistost.«\\
Z razvozlanim vozlom lastništva,\\
so popolnoma brez posvetnih želja.

%\verseref{8}
Sveti človek je šel preko meja -- za njega ni ničesar tam,\\
spoznal in uvidel je vse, kar je prisvojeno.\\
On je brez strasti, je ravnodušen do brezstrastja,\\
zanj ničesar zunanjega ni njegovo.

