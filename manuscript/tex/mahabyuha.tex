
% === Pali ===

\cleartoverso
\chapter*{Mahābyūha Sutta}

\verseref{1}
\dropCap{y}\emph{e kecime diṭṭhiparibbasānā\\
idameva saccanti vivādayanti\\
sabbeva te nindamanvānayanti\\
atho pasaṁsampi labhanti tattha}

\verseref{2}
appaṁ hi etaṁ na alaṁ samāya\\
duve vivādassa phalāni brūmi\\
etampi disvā na vivādayetha\\
khemābhipassaṁ avivādabhūmiṁ

\verseref{3}
yā kācimā sammutiyo puthujjā\\
sabbāva etā na upeti vidvā\\
anūpayo so upayaṁ kimeyya\\
diṭṭhe sute khantimakubbamāno

\verseref{4}
sīluttamā saññamenāhu suddhiṁ\\
vataṁ samādāya upaṭṭhitāse\\
idheva sikkhema athassa suddhiṁ\\
bhavūpanītā kusalāvadānā

% === Slovenian ===

\cleartorecto
\chapter{Daljše učenje o slepi ulici}

%\verseref{1}
\dropCap{V}\emph{si, ki bivajo v pogledih,\\
prepirljivo govorijo: »Le to je resnica« --\\
so vsi taki kritizirani\\
ali so lahko tudi pohvaljeni?}

%\verseref{2}
Taka pohvala je premajhna za spokojnost.\\
Toda naj vam povem o dveh sadovih prepirov.\\
S tem videnjem se naj nobeden ne prepira,\\
in naj bo to temeljno zatočišče za mir.

%\verseref{3}
Ne glede na vrsto dogovora,\\
za izkušenega človeka ni potrebe, da se vpleta v njih.\\
Le zakaj bi se vpletal,\\
če v njem ni naklonjenosti do vsega, kar je videl ali slišal?

%\verseref{4}
Ko mislijo, da je morala najpomembnejša\\\vin in da je čistost v samoomejevanju,\\
sprejmejo svoje običaje in se jim poklonijo:\\
»Naj vadimo prav tu in zdaj, saj v tem je čistost!«\\
Tisti, ki mislijo, da so mojstri, se zgolj ohranjajo v bivanju.

% === Pali ===

\clearpage

\verseref{5}
sace cuto sīlavatato hoti\\
pavedhatī kamma virādhayitvā\\
pajappatī patthayatī ca suddhiṁ\\
satthāva hīno pavasaṁ gharamhā

\verseref{6}
sīlabbataṁ vāpi pahāya sabbaṁ\\
kammañca sāvajjanavajjametaṁ\\
suddhiṁ asuddhinti apatthayāno\\
virato care santimanuggahāya

\verseref{7}
tamūpanissāya jigucchitaṁ vā\\
athavāpi diṭṭhaṁ va sutaṁ mutaṁ vā\\
uddhaṁsarā suddhimanutthunanti\\
avītataṇhāse bhavābhavesu

\verseref{8}
patthayamānassa hi jappitāni\\
pavedhitaṁ vāpi pakappitesu\\
cutūpapāto idha yassa natthi\\
sa kena vedheyya kuhiṁ va jappe

\verseref{9}
\emph{yamāhu dhammaṁ paramanti eke\\
tameva hīnanti panāhu aññe}\\
\emph{sacco nu vādo katamo imesaṁ\\
sabbeva hīme kusalāvadānā.}

% === Slovenian ===

\clearpage

%\verseref{5}
A ko prekrši svojo moralo in običaje,\\
je razburjen, ker mu ni uspelo v svojih dejanjih.\\
Hrepeni in si prizadeva za popolno\\\vin svobodo od vsega nizkotnega,\\
kot tisti, ki je izgubil svojo karavano in je daleč od doma.

%\verseref{6}
Vendar le z opuščanjem morale in običajev\\
in kritiziranih ali odobravajočih dejanj,\\
brez želja po »čistosti« ali »ne-čistosti«\\
živel bo zadržan, celo na mir se ne bo vezal.

%\verseref{7}
Odvisni od nevedne teme ali od tega,\\\vin kar je bilo zavrnjeno z odporom,\\
ter odvisni od tega, kar je videno, slišano ali občuteno,\\
brez osvobojenih želja po različnih oblikah obstoja,\\
vzdihujejo po čistosti v onostranstvu.

%\verseref{8}
Če je kdo čemurkoli predan, je to delo hrepenenja;\\
in kjer je prisotna vznemirjenost, je to zaradi oklepanja.\\
Ampak za tistega, ki ne premine in se ponovno ne pojavlja,\\
zakaj bi bil razburjen? Le po čem bi hrepenel?

%\verseref{9}
\emph{Prav tej ideji, ki ji nekateri pravijo »univerzalna«,}\\
\emph{ji pravijo drugi, da je »pomanjkljiva«.}\\
\emph{Kateri od teh je pravi argument?}\\
\emph{Seveda, vsi trdijo, da so mojstri.}

% === Pali ===

\clearpage

\verseref{10}
sakañhi dhammaṁ paripuṇṇamāhu\\
aññassa dhammaṁ pana hīnamāhu\\
evampi viggayha vivādayanti\\
sakaṁ sakaṁ sammutimāhu saccaṁ

\verseref{11}
parassa ce vambhayitena hīno\\
na koci dhammesu visesi assa\\
puthū hi aññassa vadanti dhammaṁ\\
nihīnato samhi daḷhaṁ vadānā

\verseref{12}
saddhammapūjāpi nesaṁ tatheva\\
yathā pasaṁsanti sakāyanāni\\
sabbeva vādā tathiyā bhaveyyuṁ\\
suddhī hi nesaṁ paccattameva

\verseref{13}
na brāhmaṇassa paraneyyamatthi\\
dhammesu niccheyya samuggahītaṁ\\
tasmā vivādāni upātivatto\\
na hi seṭṭhato passati dhammamaññaṁ

\verseref{14}
jānāmi passāmi tatheva etaṁ\\
diṭṭhiyā eke paccenti suddhiṁ\\
addakkhi ce kiñhi tumassa tena\\
atisitvā aññena vadanti suddhiṁ

% === Slovenian ===

\clearpage

%\verseref{10}
Pravijo, da so njihove ideje popolne,\\
ideje drugih pa pomanjkljive.\\
S takimi sprejetimi stališči se prepirajo,\\
vsak zase pravi, da je njegovo stališče pravilno.

%\verseref{11}
Če je zaradi zaničevanja nekdo pomanjkljiv,\\
potem med temi »duhovnimi ljudmi« ni nihče izjema.\\
Vsak zase razpravlja, da so ideje drugih pomanjkljive,\\
medtem ko se vsi odločno držijo svojih.

%\verseref{12}
A če bi bilo čaščenje njihovih lastnih učenj\\
tako resnično, kot sami hvalijo svoje,\\
potem bi bile vsakogar trditve pravilne,\\
saj bi bila čistost za vsakega posameznika osebna resnica.

%\verseref{13}
Toda za svetega moža ni ničesar, kar bi ga lahko vodilo:\\
nobene teorije, ki so prisvojene med idejami.\\
Tak človek je prerasel prepire,\\
v nobeni od ponujenih idej ne vidi vrednosti.

%\verseref{14}
»Razumem, vidim, to je to!« --\\
nekateri se tako zanašajo na čistost pogledov.\\
Pa tudi, če je kaj uvidel, le kakšno korist ima od tega?\\
Saj, ko se naslednjič pregrešijo, zopet pravijo,\\\vin da je čistost nekje drugje.

% === Pali ===

\clearpage

\verseref{15}
passaṁ naro dakkhati nāmarūpaṁ\\
disvāna vā ñassati tānimeva\\
kāmaṁ bahuṁ passatu appakaṁ vā\\
na hi tena suddhiṁ kusalā vadanti

\verseref{16}
nivissavādī na hi subbināyo\\
pakappitaṁ diṭṭhi purakkharāno\\
yaṁ nissito tattha subhaṁ vadāno\\
suddhiṁvado tattha tathaddasā so

\verseref{17}
na brāhmaṇo kappamupeti saṅkhaṁ\\
na diṭṭhisārī napi ñāṇabandhu\\
ñatvā ca so sammutiyo puthujjā\\
upekkhatī uggahaṇanti maññe

\verseref{18}
vissajja ganthāni munīdha loke\\
vivādajātesu na vaggasārī\\
santo asantesu upekkhako so\\
anuggaho uggahaṇanti maññe

\verseref{19}
pubbāsave hitvā nave akubbaṁ\\
na chandagū nopi nivissavādī\\
sa vippamutto diṭṭhigatehi dhīro\\
na lippati loke anattagarahī

% === Slovenian ===

\clearpage

%\verseref{15}
Kar človek res vidi, je to le ime in materija\\
in s takim gledanjem bo razumel prav to.\\
Naj vidi veliko ali malo, kar si želi,\\
mojstri pravijo, da v tem zagotovo ni čistosti.

%\verseref{16}
Res ni enostavno voditi takega dogmatika,\\
ki si postavlja v ospredje že izoblikovane poglede.\\
Trdi, da je dobrota v tem, od česar je sam odvisen,\\
trdi, da je on govornik »čistosti«, videc »realnosti«.

%\verseref{17}
A svetega moža, se na noben način ne da ovrednotiti:\\
on ne sledi pogledom, ni privrženec znanja\\
in na znanje vsakdanjih navad,\\
ki se jih ljudje držijo, gleda ravnodušno.

%\verseref{18}
Modrijan, ki je razvezal posvetne vezi,\\
se ne vpleta v nastale prepire:\\
je miren med nemirnimi, je ravnodušen opazovalec\\
in ničesar si ne jemlje vase, medtem ko drugi počno prav to.

%\verseref{19}
Ko je za seboj zapustil stare vplive\\
in si ne ustvarja novih, ne deluje z željami.\\
Ni dogmatik, modrec, je prost vseh prepričanj,\\
ni zatopljen v svet in si ničesar ne očita.

% === Pali ===

\clearpage

\verseref{20}
sa sabbadhammesu visenibhūto\\
yaṁ kiñci diṭṭhaṁ va sutaṁ mutaṁ vā\\
sa pannabhāro muni vippamutto\\
na kappiyo nūparato na patthiyoti

% === Slovenian ===

\clearpage

%\verseref{20}
Za njega ni ničesar, za kar bi se moral boriti\\
ali biti proti temu, kar je videl, slišal ali občutil.\\
On, modrijan, je odložil svoje breme in je svoboden,\\
brez novega oklepanja, brez inspiracij,\\\vin s prenehanjem dejanj.

