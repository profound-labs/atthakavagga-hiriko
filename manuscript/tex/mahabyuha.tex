\cleartorecto
\chapterNote{Mahābyūha Sutta}
\chapter{Daljše učenje o slepi ulici}

\enlargethispage*{\baselineskip}

%\verseref{1}
\dropCap{V}\emph{si, ki bivajo v pogledih,\\
prepirljivo govorijo: »Le to je resnica« --\\
so vsi taki kritizirani\\
ali so lahko tudi pohvaljeni?}

%\verseref{2}
Taka pohvala je premajhna za spokojnost.\\
Toda naj vam povem o dveh sadovih prepirov.\\
S tem videnjem se naj nobeden ne prepira,\\
in naj bo to temeljno zatočišče za mir.

%\verseref{3}
Ne glede na vrsto dogovora,\\
za izkušenega človeka ni potrebe, da se vpleta v njih.\\
Le zakaj bi se vpletal,\\
če v njem ni naklonjenosti do vsega, kar je videl ali slišal?

%\verseref{4}
Ko mislijo, da je morala najpomembnejša\\\vin in da je čistost v samoomejevanju,\\
sprejmejo svoje običaje in se jim poklonijo:\\
»Naj vadimo prav tu in zdaj, saj v tem je čistost!«\\
Tisti, ki mislijo, da so mojstri, se zgolj ohranjajo v bivanju.

% === Slovenian ===

\clearpage

%\verseref{5}
A ko prekrši svojo moralo in običaje,\\
drhti, ker mu ni uspelo v svojih dejanjih.\\
Hrepeni in si prizadeva za popolno svobodo,\\
kot tisti, ki je izgubil svojo karavano in je daleč od doma.

%\verseref{6}
Vendar le z opuščanjem morale in običajev\\
in kritiziranih ali odobravajočih dejanj,\\
brez želja po »čistosti« ali »ne-čistosti«\\
živel bo zadržan, celo na mir se ne bo vezal.

%\verseref{7}
Odvisni od nevedne teme ali trpinčenja,\\
ter odvisni od tega, kar je videno, slišano ali občuteno,\\
brez osvobojenih želja po različnih oblikah obstoja,\\
vzdihujejo po čistosti v različnih bivanjih.

%\verseref{8}
Če je kdo čemurkoli predan, je to delo hrepenenja;\\
in kjer je prisotna vznemirjenost, je to zaradi oklepanja.\\
Ampak za tistega, ki ne premine in se ponovno ne pojavlja,\\
zakaj bi bil razburjen? Le po čem bi hrepenel?

%\verseref{9}
\emph{Prav tej ideji, ki ji nekateri pravijo »univerzalna«,}\\
\emph{ji pravijo drugi, da je »slabša«.}\\
\emph{Kateri od teh je pravi argument?}\\
\emph{Seveda, vsi trdijo, da so mojstri.}

\clearpage

%\verseref{10}
Pravijo, da so njihove ideje popolne,\\
ideje drugih pa slabše.\\
S takimi sprejetimi stališči se prepirajo,\\
vsak zase pravi, da je njegovo stališče pravilno.

%\verseref{11}
Če je zaradi zaničevanja nekdo slabši,\\
potem med temi »duhovnimi ljudmi« ni nihče izjema.\\
Vsak zase razpravlja, da so ideje drugih slabše,\\
medtem ko se vsi odločno držijo svojih.

%\verseref{12}
A če bi bilo čaščenje njihovih lastnih učenj\\
tako resnično, kot sami hvalijo svoje,\\
potem bi bile vsakogar trditve pravilne,\\
saj bi bila čistost za vsakega le osebna resnica.

%\verseref{13}
Toda za svetega moža ni ničesar, kar bi ga lahko vodilo:\\
nobene teorije, ki so prisvojene med idejami.\\
Tak človek je prerasel prepire,\\
v nobeni od ponujenih idej ne vidi vrednosti.

%\verseref{14}
»Razumem, vidim, to je to!« --\\
nekateri se tako zanašajo na čistost pogledov.\\
Pa tudi, če je kaj uvidel, le kakšno korist ima od tega?\\
Saj, ko se naslednjič pregrešijo, zopet pravijo,\\\vin da je čistost nekje drugje.

\clearpage

%\verseref{15}
Kar človek res vidi, je to le ime in snov\\
in s takim gledanjem bo razumel prav to.\\
Naj vidi veliko ali malo, kar si želi,\\
mojstri pravijo, da v tem zagotovo ni čistosti.

%\verseref{16}
Res ni enostavno voditi takega dogmatika,\\
ki si postavlja v ospredje že izoblikovane poglede.\\
Trdi, da je »lepota« v tem, od česar je sam odvisen,\\
trdi, da je on govornik »čistosti«, videc »realnosti«.

%\verseref{17}
A svetega moža se na noben način ne da ovrednotiti:\\
on ne sledi pogledom, ni privrženec znanja.\\
In na znanje vsakdanjih navad,\\
ki se jih ljudje držijo, gleda mirnodušno.

%\verseref{18}
Modrijan, ki je razvezal posvetne vezi,\\
se ne vpleta v nastale prepire:\\
je miren med nemirnimi, je mirnodušen opazovalec\\
in ničesar si ne jemlje vase, medtem ko drugi počno prav to.

%\verseref{19}
Ko je za seboj zapustil stare vplive\\
in si ne ustvarja novih, ne deluje z željami.\\
Ni dogmatik, modrec, je prost vseh prepričanj,\\
ni zatopljen v svet in si ničesar ne očita.

\clearpage

%\verseref{20}
Za njega ni ničesar, za kar bi se moral boriti\\
ali biti proti temu, kar je videl, slišal ali občutil.\\
On, modrijan, je odložil svoje breme in je svoboden,\\
brez oklepanja, brez inspiracij in brez uživanj.

