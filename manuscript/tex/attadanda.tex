\cleartorecto
\chapterNote{Attadaṇḍa Sutta}
\chapter{Učenje o nasilju}

%\verseref{1}
\dropCap{S}trah nastane tam, kjer je nasilje --\\
le poglej ljudi v sporu!\\
Naj vam povem, kaj sem občutil,\\
kakšna tesnoba me je obšla.

%\verseref{2}
Vidim, kako se človeštvo premetava,\\
kot ribe v plitki vodi\\
nasprotujejo se med sabo --\\
ko sem videl to, me strah prevzel je.

\clearpage

%\verseref{3}
Svet mi je bil povsem brez pomena,\\
treslo se je iz vseh strani.\\
Želel sem najti si zavetje,\\
a je bilo vse že ovirano.

%\verseref{4}
Naseljenost je bila polna nasprotovanja.\\
Ko sem to videl, so me obšli tesni občutki.\\
A potem sem našel pravi trn,\\
skoraj neopaznega, zataknjenega v srcu.

%\verseref{5}
Ko sem ranjen bil s tem trnom,\\
sem tekal na vse strani.\\
A ko sem ga izdrl,\\
nisem nič več tekal, niti se utapljal.

\emph{Naj zdaj tu recitiramo o vadbi:}

%\verseref{6}
Kjerkoli so posvetne vezi,\\
naj človek od njih ne bo odvisen.\\
In ko se prebije skozi vse vrste čutnih užitkov,\\
naj vadi samo-ugasnitev.

%\verseref{7}
Naj bo resnicoljuben, ne drzen,\\
naj ne ustvarja iluzij in naj nikogar ne žali,\\
naj bo brez jeze, kot modrec, ki je prekoračil\\
zlo hrepenenja in mnogovrstne želje.

%\verseref{8}
Naj premaga zaspanost, dolgčas in lenobo,\\
naj se ne ukloni lahkomiselnosti,\\
naj ne vztraja trmasto v ponosu,\\
v sebi naj bo odločen za razbremenitev.

%\verseref{9}
Naj ne zaide v laži,\\
v stvareh si naj ne gradi želja,\\
naj dobro pozna naravo domišljavosti\\
ter naj se nasilju izogiba.

%\verseref{10}
Naj se ne veseli starega,\\
naj se ne ukloni novemu,\\
naj ne žaluje po izgubljenem\\
ter naj se ne ujame v lepoto.

%\verseref{11}
Pohlepu pravim »velika poplava«,\\
hrepenenju pravim »tok«\\
in iskanje poživil s čutnim poželenjem,\\
je kot blato, ki se ga je težko otresti.

%\verseref{12}
Brez odstopanja od resnic, modrec,\\
sveti človek, stoji na visokih tleh.\\
In ko je opustil vse stvari,\\
biva v svojem miru.

%\verseref{13}
On je resnično vedec, popoln v znanju,\\
postal je neodvisen z razumevanjem Dhamme.\\
Harmonično hodi skozi svet;\\
nikomur ničesar ne zavida.

%\verseref{14}
Kdorkoli je prešel čutna poželenja --\\
ta navezovanja v svetu za premagat težka --\\
je brez žalosti in brez skrbi,\\
je prekinil tok, je brez vezi.

%\verseref{15}
Karkoli je bilo, naj zdaj zamre\\
in za njega naj ne bo ničesar, kar naj bi še prišlo.\\
V sedanjosti se naj ničesar ne oprime,\\
saj le tako lahko živi v miru.

%\verseref{16}
Če ničesar ne jemlje in ne doživlja kot »moje«\\
v kakršnemkoli pogledu na pojavnost,\\
in ko ga ne žalosti to, česar ni,\\
ta resnično ne trpi posvetne izgube.

%\verseref{17}
Tistega, ki ne goji misli »to je zame«,\\
in ki ne misli »to je za druge«,\\
tistega, ki se zaveda nesmiselnosti »mojega«,\\
tega ne žalostijo misli »oh, ničesar ni zame!«

%\verseref{18}
Ni krut, niti ni pohlepen,\\
je vedno v miru brez strasti --\\
to je tisto, za kar sem vprašan,\\
čemur pravim prava korist.

%\verseref{19}
Ko je brez poželenja in ima znanje,\\
nobenih pogojenosti ni več za njega.\\
Vzdržal se je vse sile\\
in za njega ni nevarnosti.

%\verseref{20}
Ne med enakimi, niti ne med nižjimi\\
in ne med višjimi, modrec govori.\\
On je v miru, brez potreb,\\
ničesar ne jemlje, ničesar ne odvrača.

