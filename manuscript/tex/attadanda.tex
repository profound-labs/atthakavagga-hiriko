
% === Pali ===

\cleartoverso
\chapter*{Attadaṇḍa Sutta}

\verseref{1}
\dropCap{a}ttadaṇḍā bhayaṁ jātaṁ\\
janaṁ passatha medhagaṁ\\
saṁvegaṁ kittayissāmi\\
yathā saṁvijitaṁ mayā

\verseref{2}
phandamānaṁ pajaṁ disvā\\
macche appodake yathā\\
aññamaññehi byāruddhe\\
disvā maṁ bhayamāvisi

\verseref{3}
samantamasāro loko\\
disā sabbā sameritā\\
icchaṁ bhavanamattano\\
nāddasāsiṁ anositaṁ

\verseref{4}
osānetveva byāruddhe\\
disvā me aratī ahu\\
athettha sallamaddakkhiṁ\\
duddasaṁ hadayanissitaṁ

% === Slovenian ===

\cleartorecto
\chapter{Učenje o nasilju}

%\verseref{1}
\dropCap{S}trah nastane tam, kjer je nasilje --\\
le poglej ljudi v sporu!\\
Naj vam povem, kaj sem občutil,\\
kakšna tesnoba me je obšla nekoč.

%\verseref{2}
Vidim, kako se človeštvo premetava,\\
kot ribe v plitki vodi\\
tekmujejo med sabo --\\
ko sem videl to, me strah prevzel je.

%\verseref{3}
Svet mi je bil povsem brez pomena,\\
treslo se je iz vseh strani.\\
Želel sem najti si zavetje,\\
a je bilo vse že zasedeno.

%\verseref{4}
Naseljenost je bila polna nasprotovanja.\\
Ko sem to videl, so me obšli tesni občutki.\\
A potem sem našel pravi trn,\\
skoraj neopaznega, zataknjenega v srcu.

% === Pali ===

\clearpage

\verseref{5}
yena sallena otiṇṇo\\
disā sabbā vidhāvati\\
tameva sallamabbuyha\\
na dhāvati na sīdati

\verseref{6}
\emph{tattha sikkhānugīyanti}\\
yāni loke gadhitāni\\
na tesu pasuto siyā\\
nibbijjha sabbaso kāme\\
sikkhe nibbānamattano

\verseref{7}
sacco siyā appagabbho\\
amāyo rittapesuṇo\\
akkodhano lobhapāpaṁ\\
vevicchaṁ vitare muni

\verseref{8}
niddaṁ tandiṁ sahe thīnaṁ\\
pamādena na saṁvase\\
atimāne na tiṭṭheyya\\
nibbānamanaso naro

\verseref{9}
mosavajje na nīyetha\\
rūpe snehaṁ na kubbaye\\
mānañca parijāneyya\\
sāhasā virato care

% === Slovenian ===

\clearpage

%\verseref{5}
Ko sem ranjen bil s tem trnom,\\
sem tekal na vse strani.\\
A ko sem ga izdrl,\\
nisem nič več tekal, niti se več utapljal.

%\verseref{6}
\emph{Naj zdaj tu recitiramo o vadbi:}\\
Kjerkoli so posvetne vezi,\\
naj človek od njih ne bo odvisen.\\
In ko se prebije skozi vse vrste čutnih užitkov,\\
naj vadi samo-ugasnitev.

%\verseref{7}
Naj bo resnicoljuben, ne drzen,\\
naj ne ustvarja iluzij in naj nikogar ne žali,\\
naj bo brez jeze, kot modrec, ki je prekoračil\\
zlo hrepenenja in mnogovrstne želje.

%\verseref{8}
Naj premaga zaspanost, dolgčas in lenobo,\\
naj se ne ukloni lahkomiselnosti,\\
naj ne vztraja trmasto v ponosu,\\
v sebi naj bo odločen za \emph{nibbāno}.

%\verseref{9}
Naj ne zaide v laži,\\
v materialnem si naj ne gradi želja,\\
naj dobro pozna naravo domišljavosti\\
ter naj se nasilju izogiba.

% === Pali ===

\clearpage

\verseref{10}
purāṇaṁ nābhinandeyya\\
nave khantiṁ na kubbaye\\
hiyyamāne na soceyya\\
ākāsaṁ na sito siyā

\verseref{11}
gedhaṁ brūmi mahoghoti\\
ājavaṁ brūmi jappanaṁ\\
ārammaṇaṁ pakappanaṁ\\
kāmapaṅko duraccayo

\verseref{12}
saccā avokkamma muni\\
thale tiṭṭhati brāhmaṇo\\
sabbaṁ so paṭinissajja\\
sa ve santoti vuccati

\verseref{13}
sa ve vidvā sa vedagū\\
ñatvā dhammaṁ anissito\\
sammā so loke iriyāno\\
na pihetīdha kassaci

\verseref{14}
yodha kāme accatari\\
saṅgaṁ loke duraccayaṁ\\
na so socati nājjheti\\
chinnasoto abandhano

% === Slovenian ===

\clearpage

%\verseref{10}
Naj se ne veseli starega,\\
naj se ne ukloni novemu,\\
naj ne žaluje po tem, kar je bilo izgubljeno\\
ter naj se ne ujame v to, kar se blešči in sije.

%\verseref{11}
Pohlepu pravim »velika poplava«,\\
hrepenenju pravim »tok«\\
in iskanje poživil s čutnim poželenjem,\\
je kot blato, ki se ga je težko otresti.

%\verseref{12}
Brez odstopanja od resnic, modrec,\\
sveti človek, stoji na varnih tleh.\\
In ko je opustil vse stvari,\\
biva v svojem miru.

%\verseref{13}
On je resnično človek, ki je spoznal resnico, on je tisti, ki ve,\\
postal je neodvisen z razumevanjem Dhamme.\\
S tem znanjem hodi skozi svet;\\
nikomur ničesar ne zavida.

%\verseref{14}
Kdorkoli je prešel čutna poželenja --\\
ta navezovanja v svetu za premagat težka --\\
je brez žalosti in brez skrbi,\\
je prekinil tok, je brez vezi.

% === Pali ===

\clearpage

\verseref{15}
yaṁ pubbe taṁ visosehi\\
pacchā te māhu kiñcanaṁ\\
majjhe ce no gahessasi\\
upasanto carissasi

\verseref{16}
sabbaso nāmarūpasmiṁ\\
yassa natthi mamāyitaṁ\\
asatā ca na socati\\
sa ve loke na jīyati

\verseref{17}
yassa natthi idaṁ meti\\
paresaṁ vāpi kiñcanaṁ\\
mamattaṁ so asaṁvindaṁ\\
natthi meti na socati

\verseref{18}
aniṭṭhurī ananugiddho\\
anejo sabbadhī samo\\
tamānisaṁsaṁ pabrūmi\\
pucchito avikampinaṁ

\verseref{19}
anejassa vijānato\\
natthi kāci nisaṅkhati\\
virato so viyārabbhā\\
khemaṁ passati sabbadhi

% === Slovenian ===

\clearpage

%\verseref{15}
Karkoli je bilo, naj zdaj zamre\\
in za njega naj ne bo ničesar, kar naj bi še prišlo.\\
V sedanjosti se naj ničesar ne oprime,\\
saj le tako lahko živi v miru.

%\verseref{16}
Če ničesar ne jemlje in ne doživlja kot »moje«\\
v kakršnemkoli pogledu na ime in materijo,\\
in ko ga ne žalosti to, česar ni,\\
ta resnično ne trpi posvetne izgube.

%\verseref{17}
Tistega, ki ne goji misli »to je zame«,\\
in ki ne misli »to je za druge«,\\
tistega, ki se zaveda nesmiselnosti »mojega«,\\
tega ne žalostijo misli »oh, ničesar ni zame!«

%\verseref{18}
Ni krut, niti ni pohlepen,\\
je vedno v miru brez strasti --\\
to je tisto, za kar sem vprašan,\\
čemur pravim prava korist.

%\verseref{19}
Ko je brez poželenja in ima znanje,\\
nobenih pogojenosti ni več za njega.\\
Vzdržal se je vse sile\\
in za njega ni nevarnosti.

% === Pali ===

\clearpage

\verseref{20}
na samesu na omesu\\
na ussesu vadate muni\\
santo so vītamaccharo\\
nādeti na nirassatīti

% === Slovenian ===

\clearpage

%\verseref{20}
Ne med enakimi, niti ne med nižjimi\\
in ne med višjimi, modrec govori.\\
On je v miru, brez strahu pred izgubo,\\
saj ni tu ničesar, od česar bi jemal slovo.

