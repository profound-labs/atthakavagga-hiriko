\cleartorecto
\chapterNote{Kalahavivāda Sutta}
\chapter{Učenje o prepirih in sporih}

%\verseref{1}
\dropCap{O}\emph{d kod izvirajo prepiri in spori,\\
žalost in skrbi, skupaj z nevoščljivostjo,\\
od kod domišljavost in prezir, skupaj z žalitvami?\\
Od kod vse to izvira? Prosimo, povejte nam to.}

%\verseref{2}
Prepiri in spori nastanejo, ko jemljemo stvari, kot nam drage,\\
enako tudi žalost in skrbi, skupaj z nevoščljivostjo,\\
kot tudi domišljavost in prezir, skupaj z žalitvami.\\
Prepiri in spori so združeni z nevoščljivostjo,\\
in temu sledijo žalitve.

%\verseref{3}
\emph{Na čem je osnovano to, da si jemljemo stvari, kot nam drage,}\\
\emph{na čem je osnovano hrepenenje, ki se pretaka v svetu?}\\
\emph{In na čem so osnovani upi in cilji,}\\
\emph{ta človekova prihodnja stanja?}

%\verseref{4}
To, kar nam je drago, je osnovano na željah,\\
kakor tudi vsako hrepenenje, ki se pretaka v svetu.\\
In na tem so osnovani tudi upi in cilji,\\
ta človekova prihodnja stanja.

\clearpage

%\verseref{5}
\emph{A na čem so osnovane želje?}\\
\emph{In od kod izhajajo teorije,}\\
\emph{jeza, laži, in negotovost,}\\
\emph{ter razne ideje, ki jih razglašajo misleci?}

%\verseref{6}
Želje so osnovane na tem,\\
kar prinaša užitek in neužitek v svetu.\\
Z videnjem obstoja in ne-obstoja v stvareh,\\
si človek ustvarja teorije o svetu.

%\verseref{7}
Jeza, laž in dvom so tu,\\
ko je prisotna ta dvojnost.\\
Negotovi naj zato vadi na poti znanja,\\
skozi razumevanja učenj mislecev.

%\verseref{8}
\emph{V čem je osnovano užitek in neužitek?}\\
\emph{Kaj mora biti odsotno, da to ne bi več obstajalo?}\\
\emph{In tudi obstoj in neobstoj, karkoli že to pomeni,}\\
\emph{povejte mi, v čem je vse to osnovano?}

%\verseref{9}
V sprejetju sta osnovana užitek in neužitek;\\
če ni sprejetja, ti občutki ne obstajajo.\\
In tudi obstoj in neobstoj, karkoli že to pomeni,\\
povem ti, da sta osnovana prav v tem.

\clearpage

%\verseref{10}
\emph{Na čem je na svetu osnovano sprejetje?\\
Kaj mora biti odsotno, da ne obstaja »moje«?}\\
\emph{In iz česa izhaja imetje?}\\
\emph{Kaj mora izginiti, da se sprejetje ne sprime?}

%\verseref{11}
Sprejetje je pogojeno od imena in snovnosti.\\
Če snov izgine, se sprejetje ne sprime.\\
Imetje je osnovano v željah.\\
Ko ni prisotne želje, ni imetja.

%\verseref{12}
\emph{Kaj se more doseči, da snov izgine?}\\
\emph{In kako lahko nezadovoljstvo in zadovoljstvo izgine?}\\
\emph{Povejte mi, na kakšen način stvari izginejo,}\\
\emph{saj rad bi razumel prav to.}

%\verseref{13}
Ko nezaznava zaznavo; ko nezaznava nezaznavo;\\
ko ne nezaznava; ko ne zaznava »izginitve«.\\
Za tistega, ki to doseže, snov izgine,\\
saj je zaznava osnova za uveljavljeno obsedenost.

%\verseref{14}
\emph{To, kar smo vas vprašali, ste nam pojasnili.}\\
\emph{Toda, naj vas vprašamo še nekaj -- prosimo, povejte nam --}\\
\emph{ali je res, da nekateri pametni ljudje razpravljajo,}\\
\emph{da je to najvišja čistost duha le do te mere,}\\
\emph{in ali ti razpravljajo, da je tu še nekaj več?}

\clearpage

%\verseref{15}
Res je, da nekateri pametni ljudje razpravljajo,\\
da je to najvišja čistost duha.\\
Toda nekateri drugi pravijo, da je to anihilacija,\\
češ da imajo znanje o ugasnitvi.

%\verseref{16}
Modrec razume, da je vse to odvisnost,\\
raziskovalec razume naravo odvisnosti.\\
Z razumevanjem je osvobojen in se ne prepira,\\
modri človek ne sprejema obstoja ali neobstoja.

