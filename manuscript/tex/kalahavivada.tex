
% === Pali ===

\cleartoverso
\chapter*{Kalahavivāda Sutta}

\verseref{1}
\dropCap{k}\emph{utopahūtā kalahā vivādā\\
paridevasokā sahamaccharā ca\\
mānātimānā sahapesuṇā ca\\
kutopahūtā te tadiṅgha brūhi}

\verseref{2}
piyappahūtā kalahā vivādā\\
paridevasokā sahamaccharā ca\\
mānātimānā sahapesuṇā ca\\
maccherayuttā kalahā vivādā\\
vivādajātesu ca pesuṇāni

\verseref{3}
\emph{piyā su lokasmiṁ kutonidānā\\
ye vāpi lobhā vicaranti loke}\\
\emph{āsā ca niṭṭhā ca kutonidānā\\
ye samparāyāya narassa honti}

\verseref{4}
chandānidānāni piyāni loke\\
ye cāpi lobhā vicaranti loke\\
āsā ca niṭṭhā ca itonidānā\\
ye samparāyāya narassa honti

% === Slovenian ===

\cleartorecto
\chapter{Učenje o prepirih in sporih}

%\verseref{1}
\dropCap{O}\emph{d kod izvirajo prepiri in spori,\\
žalostinke in skrbi, skupaj s strahom pred izgubo,\\
od kod domišljavost in prezir, skupaj z žalitvami?\\
Od kod vse to izvira? Prosimo, povejte nam to.}

%\verseref{2}
Prepiri in spori nastanejo, ko jemljemo stvari, kot nam drage,\\
enako tudi žalostinke in skrbi, skupaj s strahom pred izgubo,\\
kot tudi domišljavost in prezir, skupaj z žalitvami.\\
Prepiri in spori so združeni s strahom pred izgubo,\\
in temu sledijo žalitve.

%\verseref{3}
\emph{Na čem je osnovano to, da si jemljemo stvari, kot nam drage,}\\
\emph{na čem je osnovano hrepenenje, ki se pretaka v svetu?}\\
\emph{In na čem so osnovani upi in cilji,}\\
\emph{ta človekova prihodnja stanja?}

%\verseref{4}
To, kar nam je drago, je osnovano na željah,\\
kakor tudi vsako hrepenenje, ki se pretaka v svetu.\\
In na tem so osnovani tudi upi in cilji,\\
ta človekova prihodnja stanja.

% === Pali ===

\clearpage

\verseref{5}
\emph{chando nu lokasmiṁ kutonidāno\\
vinicchayā vāpi kutopahūtā}\\
\emph{kodho mosavajjañca kathaṁkathā ca\\
ye vāpi dhammā samaṇena vuttā}

\verseref{6}
sātaṁ asātanti yamāhu loke\\
tamūpanissāya pahoti chando\\
rūpesu disvā vibhavaṁ bhavañca\\
vinicchayaṁ kubbati jantu loke

\verseref{7}
kodho mosavajjañca kathaṁkathā ca\\
etepi dhammā dvayameva sante\\
kathaṁkathī ñāṇapathāya sikkhe\\
ñatvā pavuttā samaṇena dhammā

\verseref{8}
\emph{sātaṁ asātañca kutonidānā\\
kismiṁ asante na bhavanti hete}\\
\emph{vibhavaṁ bhavañcāpi yametamatthaṁ\\
etaṁ me pabrūhi yatonidānaṁ}

\verseref{9}
phassanidānaṁ sātaṁ asātaṁ\\
phasse asante na bhavanti hete\\
vibhavaṁ bhavañcāpi yametamatthaṁ\\
etaṁ te pabrūmi itonidānaṁ

% === Slovenian ===

\clearpage

%\verseref{5}
\emph{A na čem so osnovane želje?}\\
\emph{In od kod izhajajo teorije,}\\
\emph{jeza, laži, in negotovost,}\\
\emph{ter razne ideologije, ki jih razglašajo misleci?}

%\verseref{6}
Želje so osnovane na tem,\\
kar prinaša užitek in neužitek v svetu.\\
Z videnjem obstoja in ne-obstoja materije,\\
si človek ustvarja teorije o svetu.

%\verseref{7}
Jeza, laž, negotovost in ideje so tu,\\
ko je prisotna ta dvojnost.\\
Negotovi človek naj zato vadi na poti razumevanja,\\
s pomočjo razumevanja spoznanj mislecev.

%\verseref{8}
\emph{V čem je osnovano užitek in neužitek?}\\
\emph{Kaj mora biti odsotno, da to ne bi več obstajalo?}\\
\emph{In tudi obstoj in neobstoj, karkoli že to pomeni,}\\
\emph{povejte mi, v čem je vse to osnovano?}

%\verseref{9}
V kontaktih sta osnovana užitek in neužitek;\\
če ni kontakta, ti občutki ne obstajajo.\\
In tudi obstoj in neobstoj, karkoli že to pomeni,\\
povem ti, da sta osnovana prav v tem.

% === Pali ===

\clearpage

\verseref{10}
\emph{phasso nu lokasmiṁ kutonidāno\\
pariggahā cāpi kutopahūtā}\\
\emph{kismiṁ asante na mamattamatthi\\
kismiṁ vibhūte na phusanti phassā}

\verseref{11}
nāmañca rūpañca paṭicca phasso\\
icchānidānāni pariggahāni\\
icchāyasantyā na mamattamatthi\\
rūpe vibhūte na phusanti phassā

\verseref{12}
\emph{kathaṁ sametassa vibhoti rūpaṁ\\
sukhaṁ dukhañcāpi kathaṁ vibhoti}\\
\emph{etaṁ me pabrūhi yathā vibhoti\\
taṁ jāniyāmāti me mano ahu}

\verseref{13}
na saññasaññī na visaññasaññī\\
nopi asaññī na vibhūtasaññī\\
evaṁ sametassa vibhoti rūpaṁ\\
saññānidānā hi papañcasaṅkhā

\verseref{14}
\emph{yaṁ taṁ apucchimha akittayī no}\\
\emph{aññaṁ taṁ pucchāma tadiṅgha brūhi}\\
\emph{ettāvataggaṁ nu vadanti heke}\\
\emph{yakkhassa suddhiṁ idha paṇḍitāse}\\
\emph{udāhu aññampi vadanti etto}

% === Slovenian ===

\clearpage

%\verseref{10}
\emph{Na čem je na svetu osnovan kontakt?\\
Kaj mora biti odsotno, da ne obstaja »moje«?}\\
\emph{In iz česa izhaja imetje?}\\
\emph{Kaj mora izginiti, da kontakt se ne sklene?}

%\verseref{11}
Kontakt je pogojen od imena in materije.\\
Če materija izgine, se kontakt ne sklene.\\
Imetje je osnovano v željah.\\
Ko ni prisotne želje, ni potrebe po »mojem«.

%\verseref{12}
\emph{Kaj se more doseči, da materija izgine?}\\
\emph{In kako lahko nezadovoljstvo in zadovoljstvo izgine?}\\
\emph{Povejte mi, na kakšen način stvari izginejo,}\\
\emph{saj rad bi razumel prav to.}

%\verseref{13}
Ko nima zaznave o zaznavi; ko nima zaznave o nezaznavi;\\
ko ni brez zaznave; ko nima zaznave »izginitve«.\\
Za tistega, ki to doseže, materija izgine,\\
saj je zaznava osnova za uveljavljeno obsedenost.

%\verseref{14}
\emph{To, kar smo vas vprašali, ste nam pojasnili.}\\
\emph{Toda, naj vas vprašamo še nekaj -- prosimo, povejte nam --}\\
\emph{ali je res, da nekateri pametni ljudje razpravljajo,}\\
\emph{da je to najvišja čistost duha le do te mere,}\\
\emph{in ali ti razpravljajo, da je tu še nekaj več?}

% === Pali ===

\clearpage

\verseref{15}
ettāvataggampi vadanti heke\\
yakkhassa suddhiṁ idha paṇḍitāse\\
tesaṁ paneke samayaṁ vadanti\\
anupādisese kusalā vadānā

\verseref{16}
ete ca ñatvā upanissitāti\\
ñatvā munī nissaye so vimaṁsī\\
ñatvā vimutto na vivādameti\\
bhavābhavāya na sameti dhīroti

% === Slovenian ===

\clearpage

%\verseref{15}
Res je, da nekateri pametni ljudje razpravljajo,\\
da je to najvišja čistost duha.\\
In nekateri mojstri pravijo, da je to le začasno --\\
da imajo znanje o ugasnitvi.

%\verseref{16}
Modrec razume, da je vse to odvisnost,\\
raziskovalec razume naravo odvisnosti.\\
Z razumevanjem je osvobojen in se ne prepira.\\
Modri človek ne sprejema nikakršnega obstoja.

