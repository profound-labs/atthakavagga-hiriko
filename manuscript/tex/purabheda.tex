\cleartorecto
\chapterNote{Purābheda Sutta}
\chapter{Učenje o pred razpadom}

%\verseref{1}
\dropCap{S}\emph{kakšno vizijo, s kakšno moralo\\
se lahko komu pravi, da je v miru?\\
Povejte mi to, o Gotama.\\
Sprašujem vas o najboljšem človeku.}

%\verseref{2}
S prenehanjem hrepenenja pred razpadom,\\
ni odvisen od preteklosti,\\
ni pogojen v sedanjosti,\\
si ta ničesar ne daje v ospredje.

\clearpage

%\verseref{3}
Brez jeze in strahu,\\
brez hvale in skrbi,\\
jasen govornik, nedomišljav,\\
pravi modrec v govoru.

%\verseref{4}
Brez navezanosti na prihodnost,\\
ne obžaluje preteklosti.\\
Vidi osamo med vsemi kontakti,\\
ničesar med pogledi ga ne zapelje.

%\verseref{5}
Je odmaknjen, ni spletkar,\\
ni pohlepen in se ne boji izgube,\\
ni drzen, je brez odpora\\
in žalitvam ni naklonjen.

%\verseref{6}
Se ne predaja prijetnostim,\\
ni naklonjen zaničevanju,\\
je blag in bistrega duha,\\
ni pobožen niti ni pasiven.

%\verseref{7}
Ne vodijo ga želje po koristih\\
in ni razočaran, če le teh ni.\\
Hrepenenje ga ne ovira,\\
niti ni lakomen za okusne draži.

%\verseref{8}
Za mirnodušnega in pozornega,\\
ki se nima za enakega,\\
za boljšega ali slabšega --\\
za njega ni časti.

%\verseref{9}
Za njega ni odvisnosti,\\
z razumevanjem Dhamme je neodvisen.\\
V njem ne obstaja hrepenenje\\
niti po obstoju niti po ne-obstoju.

%\verseref{10}
Takemu pravim, da je v miru:\\
ki si ne želi čutnih užitkov,\\
v katerem ni najti nobenih vezi,\\
saj je za seboj pustil vse navezanosti.

%\verseref{11}
Zanj ne obstajajo sinovi ali živina,\\
niti polja ali lastnina.\\
Skratka, pri njem ni ničesar,\\
kar bi lahko pridobil ali zavrgel.

%\verseref{12}
S tem, s čimer bi ga lahko ljudje kritizirali,\\
ali tudi misleci in sveti možje,\\
on ničesar ne daje v ospredje,\\
zato je tudi sredi kritike miren.

%\verseref{13}
Je brez pohlepa in se ne boji izgub,\\
takšen modrijan se nima za nadrejenega,\\
niti za enakega niti za podrejenega.\\
Ni zapadel v predstave; on je brez predstav.

%\verseref{14}
Tistemu, ki ničesar ne poseduje,\\
ki se ne žalosti, ker ničesar nima,\\
in ki ne sledi nobenim idejam --\\
le takemu se lahko resnično pravi, da je v miru.

