
% === Pali ===

\cleartoverso
\chapter{Jarā Sutta}

\verseref{1}
\dropCap{a}ppaṁ vata jīvitaṁ idaṁ\\
oraṁ vassasatāpi miyyati\\
yo cepi aticca jīvati\\
atha kho so jarasāpi miyyati

\verseref{2}
socanti janā mamāyite\\
na hi santi niccā pariggahā\\
vinābhāvasantamevidaṁ\\
iti disvā nāgāramāvase

\verseref{3}
maraṇenapi taṁ pahīyati\\
yaṁ puriso mamayidanti maññati\\
etampi viditvā paṇḍito\\
na mamattāya nametha māmako

\verseref{4}
supinena yathāpi saṅgataṁ\\
paṭibuddho puriso na passati\\
evampi piyāyitaṁ janaṁ\\
petaṁ kālaṅkataṁ na passati

% === Slovenian ===

\cleartorecto
\chapter*{Učenje o starosti}

%\verseref{1}
\dropCap{R}esnično, to življenje je malenkostno.\\
Človek umre, še preden doživi sto let.\\
Pa tudi, če živi dlje,\\
zagotovo umre od onemoglosti.

%\verseref{2}
Ljudje objokujejo nad tem, kar imenujejo »moje« --\\
resnično, trajno imetje ne obstaja.\\
S spoznanjem, da so ločitve neizogibne,\\
se naj človek izogne družinskemu življenju.

%\verseref{3}
Ob smrti izgine vse,\\
v kar človek je verjel, da je njegovo.\\
Ko pametni človek uvidi to spoznanje,\\
moj učenec, ta ne bo naklonjen posedovanju.

%\verseref{4}
Tako, kot prebujeni človek\\
nikoli več ne sreča tistega iz sanj,\\
tako tudi mi ne vidimo več\\
naših ljubljenih, ki so umrli, preminuli.

% === Pali ===

\clearpage

\verseref{5}
diṭṭhāpi sutāpi te janā\\
yesaṁ nāmamidaṁ pavuccati\\
nāmamevāvasissati\\
akkheyyaṁ petassa jantuno

\verseref{6}
sokapparidevamaccharaṁ\\
na jahanti giddhā mamāyite\\
tasmā munayo pariggahaṁ\\
hitvā acariṁsu khemadassino

\verseref{7}
patilīnacarassa bhikkhuno\\
bhajamānassa vivittamāsanaṁ\\
sāmaggiyamāhu tassa taṁ\\
yo attānaṁ bhavane na dassaye

\verseref{8}
sabbattha munī anissito\\
na piyaṁ kubbati nopi appiyaṁ\\
tasmiṁ paridevamaccharaṁ\\
paṇṇe vāri yathā na limpati

\verseref{9}
udabindu yathāpi pokkhare\\
padume vāri yathā na limpati\\
evaṁ muni nopalippati\\
yadidaṁ diṭṭhasutaṁ mutesu vā

% === Slovenian ===

\clearpage

%\verseref{5}
V vsakdanu srečujemo in slišimo ljudi,\\
ki jih poznamo pod različnimi imeni --\\
za mrtvim bo ostalo samo ime,\\
po katerem se ga bomo spominjali.

%\verseref{6}
Tisti, ki si lastijo in posedujejo,\\
so v oblasti žalosti, tegobe in strahu pred izgubo.\\
Zato so modreci, ki so zapustili imetje,\\
postali redovniki in si našli zatočišče.

%\verseref{7}
O menihu, ki živi odmaknjeno,\\
ki se zateka v samotna bivališča,\\
pravijo, da to ustreza njegovi naravi\\
ter da se ne bo nikoli ustalil v trajnem bivališču.

%\verseref{8}
Modrijan je povsod neodvisen:\\
ničesar si ne ustvarja kot ljubljeno ali neljubljeno\\
ter žalovanje in strah pred izgubo\\
se ga ne oprimeta, tako kot voda lista ne.

%\verseref{9}
Tako kot kaplja vlage na vodni liliji\\
in voda na lokvanju spolzita,\\
tako tudi modrijan ni omadeževan\\
s tem, kar je videl, slišal ali čutil.

% === Pali ===

\clearpage

\verseref{10}
dhono na hi tena maññati\\
yadidaṁ diṭṭhasutaṁ mutesu vā\\
nāññena visuddhimicchati\\
na hi so rajjati no virajjatīti

% === Slovenian ===

\clearpage

%\verseref{10}
Prečiščen človek si ničesar ne predstavlja,\\
o tem kar je videl, slišal ali čutil.\\
Ne želi si pridobiti čistosti preko drugih stvari,\\
ni več strasten niti ravnodušen.

