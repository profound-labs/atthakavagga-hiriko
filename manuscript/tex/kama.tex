
% === Pali ===

\cleartoverso
\chapter*{Kāma Sutta}

\verseref{1}
\dropCap{K}āmaṁ kāmayamānassa\\
tassa ce taṁ samijjhati\\
addhā pītimano hoti\\
laddhā macco yadicchati

\verseref{2}
assa ce kāmayānassa\\
chandajātassa jantuno\\
te kāmā parihāyanti\\
sallaviddhova ruppati

\verseref{3}
yo kāme parivajjeti\\
sappasseva padā siro\\
somaṁ visattikaṁ loke\\
sato samativattati

\verseref{4}
khettaṁ vatthuṁ hiraññaṁ vā\\
gavassaṁ dāsaporisaṁ\\
thiyo bandhū puthu kāme\\
yo naro anugijjhati

% === Slovenian ===

\cleartorecto
\chapter{Učenje o čutnih užitkih}

%\verseref{1}
\dropCap{Č}e smrtniku uspe\\
uresničiti svoje čutne želje,\\
postane vesel v svojem srcu,\\
saj je dobil tisto, kar si je želel.

%\verseref{2}
Če za tega človeka\\
s čutnostjo iz želja rojeno,\\
ti predmeti čutnosti oslabijo,\\
je v taki stiski, kot da je preboden s trnom.

%\verseref{3}
Kdor pa se čutnosti izogiba\\
kot stopalo kačji glavi,\\
tak človek, ki je pozoren,\\
preseže posvetno navezanost.

%\verseref{4}
Tistega, ki je pohlepen\\
na zemljo, lastnino, zlato,\\
krave, konje, sužnje, služabnike, ženske,\\
odnose in raznovrstne čutnosti,

% === Pali ===

\clearpage

\verseref{5}
abalā naṁ balīyanti\\
maddantenaṁ parissayā\\
tato naṁ dukkhamanveti\\
nāvaṁ bhinnamivodakaṁ

\verseref{6}
tasmā jantu sadā sato\\
kāmāni parivajjaye\\
te pahāya tare oghaṁ\\
nāvaṁ sitvāva pāragūti

% === Slovenian ===

\clearpage

%\verseref{5}
premagajo nemočni,\\
težave ga potisnejo k tlom,\\
tako, da neprijetnost pride k njemu,\\
kot voda v polomljen čoln.

%\verseref{6}
Zato naj se tisti, ki je vedno pozoren, izogne čutnosti;\\
ko jo bo opustil, bo prečkal to poplavo,\\
tako kot tisti, ki je s črpanjem vode iz svojega čolna,\\
uspel priti na drugi breg.

