\cleartorecto
\chapterNote{Kāma Sutta}
\chapter{Učenje o čutnih užitkih}

%\verseref{1}
\dropCap{Č}e človeku uspe\\
uresničiti svoje čutne želje,\\
postane vesel v svojem srcu,\\
saj je dobil zaželjeno.

%\verseref{2}
Če za tega človeka\\
s čutnostjo iz želja rojeno,\\
ti predmeti čutnosti oslabijo,\\
ga zaboli, kot da je preboden s trnom.

%\verseref{3}
Kdor pa se čutnosti izogiba\\
kot stopalo kačji glavi,\\
tak človek, ki je pozoren,\\
preseže navezanost na svet.

%\verseref{4}
Tistega, ki je pohlepen\\
na zemljo, lastnino in zlato,\\
krave in konje, služabnike in ženske,\\
odnose in raznovrstne čutnosti,

\clearpage

%\verseref{5}
premagajo nemočni,\\
težave ga potisnejo k tlom,\\
neprijetnost pride k njemu,\\
kot voda v polomljen čoln.

%\verseref{6}
Zato naj se tisti, ki je vedno pozoren, izogne čutnosti;\\
ko jo bo opustil, bo prečkal to poplavo,\\
tako kot tisti, ki je s črpanjem vode iz svojega čolna,\\
uspel priti na drugi breg.

