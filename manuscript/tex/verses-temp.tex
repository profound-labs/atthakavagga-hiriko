\textbf{1 }

\textbf{Učenje o čutnih užitkih }

v1. Če smrtniku uspe uresničiti svoje čutne želje, postane vesel v svojem srcu, saj je dobil tisto, kar si je želel.

v2. Če za tega človeka s čutnostjo iz želja rojeno, ti predmeti čutnosti oslabijo, je v taki stiski, kot da je preboden s trnom.

v3 Kdor pa se čutnosti izogiba kot stopalo kačji glavi, tak človek, ki je pozoren, preseže posvetno navezanost.

v4. Tistega, ki je pohlepen na zemljo, lastnino, zlato, krave, konje, sužnje, služabnike, ženske, odnose in raznovrstne čutnosti,

v5. premagajo nemočni, težave ga potisnejo k tlom, tako, da neprijetnost pride k njemu, kot voda v polomljen čoln.

v6. Zato naj se tisti, ki je vedno pozoren, izogne čutnosti; ko jo bo opustil, bo prečkal to poplavo, tako kot tisti, ki je s črpanjem vode iz svojega čolna, uspel priti na drugi breg.

\textbf{2 Osemverzno učenje o votlini }

v1. Navezan človek je zelo skrit v votlini in trdno pogreznjen v zbegano nevednost. Takšen je daleč od samote, saj se resnično ni lahko osvoboditi posvetne čutnosti.

v2. Osnovani v željah, vezani na užitke obstoja, se ljudje s težavo osvobodijo, tudi medsebojnih vezi. Z upanjem o tem, kaj sledi ali kaj je že minilo, hrepeneči po sedanji čutnosti ali tisti prejšnji,

v3. so pohlepni, odvisni, zbegani v navezanosti, skopi in tako utrjeni v krivi poti. In ko zakrivijo neprijetnosti, objokujejo: »Kaj bo z nami, ko bomo odšli od tu?«

v4. Zato naj človek vadi tukaj in zdaj --- ko uvidi v svetu pota, ki so kriva, naj jim zaradi tega ne sledi. Modri pravijo, da življenje res je kratko.

v5. V svetu vidim to človeško raso, ki se ničvredno ukvarja s hrepenenjem po obstoju. Slabiči jočejo v čeljustih Smrti, saj jih hrepenenje po tem ali onem oddaljuje od osvoboditve.

v6. Le poglej! Premetavajo se v tem, čemur pravijo »moje«, so kot ribe v osušenem potoku z malo vode. Ko to vidiš, le sledi poti kar »ni-moje«, brez ustvarjanja navezanosti do obstoja.

v7. Z opuščenimi željami po obeh skrajnostih, z jasnim razumevanjem kontakta, brez pohlepa po ničemer, brez aktivnosti, ki bi si jih kasneje očital, se modri ne oklepa ničesar, kar vidi in sliši.

v8. Modri, s popolnim razumevanjem zaznave, bo prečkal to poplavo: ni pogreznjen v posedovanje, z izpuljenim trnom in vedrim srcem, si ne želi ne tega, ne drugega sveta.

\textbf{3 }

\textbf{Osemverzno učenje o nepoštenosti }

v1. Razpravljajo tisti z nepoštenimi nameni, razpravljajo tudi tisti z iskrenimi nameni. A modrijan se ne vpleta v nobeno razpravo, on je človek, ki ni v nič vezan.

v2. Le kako naj nekdo preseže svoje poglede, ko pa ga želje zapeljejo in jim nato sledi ter si popolnost ustvarja kakor mu godi? Kakor razume, tako tudi razpravlja.

v3. Če kdorkoli, čeprav ni vprašan, govori drugim o svoji lastni morali in navadah ter sam govori o samem sebi, temu mojstri pravijo, da je na slabi poti.

v4. Toda umirjen menih, ki je dosegel \emph{nibbāno}, se ne hvali o svoji morali: »Jaz sem tak«. Za tistega, za katerega ne obstajajo posvetne časti --- temu mojstri pravijo, da je na plemeniti poti.

v5. Kogarkoli ideje, ki so oblikovane, pogojene in postavljene v ospredje, niso brezhibne. Za tistega, ki vidi v tem korist za sebe, je njegov mir odvisen od nestabilnosti.

v6. Res ni lahko opustiti utrjenih pogledov, ki so le teorije prisvojene med idejami. Človek med temi utrjenimi pogledi zavrača in sprejema le različne ideje.

v7. Za osvobojenega človeka na tem svetu ni izoblikovanih pogledov na tak ali drugačen obstoj. Osvobojen človek je opustil prevaro in ponos --- saj le po čem bi se lahko ravnal, ko pa ni v nič zavezan?

v8. Brez dvoma, če je nekdo zavezan idejam, je zavezan tudi v razprave. A če ni zavezan, o čem in kako naj bi razpravljal? Za njega ni ničesar pridobljenega niti zavrženega; celo tukaj se je že otresel vseh pogledov.

\textbf{4 Osemverzno učenje o čistosti}

v1. »Vidim čistost, univerzalnost, neškodljivost. V svojih pogledih je človek popolnoma čist.« S takim razumevanjem in s spoznanjem »univerzalnosti« se ta »videc čistosti« zanaša na nič drugega, kot na znanje.

v2. Če je človek »čist« s svojimi pogledi in mislimi, če je z znanjem opustil trpljenje, potem je »očiščen« z nečim posebnim. Resnično, takšnega človeka izdaja njegov pogled.

v3. Sveti človek ne pravi, da je čistost nekaj pogojenega ali kar je videno, slišano, občuteno ali kar govori morala in običaj. Tak človek ni umazan z zaslugami ali zlom, za sabo je pustil vse pridobljeno in ničesar novega si več ne ustvari.

v4. Ko ljudje opustijo, kar je bilo, se držijo tega, kar sledi. Premaga jih navezanost, ker sledijo svoji strasti: sprejemajo in opuščajo, kakor opice, ko spuščajo veje se jih že naslednje oprijemajo.

v5. Človek, ki je sprejel religiozne običaje, je kdaj potrt, kdaj vesel, ker se trdno oklepa zaznav. Toda mojster, ki je s spoznanjem odkril Pot, ni nikoli potrt ali vesel, ker je njegovo razumevanje jasno.

v6. On je v miru med vsemi idejami in v vsem, kar je videno, slišano ali občuteno. Okrog hodi bister in odprt --- le s čim na svetu bi se ga lahko sodilo?

v7. Ti se ničesar ne oklepajo, ničesar ne dajejo v ospredje, in ne razpravljajo: »To je absolutna čistost.« Z razvozlanim vozlom lastništva, so popolnoma brez posvetnih želja.

v8. Sveti človek je šel preko meja --- za njega ni ničesar tam, spoznal in uvidel je vse, kar je prisvojeno. On je brez strasti, je ravnodušen do brezstrastja, zanj ničesar zunanjega ni njegovo.

\textbf{5 Osemverzno učenje o univerzalnosti}

v1. Tisti, ki živi v svojih pogledih: »To je univerzalnost«, ceni eno znanje iz sveta kot najboljše; za vsa ostala znanja pravi, da so »nižja« znanja. Ta zagotovo še ni šel preko prepirov.

v2. Kakršnokoli korist vidi za sebe v tem, kar je videno, slišano, občuteno, v morali in običajih, poveličuje in vidi le to, vse ostalo vidi kot manjvredno.

v3. Ampak mojstri pravijo, da je v tem skrita past, saj je tako videnje odvisno od »nižjih« videnj. Torej, od vsega kar je videno, slišano ali občuteno, in od vsega, kar izhaja iz morale in običajev, naj menih ne bo odvisen.

v4. Poleg tega, se naj ne oklepa pogledov na svet, ki so osnovani na znanju, kakor tudi na morali in običajih. Sebe naj ne predstavlja kot enakega, niti ne slabšega niti boljšega.

v5. Ko bo opustil pridobljeno in se ne bo oklepal na novo, ne bo odvisen niti od znanja. On resnično ni privrženec učenjakov in ne popušča nobenemu pogledu.

v6. Brez težnje do obeh skrajnosti --- za ta ali drugi obstoj, za tukaj in zdaj ali v bodoče --- za njega ni nobenih utrjenih pogledov, ki izhajajo iz različnih ideologij.

v7. Za njega, tu, v vidnem, slišnem ali čutnem, ni prirejena niti najmanjša zaznava. Ta sveti človek si pogledov ne prisvaja --- le s čim na svetu bi se ga lahko sodilo?

v8. Ničesar si ne izmišljuje, nič si ne postavi na svojo pot; ne sprejema nobene miselnosti. Sveti človek se ne pusti zapeljati morali in običajem, ko stopi na drugi breg, se ne vrne več nazaj.

\textbf{6 Učenje o starosti}

v1. Resnično, to življenje je malenkostno. Človek umre, še preden doživi sto let. Pa tudi, če živi dlje, zagotovo umre od onemoglosti.

v2. Ljudje objokujejo nad tem, kar imenujejo »moje« --- resnično, trajno imetje ne obstaja. S spoznanjem, da so ločitve neizogibne, se naj človek izogne družinskemu življenju.

v3. Ob smrti izgine vse, v kar človek je verjel, da je njegovo. Ko pametni človek uvidi to spoznanje, moj učenec, ta ne bo naklonjen posedovanju.

v4. Tako, kot prebujeni človek nikoli več ne sreča tistega iz sanj, tako tudi mi ne vidimo več naših ljubljenih, ki so umrli, preminuli.

v5. V vsakdanu srečujemo in slišimo ljudi, ki jih poznamo pod različnimi imeni --- za mrtvim bo ostalo samo ime, po katerem se ga bomo spominjali.

v6. Tisti, ki si lastijo in posedujejo, so v oblasti žalosti, tegobe in strahu pred izgubo. Zato so modreci, ki so zapustili imetje, postali redovniki in si našli zatočišče.

v7. O menihu, ki živi odmaknjeno, ki se zateka v samotna bivališča, pravijo, da to ustreza njegovi naravi ter da se ne bo nikoli ustalil v trajnem bivališču.

v8. Modrijan je povsod neodvisen: ničesar si ne ustvarja kot ljubljeno ali neljubljeno ter žalovanje in strah pred izgubo se ga ne oprimeta, tako kot voda lista ne.

v9. Tako kot kaplja vlage na vodni liliji in voda na lokvanju spolzita, tako tudi modrijan ni omadeževan s tem, kar je videl, slišal ali čutil.

v10. Prečiščen človek si ničesar ne predstavlja, o tem kar je videl, slišal ali čutil. Ne želi si pridobiti čistosti preko drugih stvari, ni več strasten niti ravnodušen.

\textbf{7 }

\textbf{Učenje Tissu Matteyi }

v1. \emph{Za iskalca Dhamme, ki se predaja spolnosti, povejte nam, o gospod, kako mu to škoduje? Po poslušanju tvojega učenja, bomo šli vaditi v samoto.}

v2. Tisti, ki se predaja spolnosti, Metteyya, je pozabil vse učenje. Tako tava po napačni poti -- in to je v njem nizkotno.

v3. Kdorkoli je prej živel sam zase in se sedaj predaja spolnosti, je kot vozilo brez nadzora --- takemu človeku se »slabič« v svetu pravi.

v4. In ne glede na to, kakšno slavo in ugled je osvojil, vse je izgubljeno. S tem razumevanjem naj sedaj vadi tako, da popolnoma spolnost opusti.

v5. Zatiran z mislimi, tava kot nesrečnik. Ko sliši pridige drugih ljudi, postane tak človek zmeden.

v6. In ko ga potem karajo, si izmisli orožja obrambe. Eno izmed njih je pohlep in z njim drvi v laži.

v7. Prej poznan kot pameten, odločen za samotno življenje, a ko se preda spolnosti, kot idiot potegnjen je v težave.

v8. Z razumevanjem teh slabih dejanj, se modrijan slej ko prej, odločen za svoje samotno življenje, ne bo več vdajal spolnosti.

v9. Vadil bo v samoti --- to je odličnost plemenitih. A zaradi tega ne bo bil vzvišen -- on resnično je \emph{nibbāni} blizu.

v10. Človeštvo, ki je zavezano čutnim užitkom, zavida modrijanu, ki živi v praznini, ki ne upa na čutne užitke in ki je prečkal to poplavo.

\textbf{8 Učenje Pasūri}

v1. »Samo tukaj je čistost« --- tako trdijo. Pravijo, da čistosti ni v drugih ideologijah, da je Dobro le v tem, v kar oni zaupajo --- tako so prepričani vsak v svojo resnico.

v2. Z željami po prepirih se zapodijo v zborovanja, nasprotniki, drug drugega imajo za bedaka. Navezani na debate razpravljajo med seboj, častihlepni trdijo, da so mojstri.

v3. V razpravi sredi zborovanja, želeč pohval, skrbi ga, da bo poražen. In ko je ovržen, je otožen, razburjen zaradi graje, išče šibkosti.

v4. Ko sodniki povedo, da je njegov dokaz pomanjkljiv in izpodbit, se s slabimi argumenti objokuje in žalosti; »Premagal me je,« se joče.

v5. Taki prepiri nastajajo med misleci, med njimi so le zmagovalci in poraženci. S tem razumevanjem naj se vsak izogiba prerekanjem, saj v njih ni drugega namena, kot gonja za slavo.

v6. A če je pohvaljen sredi zborovanja ob razglasitvi svojih argumentov, je pri tem prešeren in samozadovoljen, saj je dosegel svoj namen.

v7. Toda vsako samozadovoljstvo je vir lastne škode, saj se prereka z domišljavostjo in zaničevanjem. Ko človek to razume, naj se ne prepira --- mojstri pravijo, da v tem zagotovo ni čistosti.

v8. Kot junak, nahranjen s kraljevsko hrano, človek kot grom vneto išče si nasprotnika. Pobegni, kjerkoli se pojavi on, junak. Tu ni ničesar zate, kar je vredno boja.

v9. Tisti, ki so si prisvojili poglede, prepirajoč trdijo: »Le to je resnica.« A ti jim reci, da zate ni nihče nasprotnik, ne glede na to, do kakšnih prepirov je prišlo.

v10. Tisti, ki so prenehali z bojem, ne izpodbijajo pogleda s pogledom. Kaj bi rad pridobil od njih, Pasūra, ko pa si ti ničesar ne lastijo?

v11. In sedaj si prišel sem razpravljat pod vplivom notranjega prepričanja. A stopil si v kontakt s prečiščenim človekom in na tak način res ne boš mogel napredovati.

\textbf{9 }

\textbf{Učenje Māgaṇ{[}1E0D?{]}iyi }

v1. Spoznal sem hrepenenje, nezadovoljstvo in strast, ne čutim več niti želja do spolnosti --- le kaj je to, polno urina in gnoja? Tega se niti s svojo nogo ne bi hotel dotakniti.

v2. \emph{Če si ne želite zaklada, kot je ta, ženske, ki je zaželena med mnogimi gospodi, kakšno prepričanje, moralo, običaje, način življenja, in o kakšnem posmrtnem bivanju govorite?}

v3. Tu ni ničesar, čemur bi lahko rekel: »To razglašam«, Māgandiya, teorije, ki so bile prisvojene med ideologijami. Vendar, ko sem brez oklepanja na njih opazoval poglede, sem z razumevanjem spoznal notranji mir.

v4. \emph{Katerakoli teorija se je oblikovala, resnično, Modrijan, govorite o tem brez oklepanja. Ta »notranji mir«, karkoli že to pomeni, kako je ta spoznan med modrijani?}

v5. Ne s tem, kar je bilo videno ali slišano, niti ne z znanjem, Māgandiya, ne z moralo in običaji; reče se, da čistost je; tudi ne z odsotnostjo tega, kar je videno ali slišano, niti ne z neznanjem, brez morale in običajev -- tudi s tem ne. Z opuščanjem tega, brez odvisnosti od česarkoli drugega, miren človek, neodvisen, ne bo hrepenel po obstoju.

v6. \emph{Če praviš, da se o čistosti ne govori s tem, kar je videno ali slišano, niti ne z znanjem, ne z moralo in običaji; tudi ne z odsotnostjo tega, kar je videno ali slišano, niti ne z neznanjem, ne brez morale in običajev -- tudi s tem ne; potem si predstavljam, da je to res zmedena ideologija. Nekateri se zanašajo na čistost s pogledom.}

v7. Zaradi takega spraševanja, ki je odvisno od pogledov, Māgandiya, si postal zbegan s tem, ko si sebi prisvojil domneve. Zato ne vidiš niti najmanjšega smisla v vsem tem in vse to imaš za zmedo.

v8. Kdor se ima za enakega, boljšega ali slabšega, se bo boril glede na te oznake. Tistemu, ki ne okleva med temi tremi razlikami, se ne pojavljajo oznake »enak« ali »boljši«.

v9. Čemu bi lahko ta sveti človek pravil: »To je resnica«, ali se prepiral: »To je napačno«? Za takega ni niti enakosti niti neenakosti --- le s kom bi se lahko prepiral?

v10. Modrijan, ki je zapustil dom in je brez stalnega bivališča, si ne gradi intime med ljudmi, osvobodil se čutnih je užitkov, ničesar si ne postavlja v ospredje, za nič se ne zavzema in ne razpravlja z ljudmi.

v11. Od vseh stvari se je odrešil, med tem, ko še živi v tem svetu, veliki človek se jih ne oklepa in se ne prepira. Tako kot beli lokvanj, čigar pecelj raste iz vode ni umazan z vodo in blatom, tako tudi modrijan, govornik miru, človek brez pohlepa, ni umazan s čutnostjo in svetom.

v12. Tisti, ki je spoznal resnico v vidnem ali čutnem, ni zato ošaben, saj se s tem ne enači. Ne vodi ga, kar je bilo ustvarjeno ali naučeno, in ne dela si zaključkov o tem, kar se je že uveljavilo.

v13. Tu ni vezi za tistega, ki je nenavezan na zaznave, tu ni zmedenosti za tistega, ki je osvobojen z razumevanjem. Toda tisti, ki se močno oklepajo pogledov in zaznav, pohajkujejo po svetu in povzročajo probleme.

\textbf{10 }

\textbf{Učenje o pred razpadom }

v1. \emph{S kakšno vizijo, s kakšno moralo se lahko komu pravi, da je v miru? Povejte mi to, o Gotama. Sprašujem vas o najboljšem človeku.}

v2. S prenehanjem hrepenenja pred razpadom, ni odvisen od preteklosti, ni pogojen v sedanjosti, si ta ničesar ne daje v ospredje.

v3. Brez jeze in strahu, brez hvale in skrbi, jasen govornik ni domišljav, pravi modrec, ki se obvladuje v govoru.

v4. Brez navezanosti na prihodnost, ne obžaluje preteklosti. Vidi osamo med vsemi kontakti, ničesar med pogledi ga ne zapelje.

v5. Je odmaknjen, ni spletkar, ni pohlepen in se ne boji izgube, ni drzen, je brez odpora in ni naklonjen žalitvam.

v6. Se ne predaja prijetnostim, ni naklonjen zaničevanju, je blag in bistrega duha, ni pobožen niti ni pasiven.

v7. Ne vodijo ga želje po koristih in ni razočaran, če le teh ni. Hrepenenje ga ne ovira, niti ni lakomen za okusne draži.

v8. Za ravnodušnega in pozornega opazovalca, ki se nima za enakega, za boljšega ali slabšega --- za njega ni časti.

v9. Za njega ni odvisnosti, z razumevanjem Poti je neodvisen. V njem ne obstaja hrepenenje niti po obstoju niti po ne-obstoju.

v10. Takemu pravim, da je v miru: takem, ki si ne želi čutnih užitkov, v katerem ni najti nobenih vezi, saj je za seboj pustil vse navezanosti.

v11. Zanj ne obstajajo sinovi ali živina, niti polja ali lastnina. Skratka, pri njem ni ničesar, kar bi lahko pridobil ali zavrgel.

v12. S tem, s čimer bi ga lahko nevedni ljudje kritizirali, ali tudi misleci in sveti možje, on ničesar ne daje v ospredje, zato je tudi sredi kritike miren.

v13. Je brez pohlepa in se ne boji izgub, takšen modrijan se nima za nadrejenega, niti za enakega niti za podrejenega. Ni zapadel v predstave; on je brez predstav.

v14. Tistemu, ki ničesar ne poseduje, ki se ne žalosti, ker ničesar nima, in ki ne sledi nobenim idejam --- le takemu se lahko resnično pravi, da je v miru s seboj.

\textbf{11 }

\textbf{Učenje o prepirih in sporih }

v1. \emph{Od kod izvirajo prepiri in spori, žalostinke in skrbi, skupaj s strahom pred izgubo, od kod domišljavost in prezir, skupaj z žalitvami? Od kod vse to izvira? Prosimo, povejte nam to.}

v2. Prepiri in spori nastanejo, ko jemljemo stvari, kot nam drage, enako tudi žalostinke in skrbi, skupaj s strahom pred izgubo, kot tudi domišljavost in prezir, skupaj z žalitvami. Prepiri in spori so združeni s strahom pred izgubo, in temu sledijo žalitve.

v3. \emph{Na čem je osnovano to, da si jemljemo stvari, kot nam drage, na čem je osnovano hrepenenje, ki se pretaka v svetu?} \emph{In na čem so osnovani upi in cilji, ta človekova prihodnja stanja?}

v4. To, kar nam je drago, je osnovano na željah, kakor tudi vsako hrepenenje, ki se pretaka v svetu. In na tem so osnovani tudi upi in cilji, ta človekova prihodnja stanja.

v5. \emph{A na čem so osnovane želje? In} \emph{od kod izhajajo teorije, jeza, laži, in negotovost,} \emph{ter razne ideologije, ki jih razglašajo misleci?}

v6. Želje so osnovane na tem, kar prinaša užitek in neužitek v svetu. Z videnjem obstoja in ne-obstoja materije, si človek ustvarja teorije o svetu.

v7. Jeza, laž, negotovost in ideje so tu, ko je prisotna ta dvojnost. Negotovi človek naj zato vadi na poti razumevanja, s pomočjo razumevanja spoznanj mislecev.

v8. \emph{V čem je osnovano užitek in neužitek? Kaj mora biti odsotno, da to ne bi več obstajalo?} \emph{In tudi obstoj in neobstoj, karkoli že to pomeni, povejte mi, v čem je vse to osnovano?}

v9. V kontaktih sta osnovana užitek in neužitek; če ni kontakta, ti občutki ne obstajajo. In tudi obstoj in neobstoj, karkoli že to pomeni, povem ti, da sta osnovana prav v tem.

v10. \emph{Na čem je na svetu osnovan kontakt? Kaj mora biti odsotno, da ne obstaja »moje«? In iz česa izhaja imetje?} \emph{Kaj mora izginiti, da kontakt se ne sklene?}

v11. Kontakt je pogojen od imena in materije. Če materija izgine, se kontakt ne sklene. Imetje je osnovano v željah. Ko ni prisotne želje, ni potrebe po »mojem«.

v12. \emph{Kaj se more doseči, da materija izgine? In kako lahko nezadovoljstvo in zadovoljstvo izgine?} \emph{Povejte mi, na kakšen način stvari izginejo, saj rad bi razumel prav to.}

v13. Ko nima zaznave o zaznavi; ko nima zaznave o nezaznavi; ko ni brez zaznave; ko nima zaznave »izginitve«. Za tistega, ki to doseže, materija izgine, saj je zaznava osnova za uveljavljeno obsedenost.

v14. \emph{To, kar smo vas vprašali, ste nam pojasnili. Toda, naj vas vprašamo še nekaj --- prosimo, povejte nam ---} \emph{ali je res, da nekateri pametni ljudje razpravljajo, da je to najvišja čistost duha le do te mere,} \emph{in ali ti razpravljajo, da je tu še nekaj več?}

v15. Res je, da nekateri pametni ljudje razpravljajo, da je to najvišja čistost duha. In nekateri mojstri pravijo, da je to le začasno --- da imajo znanje o ugasnitvi.

v16. Modrec razume, da je vse to odvisnost, raziskovalec razume naravo odvisnosti. Z razumevanjem je osvobojen in se ne prepira. Modri človek ne sprejema nikakršnega obstoja.

\textbf{12 }

\textbf{Kratko učenje o slepi ulici }

v1. \emph{Vztrajajoč v svojih pogledih, »mojstri« razpravljajo o različnih argumentih:} \emph{»Le s tem znanjem lahko spoznaš resnico. Če kdo to zanika, ta ni popoln.«}

v2. \emph{In s takimi argumenti razpravljajo in govorijo: »Drugi je bedak, ne mojster«.} \emph{Toda kateri med argumenti je pravi? Seveda, vsi zase trdijo, da so mojstri.}

v3. Da je tisti, ki ne soglaša z idejami drugega, bedak z nižjim razumevanjem, potemtakem so \emph{vsi} bedaki z nižjim razumevanjem, saj vztrajajo v svojih pogledih.

v4. Toda če so vsi ti z lastnimi pogledi, pa čeprav niso brezmadežni, mojstri z uvidom in s čistim razumevanjem, potem noben med njimi ni slabšega razumevanja, saj prav vsi lahko postanejo popolni le z lastnimi pogledi.

v5. Toda jaz ne trdim: »Tako je«, kot si nasprotujejo bedaki. A ker vsak vidi le svoj pogled kot resnično pravi, vidijo drugega v drugem bedaka.

v6. \emph{Temu, kar nekateri pravijo, da je »pravilno«, »tako je«, spet drugi pravijo, da je »zaman«, da je »napačno« in v takšnem prepričanju se ti prepirajo. Zakaj misleci ne razodenejo zgolj ene resnice?}

v7. Obstaja le ena resnica in ne mnogo. Če bi človek to vedel, se seveda ne bi prepiral. Toda misleci razpravljajo o različnih resnicah, zato ne razodenejo le ene.

v8. \emph{Zakaj razpravljajo o različnih resnicah, in prepirljivo trdijo, da so mojstri? Slišati je mnogo različnih resnic ali pa samo sledijo drugi špekulaciji?}

v9. Mnogovrstne resnice v svetu niso trajne, razen, ko si kaj predstavljamo. In ko jim uspe izmisliti si novo špekulacijo, govorijo le o dvojnosti: »resnice« in »laži«.

v10. Odvisnost od vsega tega, kar je videno, slišano, začuteno, z moralo in običaji, on prezira. Ko je prepričan v svojo teorijo, se posmehuje drugim in pravi: »Ta je bedak in ne mojster.«

v11. Medtem ko drugega vidi kot »bedaka«, o sebi govori, kot da je »mojster«. In ko o sebi trdi, da je mojster, s tem istočasno prezira druge.

v12. Le s svojimi pretiranimi pogledi je on »popoln«, ter pijan od domišljavosti ima sebe za dovršenega. Sam sebe je blagoslovil in s tem tudi svoje poglede.

v13. Če besede delajo osebo slabo, potem je slabo tudi njeno razumevanje. A če modrijan sam zase spozna resnico, potem ni med misleci nihče bedak.

v14. »Tisti, ki razpravljajo drugače od nas, niso uspeli v čistosti in so nepopolni« --- le sektaši vsak po svoje tako razpravljajo, resnično razvneti v strasti za svojimi pogledi.

v15. »Samo v tem je čistost,« trdijo; pravijo, da čistosti ni v drugih idejah. Le sektaši so vsak po svoje zakoreninjeni, ko tako odločno razglašajo svoje poglede.

v16. Vendar, ko tako odločno razglašajo svoje poglede, le katerega bi lahko jemali za bedaka? On sam lahko izzove zgolj konflikt, ko drugega z drugačnimi idejami razglasi za bedaka.

v17. Dokler se človek drži teorij, povzdiguje zgolj sebe in nadaljuje s prepiri v svetu. A ko zapusti vse te teorije, več ne povzroča sporov v svetu.

\textbf{13 }

\textbf{Daljše učenje o slepi ulici }

v1. \emph{Vsi, ki bivajo v pogledih, prepirljivo govorijo: »Le to je resnica« ---} \emph{so vsi taki kritizirani ali so lahko tudi pohvaljeni?}

v2. Taka pohvala je premajhna za spokojnost. Toda naj vam povem o dveh sadovih prepirov. S tem videnjem se naj nobeden ne prepira, in naj bo to temeljno zatočišče za mir.

v3. Ne glede na vrsto dogovora, za izkušenega človeka ni potrebe, da se vpleta v njih. Le zakaj bi se vpletal, če v njem ni naklonjenosti do vsega, kar je videl ali slišal?

v4. Ko mislijo, da je morala najpomembnejša in da je čistost v samoomejevanju, sprejmejo svoje običaje in se jim poklonijo: »Naj vadimo prav tu in zdaj, saj v tem je čistost!« Tisti, ki mislijo, da so mojstri, se zgolj ohranjajo v bivanju.

v5. A ko prekrši svojo moralo in običaje, je razburjen, ker mu ni uspelo v svojih dejanjih. Hrepeni in si prizadeva za popolno svobodo od vsega nizkotnega, kot tisti, ki je izgubil svojo karavano in je daleč od doma.

v6. Vendar le z opuščanjem morale in običajev in kritiziranih ali odobravajočih dejanj, brez želja po »čistosti« ali »ne-čistosti« živel bo zadržan, celo na mir se ne bo vezal.

v7. Odvisni od nevedne teme ali od tega, kar je bilo zavrnjeno z odporom, ter odvisni od tega, kar je videno, slišano ali občuteno, brez osvobojenih želja po različnih oblikah obstoja, vzdihujejo po čistosti v onostranstvu.

v8. Če je kdo čemurkoli predan, je to delo hrepenenja; in kjer je prisotna vznemirjenost, je to zaradi oklepanja. Ampak za tistega, ki ne premine in se ponovno ne pojavlja, zakaj bi bil razburjen? Le po čem bi hrepenel?

v9. \emph{Prav tej ideji, ki ji nekateri pravijo »univerzalna«, ji pravijo drugi, da je »pomanjkljiva«.} \emph{Kateri od teh je pravi argument? Seveda, vsi trdijo, da so mojstri.}

v10. Pravijo, da so njihove ideje popolne, ideje drugih pa pomanjkljive. S takimi sprejetimi stališči se prepirajo, vsak zase pravi, da je njegovo stališče pravilno.

v11. Če je zaradi zaničevanja nekdo pomanjkljiv, potem med temi »duhovnimi ljudmi« ni nihče izjema. Vsak zase razpravlja, da so ideje drugih pomanjkljive, medtem ko se vsi odločno držijo svojih.

v12. A če bi bilo čaščenje njihovih lastnih učenj tako resnično, kot sami hvalijo svoje, potem bi bile vsakogar trditve pravilne, saj bi bila čistost za vsakega posameznika osebna resnica.

v13. Toda za svetega moža ni ničesar, kar bi ga lahko vodilo: nobene teorije, ki so prisvojene med idejami. Tak človek je prerasel prepire, v nobeni od ponujenih idej ne vidi vrednosti.

v14. »Razumem, vidim, to je to!« --- nekateri se tako zanašajo na čistost pogledov. Pa tudi, če je kaj uvidel, le kakšno korist ima od tega? Saj, ko se naslednjič pregrešijo, zopet pravijo, da je čistost nekje drugje.

v15. Kar človek res vidi, je to le ime in materija in s takim gledanjem bo razumel prav to. Naj vidi veliko ali malo, kar si želi, mojstri pravijo, da v tem zagotovo ni čistosti.

v16. Res ni enostavno voditi takega dogmatika, ki si postavlja v ospredje že izoblikovane poglede. Trdi, da je dobrota v tem, od česar je sam odvisen, trdi, da je on govornik »čistosti«, videc »realnosti«.

v17. A svetega moža, se na noben način ne da ovrednotiti: on ne sledi pogledom, ni privrženec znanja in na znanje vsakdanjih navad, ki se jih ljudje držijo, gleda ravnodušno.

v18. Modrijan, ki je razvezal posvetne vezi, se ne vpleta v nastale prepire: je miren med nemirnimi, je ravnodušen opazovalec in ničesar si ne jemlje vase, medtem ko drugi počno prav to.

v19. Ko je za seboj zapustil stare vplive in si ne ustvarja novih, ne deluje z željami. Ni dogmatik, modrec, je prost vseh prepričanj, ni zatopljen v svet in si ničesar ne očita.

v20. Za njega ni ničesar, za kar bi se moral boriti ali biti proti temu, kar je videl, slišal ali občutil. On, modrijan, je odložil svoje breme in je svoboden, brez novega oklepanja, brez inspiracij, s prenehanjem dejanj.

\textbf{14 }

\textbf{Učenje o hitrosti }

v1. \emph{Sprašujem sorodnika sonca, Njegovo svetost, o samoti in o stanju miru:} \emph{s kakšnim védenjem je lahko menih ugasnjen, da se ne bi vezal na nič v tem svetu?}

v2. Popolnoma naj bi prenehal misliti: »Jaz sem«, to celotno korenino uveljavljene obsedenosti. Ne glede koliko hrepenenja je še v njem, bo v popolni pozornosti vadil to razrešitev.

v3. Katerekoli ideje, ki jih pozna, ne glede če te izvirajo od znotraj ali prihajajo od zunaj, z njimi si ne bo utrjeval stališča, saj krepostni tega ne bi imenovali »mir«.

v4. Ne bo se imel za boljšega, ne za slabšega niti ne za enakovrednega. Čeprav dotaknjen z mnogimi stvarmi, ne bo zagovarjal misli o sebi.

v5. Le v sebi lahko pride do miru, menih ga ne bo iskal v zunanjem svetu. Za tistega, ki je v miru sam s seboj, ni ničesar za pridobiti, kaj šele za zavreči.

v6. Tako, kot je sredi morja popolnoma mirno in ni nobenih valov, tako tudi brez hrepenenja človek mirno biva --- tak menih si ne bo jemal časti.

v7. \emph{Čigar oči so odprte z odloženimi težavami, ta je razložil Dhammo, kot jo je sam spoznal.} \emph{Častiti gospod, povejte nam o poteku napredka, o etični dolžnosti in tudi o koncentraciji.}

v8. Menih ne dovoli, da so njegove oči nemirne, svoja ušesa zapre pred vaškimi govoricami, ni požrešen na okuse in ničesar v svetu ne jemlje, kot »to je moje«.

v9. Kadarkoli je menih v stiku z neprijetnostmi, se ne predaja žalovanju. Ne hrepeni po obstoju, niti ni pretresen zaradi strahu.

v10. Hrano in pijačo, živila in tudi oblačila --- vsa ta imetja ne bo imel za zaklad niti se ne bo bal njihovega pomanjkanja.

v11. Meditant, ki se brez želja po tavanju naokrog vzdrži obžalovanih dejanj, je pazljiv. Kjerkoli ima namen sedeti ali ležati, menih naj živi v kraju z malo motenj.

v12. Naj ne spi predolgo, marljiv v opreznosti naj vztraja v budnosti. Tako bo zapustil vse: lenobo, iluzije, smeh, igre, spolnost in vse njihove izpeljanke.

v13. Ritualno zdravljenje naj ne bo njegova praksa, niti ne razlaga sanj, tolmačenje znakov in astrologije. Učenec ne bo tolmačil živalskih krikov, zdravil neplodnost ali druge bolezni.

v14. Menih naj ne bo pretresen zaradi kritik, niti ne vzvišen zaradi pohval. Pregnal bo celotno hrepenenje, skupaj s strahom pred izgubo, jezo in žalitve.

v15. Menih naj ne kupuje in ne prodaja, nobenih kritik naj ne raznaša, naj ne bo nadloga med ljudmi in laskajoč z željami po koristi.

v16. Menih naj se ne hvali, niti naj ne izreka besed s skritimi motivi, naj ne vadi v predrznem obnašanju, in naj se izogne govorom o spornih temah.

v17. Naj se ne predaja lažem; tako s popolno pozornostjo ne bo zlorabljal zaupanja. Hkrati naj nobenega ne prezira zaradi njegovega življenja, razumevanja, morale ali običajev.

v18. Ko sliši mnogo besed in je izzvan od mislecev in nevednih ljudi, nazaj naj jim ne odgovarja ostro, saj si tisti z vrlinami ne dela sovražnikov.

v19. Menih bo s takim razumevanjem Dhamme, vadil v skladu s proučevanjem, s pozornostjo in z razumevanjem stanja prenehanja kot miru, tako ne bo zanemarjal Gotamovega učenja.

v20. On je resnično osvajalec neosvojenega, uvidel je resnico, ki temelji na izkušnji in ne na govoricah. Zato se vedno spoštljivo pokloni in pazljivo vadi po vodilu Blaženega.

\textbf{15 }

\textbf{Učenje o nasilju }

v1. Strah nastane tam, kjer je nasilje --- le poglej ljudi v sporu! Naj vam povem, kaj sem občutil, kakšna tesnoba me je obšla nekoč.

v2. Vidim, kako se človeštvo premetava, kot ribe v plitki vodi tekmujejo med sabo --- ko sem videl to, me strah je prevzel.

v3. Svet mi je bil povsem brez pomena, treslo se je iz vseh strani. Želel sem si najti zavetje, a je bilo vse že zasedeno.

v4. Naseljenost je bila polna nasprotovanja. Ko sem to videl, so me obšli tesni občutki. A potem sem našel pravi trn, skoraj neopaznega, zataknjenega v srcu.

v5. Ko sem bil obremenjen s tem trnom, sem tekal na vse strani. A ko sem ga izdrl, nisem nič več tekal, niti se več utapljal.

v6. \emph{Naj zdaj tu recitiramo o vadbi:} Kjerkoli so posvetne vezi, naj človek od njih ne bo odvisen. In ko se prebije skozi vse vrste čutnih užitkov, naj vadi samo-ugasnitev.

v7. Naj bo resnicoljuben, ne drzen, naj ne ustvarja iluzij in naj nikogar ne žali, naj bo brez jeze, kot modrec, ki je prekoračil zlo hrepenenja in mnogovrstne želje.

v8. Naj premaga zaspanost, dolgčas in lenobo, naj se ne ukloni lahkomiselnosti, naj ne vztraja trmasto v ponosu, v sebi naj bo odločen za \emph{nibbāno}.

v9. Naj ne zaide v laži, v materialnem si naj ne gradi želja, naj dobro pozna naravo domišljavosti ter naj se izogiba nasilju.

v10. Naj se ne veseli starega, naj se ne ukloni novemu, naj ne žaluje po tem, kar je bilo izgubljeno ter naj se ne ujame v to, kar se blešči in sije.

v11. Pohlepu pravim »velika poplava«, hrepenenju pravim »tok« in iskanje poživil s čutnim poželenjem, je kot blato, ki se ga je težko otresti.

v12. Brez odstopanja od resnic, modrec, sveti človek, stoji na visokih tleh. In ko je opustil vse stvari, biva v svojem miru.

v13. On je resnično človek, ki je spoznal resnico, on je tisti, ki ve, postal je neodvisen z razumevanjem Dhamme. S tem znanjem hodi skozi svet; nikomur ničesar ne zavida.

v14. Kdorkoli je prešel čutna poželenja --- ta navezovanja v svetu za premagat težka --- je brez žalosti in brez skrbi, je prekinil tok, je brez vezi.

v15. Karkoli je bilo, naj zdaj zamre in za njega naj ne bo ničesar, kar naj bi še prišlo. V sedanjosti se naj ničesar ne oprime, saj le tako lahko živi v miru.

v16. Če ničesar ne jemlje in ne doživlja kot »moje« v kakršnemkoli pogledu na ime in materijo, in ko ga ne žalosti to, česar ni, ta resnično ne trpi posvetne izgube.

v17. Tistega, ki ne goji misli »to je zame«, in ki ne misli »to je za druge«, tistega, ki se zaveda nesmiselnosti »mojega«, tega ne žalostijo misli »oh, ničesar ni zame!«

v18. Ni krut, niti ni pohlepen, je vedno v miru brez strasti --- to je tisto, za kar sem vprašan, čemur pravim prava korist.

v19. Ko je brez poželenja in ima znanje, nobenih pogojenosti ni več za njega. Vzdržal se je vse sile in za njega ni nevarnosti.

v20. Ne med enakimi, niti ne med nižjimi in ne med višjimi, modrec govori. On je v miru, brez strahu pred izgubo, saj ni tu ničesar, od česar bi jemal slovo.

\textbf{16 }

\textbf{Učenje Sāriputti }

v1. \emph{Nikoli poprej nisem videl, niti od drugih slišal,} \emph{o učitelju, ki tako lepo govori, ki je prišel sem v družbi zadovoljnih.}

v2. \emph{Kot svet s svojimi bogovi vidi tega Vidca,} \emph{ki je razblinil vso temo, ki je sam zase prišel do zadovoljstva,}

v3. \emph{k temu Razsvetljenemu, neodvisnemu, ki je takšen kot je, nespletkarskemu, ki je prišel s to skupino,} \emph{v imenu tistih tukaj, ki so še navezani, k temu sem pristopil z vprašanjem:}

v4. \emph{Za meniha, ki je potrt in biva v praznem bivališču} \emph{ali ob vznožju drevesa ali na grobu ali v jamah ali v gorah}

v5. \emph{ali v drugih bivališčih, katero mero strahu mora menih prenesti,} \emph{ne da bi se vznemiril v svojem tihem bivališču?}

v6. \emph{Koliko je tistih posvetnih nevarnosti za nekoga, ki gre proti neznanem kraju,} \emph{ki bi jih moral menih zlahka preseči v svojem odmaknjenem bivališču?}

v7. \emph{Kakšen naj bi bil njegov govor? Kako naj v sebi žanje polja?} \emph{Kakšna naj bodo morala in običaji za meniha, ki je osredotočen vase?}

v8. \emph{Kako naj vadi tisti, ki je zedinjen, bister, pozoren,} \emph{da lahko odpihne stran svojo nečistost kot kovač, ki oblikuje srebro?}

v9. Kakšno naj bo udobje za potrtega, Sāriputta, ki vadi v praznem bivališču, ki si želi popolno razsvetljenje v skladu z Dhammo --- naj ti povem to, kar je v skladu z mojim razumevanjem.

v10. Modrijan se ne bo bal petih neprijetnih stvari, menih, ki je pozoren in živi v odrekanju: pikov muh, komarjev in drugih žuželk, stikov z ljudmi in tudi štirinožcev.

v11. Ne bo mu neprijetno niti med privrženci drugih idej, čeprav je v njih videl veliko strahu. Človek, ki je iskalec resnice, bo zlahka presegel tudi te probleme:

v12. prenašal bo bolezen in lakoto, mraz in vročino. Ko bo v stiku z mnogimi stvarmi, se bo uril v trdnosti svoje odločnosti.

v13. Ne bo se umazal s krajami, ne bo napačno govoril, njegov dotik bo prijazen do trpečih in šibkih. Vsako prepoznano motnjo v umu bo pregnal z mislijo: »To je temna stran!«

v14. Ne bo padel pod vpliv jeze in zaničevanja, izkopal bo njuno korenino. Zlahka bo presegel vse: kar je prijetno in tudi neprijetno --- on bo resnično spoštovan.

v15. Ker daje prednost modrosti in se veseli pravičnosti, bo premagal te probleme --- v svojem odmaknjenem bivališču bo kos nezadovoljstvu in dvignil se bo nad štiri oblike trpljenja:

v16. »Kaj bom jedel?« ali »Kje bom jedel?« »Zagotovo bom neprijetno spal.« »Kjer naj spim nocoj?« Teh misli, ki povzročajo trpljenje, se učenec, ki ni nikjer nastanjen, ne oprime.

v17. Ko dobiva hrano in oblačila ob pravem času ve, da ga le zmernost vodi v zadovoljstvo. Zavarovan glede teh stvari in obvladan gre po vasi, ne da bi govoril ostro, četudi bi bil izzvan.

v18. S povešenimi očmi, brez želje po pohajkovanju ter predan meditaciji je vedno buden. Z začetno ravnodušnostjo in samoobvladanostjo, bo prenehal s težnjo uma k razglabljanju in skrbem.

v19. Ko je okaran, v pozornosti se veseli, saj je prekinil trmo do kolegov v svetem življenju. Dovoli si le govor, ki je dober in ne predolg in ne posveča pozornosti priljubljenim govoricam.

v20. Poleg vsega tega obstaja v svetu pet onesnaženosti, ki se jim mora pozoren človek z vadbo odpovedati: kos naj bo strastem oblike, zvoka, tudi okusa, vonja in dotika.

v21. Menih s pozornostjo in s popolnoma osvobojenim srcem, zavrne vsako željo po teh čutnih objektih. Pravočasno, pravilno in temeljito preiskuje Dhammo, z zedinjenim umom naj izniči vso to temo.
