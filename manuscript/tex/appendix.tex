\chapter{Zgodovina Aṭṭhakavagge}

V času Buddhovega življenja, učenja niso bila nikoli zapisana. Vsa učenja
so se širila med menihi le z ustnim izročilom. Ko je bila skupnost še
majhna in je živela brez strehe nad glavo, so si menihi med seboj
recitirali verze, ki so jih navdihovali za nadaljnjo vadbo ali v verzih
izražali spoštovanje do Dhamme, ki so jo tudi sami spoznali. Večina teh
verzov, ki so nastajali v tem zgodnejšem obdobju je zbranih v
\emph{Sutta-nipāti}. 

Poleg te, so v kanonu omenjene tudi sledeče zbirke:
\emph{sutta} (govori), \emph{geyya} (pesmi), \emph{veyyākaraṇa}
(razlage), \emph{gātha} (verzi), \emph{udāna} (vzkliki, govori
navdušenja), \emph{itivuttaka} (pregovori), \emph{jātaka} (zgodbe
predhodnih rojstev), \emph{abbhutadhammaṃ} (o čudežnih stvareh) in
\emph{vedalla} (odgovori na vprašanje). Danes poznamo ta besedila pod
sledečimi imeni: \emph{Dhammapada}, \emph{Udāna}, \emph{Itivuttaka,
Sutta-nipāta, Thera-therīgātha in Jātaka}. 

Kasneje, ko se je skupnost
razširila po severni Indiji in je štela na tisoče menihov ter živela
tudi v samostanih, so se učenja širila med menihi, in kasneje tudi med
nunami, v bolj poenostavljeni in sistematični obliki. Daljši govori z
zgodbami, ki so bili namenjeni bolj laičnem poslušalstvu, so zbrani v
\emph{Dīgha-nikāyi} (DN); srednje-dolgi govori, ki so bili namenjeni
laikom in menihom/nunam, so zbrani v \emph{Majjhima-nikāyi} (MN); in
krajši govori, namenjeni predvsem menihom/nunam so zbrani v
kategorizirani obliki glede na temo v \emph{Saṃyutta-nikāyi} (SN) in
glede na številke v \emph{Aṅguttara-nikāyi} (AN). Vse ostale manjše
zbirke, že omenjene zgoraj, so zbrane v \emph{Khuddaka-nikāyi}. Vse
sutte, ki jih poznamo danes, so bile zbrane v te \emph{nikāye} kmalu po
Buddhovi smrti, leta 563 pr.n.št. Tega leta, v monsunskem obdobju, se je
zbralo petsto razsvetljenih menihov v Rajagahi, Indiji, kjer so
preživeli skupaj tri mesece in zrecitirali vse sutte in meniška pravila
(\emph{vināya}), ki so jih imeli v spominu. Tradicija širjenja teh učenj
je potekala z ustnim izročilom in šele leta 80 pr.n.št., ko je budizem
doživljal krizo preživetja, so se menihi na Šrilanki odločili, da ta
besedila tudi zapišejo na palmove liste.

Da \emph{Aṭṭhakavagga} spada med najstarejša besedila, se lahko vidi v
sledečih primerih:

\begin{enumerate}
\def\labelenumi{\arabic{enumi}.}
\item
  Pri \emph{Aṭṭhakavaggi} je posebno to, da vsebuje mnogo redke in stare
  slovnične oblike, ki so podobne tistim v brahmaničnih \emph{vedah,} in
  so skorajda neuporabljene v kanonu. (Veda je bila izoblikovana med
  1500 in 700 pr.n.št.)
\item
  \emph{Aṭṭhakavagga} je omenjena v
  \emph{Soṇa Sutti} (\emph{Udāna} 5:6) in \emph{Soṇakoḷivisavatthu}
  (\emph{Mahāvagga} 5:3). V obeh besedilih je govora o mladem menihu
  Soṇa Kutikaṇṇi, ki je na prošnjo Buddhe zrecitiral celotno
  \emph{Aṭṭhakavaggo}. Buddha je potem pohvalil častitega Soṇo za
  pravilno recitirano besedilo, da je zelo jasno podal to učenje in da
  si ga je zelo dobro zapomnil, brez napak. Buddha je potem imenoval
  tega meniha kot Mahā Soṇa, Soṇa Veliki. Besedilo v \emph{Udāna} tudi
  (pravilno) omenja, da ima \emph{Aṭṭhakavagga} šestnajst delov, kar
  pomeni, da je bila ta zbirka v času Buddhe že dokončno oblikovana.
\item
  \emph{Aṭṭhakavagga} je tudi eden redkih delov kanona, ki ima svojo
  zapisano razlago za vsako vrstico, ki je tudi del kasnejšega kanona.
  Ta razlaga se imenuje \emph{Mahāniddesa} (nima velike veljave ali
  koristi pri interpretaciji \emph{Aṭṭhakavagge}).
\item
  Raziskave kažejo, da je \emph{Aṭṭhakavagga} del kanona tudi v drugih
  starodavnih budističnih šolah, kot je mahāsanghika, ki so zgodovinsko
  povezane s celo starejšo pred-therāvadsko linijo vajjiputtami iz
  drugega meniškega koncila (383 pr.n.št., to je 180 let po Buddhovi
  smrti). Tudi zgodba o Soṇi Kutikaṇṇi se pojavi v mahāsanghiškem kanonu
  in v kanonu drugih starejših šol in v mahāyānski Tripitaki.
\item
  Kot sem že omenil, \emph{Aṭṭhakavagga} ni sistematično in tehnično
  urejena, kot so kasnejše sutte. Poleg tega v njej tudi ni mističnih
  zgodb, ki so karakteristične v nekaterih drugih (in kasnejših) suttah.
\item
  Učenje \emph{Aṭṭhakavagge} naslavlja skupnosti menihov, ki so brez
  prebivališča hodili po indijskem podkontinentu. Verjetno so takrat
  menihi bili bolj naklonjeni k direktnemu razumevanju Dhamme, brez
  utapljanja v številna teoretična besedila, kot je to značilno v
  kasnejšem času. Približno petsto let po Buddhovi smrti se je duh
  \emph{Aṭṭhakavagge} izgubil (oz. ko je na svetu izginilo čisto sveto
  življenje in \emph{saddhamma} ali pravo učenje (AN 8:51)) in menihi so
  dajali več pomena svojim lastnim idejami in interpretacijam. To je bil
  čas, ko sta se liberalna mahāyāna in intelektualna-ortodoksna
  theravāda začeli oddaljevati in se kasneje tudi znotraj sebe
  razcepili.
\end{enumerate}

\emph{Pārāyanavagga}, ki je peti del \emph{Sutta-nipāte}, ima podobne
značilnosti kot \emph{Aṭṭhakavagga} in ima enako vrednost.

\chapter{Razlaga nekaterih izrazov}

Ta uvod ne bo dovolj za poglabljanje v ta učenja, zato bralca vabim, da
išče razumevanje direktno iz sutt. Kljub temu, je nekaj razlag
potrebnih. V teh besedilih bomo naleteli na besede, ki so običajne v
suttah, vendar so morda nejasne v današnjih razlagah. Tu ponujam nekaj
ključnih besed za pravilno razumevanje te zbirke.

\begin{description}

\item[Ime in snov] (\emph{nāmarūpa}) pomeni fenomen oz. objekt,
ki se nam pojavi. Doživljanje samo po sebi je del nas, ne glede ali je
to čustveno ali tiho, občuteno s katerimkoli od petih čutil ali z
mislimi. Dokler obstajamo in dokler so stvari prisotne tu za nas,
pomeni, da \emph{smo} živi, da \emph{doživljamo}. V vsakem doživljanju
je \emph{fenomen}, ki je \emph{prisoten}. Ta prisotnost (ali
prepoznavanje) je zavedanje (\emph{viññāna}); in zavest ali prisotnost
sam po sebi ni ničesar, če ni fenomena. In obratno: fenomen ne more
obstajati, če se tega ne zavedamo. Torej, \emph{nāmarūpa} in
\emph{viññāna} sta si med seboj odvisni.

Poleg tega, ima fenomen (\emph{nāmarūpa}) dve značilnosti:
\emph{subtilnost} in \emph{imenovanje}. Subtilnost je \emph{rūpa} ali
snov -- to so štirje elementi: \emph{zemlja} ali trdnost,
\emph{voda} ali tekočnost, \emph{ogenj} ali dozorelost ter \emph{veter}
ali zračnost. Imenovanje je \emph{nāma} ali ime in pomeni, da je ta
fenomen \emph{prepoznan} s čutenjem (prijetnost, neprijetnost ali
nevtralnost (\emph{vedanā})), z zaznavanjem (vidnih stvari, vonjev,
okusov, zvokov, in telesno občutenih objektov (\emph{saññā})) in z
intencijo oz. tvorjenjem namena in pomena (npr. da je ta stvar za
pisanje in branje, zatorej je to papir (\emph{saṅkhāra})).

\item[Občutki] (\emph{vedanā}) v suttah - ne pomenijo čustva. Ko je
Buddha govoril o občutkih je te opisal le kot prijetnost, neprijetnost
ali nevtralnost in ne kot, na primer, topel, sladek, jezen, zaljubljen,
itd. Čustvo je fenomen, ki se ga ne le čuti, ampak \emph{doživlja} kot
fenomen (torej vsebuje tudi zaznavo, intenco (ali namen), snov in
zavest).

\item[Zaznava] (\emph{saññā}) pomeni, da razpoznavamo oblike, barve,
vonje, zvoke, itd. Ponekod v tej zbirki se lahko to razume tudi kot
»predstavljanje« nečesa, oz. zaznavanje mentalnih podob.

\item[Želje] (\emph{taṇhā} ali \emph{chanda}): Ko govorimo o
\emph{taṇhi}, je to opisano kot želje po čutnosti (\emph{kāma}), bivanju
(\emph{bhava}) in nebivanju (\emph{vibhava}). Dokler smo nevedni, so te
želje vedno prisotne v našem doživljanju in zatorej si ne moremo
izbirati, kdaj so želje prisotne in kdaj ne. Te želje so vir
eksistenčnega problema, t.j. trpljenja. Le z razumevanjem Dhamme, ko
uvidimo, da te »bolj notranje« želje povzročajo težave, lahko začenjamo
to opuščati.

\emph{Chanda}, ki sem jo prevajal z isto besedo »želje«, po drugi
strani pomeni prav to, kako želje razumemo v našem vsakdanu: ali si
želim te stvari ali ne. Na \emph{chande} imamo seveda več vpliva in jih
zato lahko kontroliramo.

\item[Uveljavljena obsedenost] (\emph{papañcasaṅkhā}).
\emph{Papañca} je beseda, ki jo je nemogoče pravilno prevesti. Lahko
hkrati pomeni »razširjenje« ali »razpršenost« in »ovira« ali »zapreka«.
»Obsedenost« je precej poenostavljen prevod. (Ta izraz lahko poskušamo
prevesti tudi kot »biti zatopljen v nekaj.«) Za razlago glej tudi str.
\pageref{vednostjo}-\pageref{vednostjo-end}.

\item[Obstoj] ali »bivanje« ali »eksistenca«
(\emph{bhava}) je vse, kaj jemljemo kot nekaj, kar je ustvarjeno, da
\emph{je}, da obstaja. Naš odnos do obstoja je formuliran, zato pravimo,
da je to »bivanje« osnovano na nevednem domnevanju, ki je odvisno od
pogledov. S prenehanjem \emph{bhave} ne pomeni, da prisotnost
(\emph{nāmarūpa-viññāna}) izgine. Prisotnost ne potrebuje bitnosti, oz.
ne potrebuje naših pogledov in kontrole.

\item[Sprejem] (\emph{phassa}) je med subjektom in objektom. Ko ta
utvara o dualnosti izgine, izgine tudi ta stik. Ko sutte govorijo o
razumevanju sprejema, to pomeni razumevanje dualnosti.

\item[Vplivi] (\emph{āsava}) so čutnost (\emph{kāma}), bivanje
(\emph{bhava}), pogledi (\emph{diṭṭhi}) in nevednost \emph{(avijjā}). To
so stvari, ki nas neprestano vodijo v nadaljnjo nevednost in trpljenje.

\item[Morala in običaji] (\emph{sīlabbata}) izraz zajema vsa osebna
mnenja o morali, družbeno dogovorjenih običajih in moralnih naukih, ki
jih ponujajo religije in filozofije. Več o tem na str.
\pageref{silabbata}.

\end{description}

\chapter{Sporočilo Aṭṭhakavagge}

\section{Povzetki glede na tematiko}

\emph{Aṭṭhakavagga} seveda govori le o eni stvari in to je o razumevanju
stvari, takih kot so. Toda, če govorimo o posameznih kvalitetah Dhamme,
to ne pomeni, da te kvalitete lahko raziskujemo kot posamezne, ampak jih
moramo kot del celotnega učenja. Posebno pri \emph{Aṭṭhakavaggi} je
zanimivo, da je predvsem osredotočena na učenje o »domnevah« ali
\emph{upādāni}. V \emph{paṭiccasamuppādi} je Buddha opisal
\emph{upādāno} kot čutnost (\emph{kāma}), poglede (\emph{diṭṭhi}),
moralo in običaje (\emph{sīlabbata}) in sebstvo (\emph{attavāda}).
Celotna zbirka je osnovana prav na teh elementih.

Elementi, ki se pojavljajo v Aṭṭhakavaggi so tudi sledeči:
nenavezanost, razumevanje resnice, samo-očiščevanje, iskanje miru,
samotno bivanje, dosego sadov z realizacijo Dhamme, izničenje teme,
prenehanje argumentov in razvoj dobrih kvalitet. Pa poglejmo zdaj
podrobneje te elemente.

\clearpage

\subsection{Nenavezanost}

Ena izmed najbolj očitnih tem v
\emph{Aṭṭhakavaggi} je navezanost -- posebno navezanost na čutne užitke
in na poglede.

Navezanost na čutne užitke pomeni, da je človek ko »ti predmeti
čutnosti oslabijo, v taki stiski, kot da je preboden s trnom« (Aṭṭ 1:2).
Za tiste, ki so navezani, »ko zakrivijo neprijetnosti, objokujejo: »Kaj
bo z nami, ko bomo odšli od tu?«« (Aṭṭ 2:2). To učenje nakazuje, da ni
problem v občutenju prijetnega (\emph{sukha}), neprijetnega
(\emph{dukkha}) in nevtralnega (\emph{adukkhamasukha}) občutka, ampak v
navezovanju na ta občutek. Človek lahko še vedno čuti neprijeten
občutek, toda ker ni navezovanja in želja, ta človek ne čuti \emph{muke}
(\emph{dukkha}). Želja po izničenju neprijetnosti, je trpljenje. Želja
po izogibanju prijetnega občutka za bolj prijetni občutek, je trpljenje.
Tudi želja po »prebuditvi« iz nevtralnega občutka v bolj čutno
občutenje, je trpljenje. Na kratko, kontrola nad občutki, je trpljenje.

Navezanost na poglede (nazore), pomeni jemati svoje poglede kot
najboljše na svetu (Aṭṭ 5:1), zato jih jemljemo, kot da to so absolutna
(Aṭṭ 8:9) in univerzalna (Aṭṭ 4:1; 5:1) resnica. Taki pogledi vodijo k
razpravam (Aṭṭ 3.1), prepiru (Aṭṭ 5:1), zaničevanju (Aṭṭ 11:2) in
imenovanju drugih za »bedake« (Aṭṭ 12:10). Medtem, ko pri navezanosti na
čutne užitke govorimo o občutkih, pri navezanosti na poglede govorimo o
obstoju in neobstoju (\emph{bhava-vibhava})\emph{.} Vsi pogledi so
formulirani glede na to, kar izberemo iz ponujenih stvari ali izločimo
iz stvari, ki ne podpirajo naših pogledov. Toda to ne pomeni nujno, da
smo na to pozorni. Poglede napačno jemljemo kot da je to resnica, kot da
so del našega razumevanja obstojnega sveta ali eksistence. Ti so za nas
»absolutna« resnica. Buddha nas tu opozarja, da taki pogledi potrebujejo
potrdila in zagotovila, kar pomeni stalno potrjevanje te eksistence
(\emph{bhava}) od drugih stvari, ki ne podpirajo te eksistence (oz.
ne-eksistence ali \emph{vi-bhava}). Da »pozitivnost« lahko za nas
obstaja, mora hkrati obstajati tudi »negativnost«. Zato pravimo, da se
pri navezovanju po obstoju, navezujemo tudi po neobstoju. In tudi, ko se
navezujemo po neobstoju, se hkrati navezujemo tudi na obstoj.
Osvoboditve ni v navezovanju.

Kar se tiče prenehanja navezovanja, glede na \emph{Aṭṭhakavaggo}, se
to lahko doseže z vrlinami, t.j. s samoomejevanjem in odrekanjem. Človek
ne more prekoračiti poplavo obžalovanj brez vztrajnega in stalnega
»črpanja vode iz svojega čolna« (Aṭṭ 1:6). To je seveda postopna in
težka vadba (Aṭṭ 2:1), saj zahteva osvobajanje od čutnosti. A za to ni
dovolj, da se človek skriva v samoti, stran od čutenja, ampak mora tudi
razumeti odvisnost samega doživljanja (Aṭṭ 2:1). Tisti, ki ni brez vezi,
ne doseže osvoboditve (Aṭṭ 2:2).

Buddha je tudi govoril o prenehanju navezovanja na oboje, na slaba
dejanja in \emph{tudi} na koristna dejanja (Aṭṭ 4:3), oklepanja na
zaznave (Aṭṭ 4:5), na znanja (Aṭṭ 5:5), na eksistenco (Aṭṭ 5:6), na
moralo in običaje, in na vsa dejanja (\emph{kamma}) ne glede ali so
kritizirana ali odobravajoča (Aṭṭ 13:6). To so navezovanja, ki so
prisotna v vsakem običajnemu človeku.

Osvoboditev od navezovanja na moralo in običaje ne pomeni, da bi zdaj
morali biti nemoralni. Na primer, Buddha je učil v \emph{Aṭṭhakavaggi},
da je pravi modrec tisti, ki se obvladuje v govoru (Aṭṭ 10:3), ki ni
naklonjen žalitvam (Aṭṭ 10:5), itd. Namen morale ni navezovanje na neke
religiozne ideologije ali kontrola družbe, ampak da se vzdržimo vseh
dejanj, ki so pod vplivom odpora, čutnosti ali nevednosti. To se najprej
začne z zadržanjem od slabih dejanj. Buddha je svetoval, da se
\emph{vsak} človek drži \emph{vsaj} petih vodil: da ne ubija, da ne
krade, da se v spolnosti moralno vede, da ne laže in da ne uživa drog in
alkohola, kar vodi v nepazljivost. Z vadbo se lahko človek potem
osvobodi tudi dobrih dejanj -- torej, od vseh dejanj (\emph{kamma})
nasploh (Aṭṭ 13:6), ne glede ali so to fizična, verbalna ali mentalna.
Kaj to pomeni? Vsa dejanja, ki jih počnemo, so vedno pod vplivom odpora,
čutnosti ali nevednosti, dejanja in so tvorjena s pogledi sebstva: »To
je moje delo« in »Moj govor« in tudi »Mislim, torej sem«. Dokler je
človek neveden, deluje pod vplivom odpora, čutnosti in nevednosti in je
zatorej odgovoren za svoja dejanja. Osvobojeni človek, ki nima več
odpora, čutnosti ali nevednosti, ne škoduje niti sebi niti drugim.

\label{silabbata}
Častiti Soṅa je rekel, da je moralno vedenje razsvetljenega človeka
(\emph{arahat}) naravno izražanje osvobojenega uma. Ta človek ni
nepokvarjen zaradi slepega sledenja ritualom, zapovedi ali kakršenkoli
moralni etiki (\emph{silabbataparāmāsa}), ampak je odločen v
nepokvarjenosti zaradi prenehanja odpora, čutnosti in nevednosti
(\emph{khayā rāgassa vītarāgattā \ldots{} dosassa vītadosattā\ldots{}
mohassa vītamohattā abyāpajjādhimutto hoti}) (Vin 1:183-5). Kot lahko tu
vidimo, Buddha ni učil nove religije. Misli o ne-religioznosti so lepo
zapisane v Aṭṭ 4.

\subsection{Spoznanje resnice}

Buddha pravi, da je resnica le ena
(Aṭṭ 12:7). Toda, ker se ljudje v svetu identificirajo s pogledi, za
njih seveda obstaja mnogo »resnic« in zaradi tega pride do nesoglasja
(Aṭṭ 12:7). Spoznanje resnice ni s sprejemanjem nekih pravih pogledov,
ampak z uvidom, ki temelji na izkušnji (Aṭṭ 14:20). Ta, ki je spoznal
resnico, je to dosegel z opuščanjem vseh stvari (Aṭṭ 15:12), tistih, ki
prinašajo zasluge ali zlo (Aṭṭ 4:3) in so pod vplivom pogledov, glede na
moralo in običaje (Aṭṭ 13:6). Ko je popolnoma prost vseh posvetnih želja
in upanj (Aṭṭ 4:7), lahko biva v svojem miru (Aṭṭ 15:12) in mu ni treba
več dvomiti o »mnogih resnicah«.

Nekateri, ki se imajo za izkušene mislijo, da so njihove ideje,pogledi
in mnenja dokončna in jih imajo za popolna (Aṭṭ 8:1, 8:9, 13:10, 13:16)
in univerzalna (Aṭṭ 8:10, 13:10). Pogojeni v takih pogledih imajo druge
poglede za »nepopolne« (Aṭṭ 12:14) in »lažnive« (Aṭṭ 12:9).

Seveda bi bilo napačno to razumeti kot idejo, da lahko dosežemo mir z
zavračanjem vseh idej, ki so nam ponujene. V DN 74 je Dīghanakha rekel
Buddhi: »Moje mnenje in pogled je, da zame ni ničesar sprejemljivo.«
Buddha temu pripomni: »Ta tvoj pogled, da zate ni ničesar sprejemljivo,
je potem ta pogled tebi sprejemljiv?« Čeprav Buddha sprejme, da je
Dīghanakhin pogled »o prenehanju pogledov« bolj sprejemljiv, saj vodi k
zmanjševanju odpora, čutnosti in nevednosti, kljub temu razloži, da ne
samo, da se na tak način ne more izogniti pogledom, ampak se s tem celo
sprejme novi pogled. Buddha pravi, da ko se človek zase zaveda, da tudi
trdno držanje pogleda »ničesar je zame sprejemljivo« vodi k prepirom s
tistimi ljudmi, ki se držijo določenih pogledov, zato naj človek prav v
tem trenutku opusti ta svoj pogled. Tako lahko človek opusti poglede.
Buddha potem uči o naravi odvisnosti in minljivosti teh tvorjenih
pogledov.

\subsection{Samo-očiščevanje}

Modri ljudje pravijo, da je
očiščevanje srca ali uma vrh duhovne vadbe. Očiščevanje pomeni svobodo
od uveljavljene obsedenosti (Aṭṭ 11:13). Drugi filozofi, ki se držijo
svojih ideologij pravijo, da lahko poteka očiščevanje le v etični vadbi:
v samo-omejevanju (Aṭṭ 13:4) ali z odporom (Aṭṭ 13:7). Ti tudi pravijo,
da le njihovo dogmatično učenje lahko vodi k očiščevanju (Aṭṭ 12:15) in
kritizirajo druge poglede, kot da so to ovire pri očiščevanju (Aṭṭ
12:14). Toda modreci pravijo, da je tako govorjenje le opora za
nadaljnjo navezovanje (Aṭṭ 13:14). Človek se ne očisti z nečim, kaj je
pogojeno, kot so pogledi, pod vplivom čutnega doživljanja in morale in
običajev, ki jih sprejme; človek opusti tako, da ne sprejme nekaj novega
(Aṭṭ 4:3). V nasprotnem primeru, bi človek le ohranjal sebe v bivanju
ali eksistenci (Aṭṭ 13:4).

\subsection{Iskanje miru}

Včasih je cilj omenjen kot mir
(\emph{santa}). Mir ni v iskanju religioznih ideologij (Aṭṭ 9:3). Ta mir
pride iz notranjosti uma in ni v zunanjem svetu (Aṭṭ 14:5). Pride s
prenehanjem jemanjem stvari vase (Aṭṭ 13:18), z opuščanjem vsega (Aṭṭ
15:12; 15:15) do te mere, da človek ne bo imel ničesar več za zavreči
(Aṭṭ 14:5). To se doseže s prenehanjem navezovanja na sebstvo (Aṭṭ
14:19).

\subsection{Samotno bivanje}

Včasih je cilj omenjen kot samota
(\emph{vivekā}). Samota ne pomeni enostavno le fizično samoto (Aṭṭ 2:1),
ampak pomeni samoto od navezovanj, od stvari, ki mažejo um, in od
nevednosti (Aṭṭ 2:1), od vseh stvari, ki jih človek dojame kot kontakt
med osebo in objektom (Aṭṭ 10:4). To pomeni stanje miru, brez vezi na
karkoli na tem svetu (Aṭṭ 14:1), ko človek nima ničesar, kar bi mu lahko
prineslo žalost (Aṭṭ 10:14). Seveda, osama tudi pomeni fizični umik ter
tudi, ko se človek ne predaja spolnosti (Aṭṭ 7:1, 7:7, 7:8), ko nima
novega naraščaja, živine, posestva in lastnine (Aṭṭ 10:11).

V SN 4:37 »samoten človek« (\emph{ekavihāri}) pomeni nekoga, ki tudi v
gneči z ljudmi živi sam, to je brez želja (\emph{taṇhā}) in sebstva
(\emph{attā}). V SN 2:283 Buddha pravi, da je preživetje v osami le en
del popolne osame. Za popolno dosego osame mora človek opustiti 
preteklost, prihodnost, nagnjenja in želje po sedanjih stvareh, ki
tvorijo osebno bitnost.

\subsection{Dosega sadov z realizacijo Dhamme}

Kvalitete, ki pridejo
z realizacijo Dhamme so sledeče: prenehanje iluzije o sebstvu »Jaz sem«
(Aṭṭ 3:4), potrebe po časti (Aṭṭ 3:4) in ponosu (Aṭṭ 3:4); je v miru med
vsemi idejami, ki pridejo pred njega (Aṭṭ 4:6), hodi bister in odprt
(Aṭṭ 4:6); brez hrepenenja po obstoju (Aṭṭ 9:5), brez navezovanj in
splošnega hrepenenja (Aṭṭ 9:5; 10:2; 10:4); je neodvisen (Aṭṭ 10:2;
10:9); brez močnih čustev (Aṭṭ 10:3; 10:5); dobrega vedenja s telesom,
govorom in mislimi (Aṭṭ 10:3; 10:5; 10:6); brez žalosti (Aṭṭ 10:4);
pogledi ga ne vodijo (Aṭṭ 10:4); se ne vpleta v kritike in prepire (Aṭṭ
10:12; 13:18); se ne primerja z drugimi (Aṭṭ 10:8; 10:13); je ravnodušen
do čutnih užitkov (Aṭṭ 10:10); brez fiksiranih predstav (Aṭṭ 10:13) in
slepo ne sledi nobenimi idejam, ne glede na to ali so filozofske,
religiozne ali posvetne (Aṭṭ 10:14).

\subsection{Izničenje teme}

Tema je nekaj, kar se mora pregnati in
končati. Ko tema preneha in pride luč, človek doseže veselje. Tema je
značilnost motenega oz. nevednega uma, in se jo lahko konča z
razumevanjem resnice s pravilnim poenotenim umom in preiskovanjem
Buddhovega učenja (Aṭṭ 16:13, 16:21).

\subsection{Prenehanje argumentov}

Sveta bitja se ne vpletajo v
razprave in prepire (Aṭṭ 3:1; 9:10), saj niso zavezana v nobene ideje
(Aṭṭ 3:8). Ne popuščajo nobenim pogledom (Aṭṭ 5:5). Tudi ne povzdigujejo
sebe in tako ne povzročajo sporov v svetu (Aṭṭ 12:17). Za njih je
ne-prepir »temeljno zatočišče za mir.« (Aṭṭ 13:2)

Toda tisti, ki pravijo, da so njihovi pogledi univerzalna resnica,
imenujejo ideje drugih kot »nižje« in se tako ne izognejo prepirom (Aṭṭ
5:1). S takimi mislimi bodo zagovarjali svoja učenja (Aṭṭ 8:1) in si
začeli iskati nasprotnike, s katerimi bi lahko debatirali, razpravljali
in poskusili pridobiti zasluge, medtem ko imajo druge za bedake (Aṭṭ
8:2). In če izgubijo v debati, so prizadeti (Aṭṭ 8:3) in objokujejo v
žalosti (Aṭṭ 8:4). Tudi zmaga v debatah ne prinese drugega kot ponos.
Buddha temu pravi vir lastne škode (Aṭṭ 8:7). S tem videnjem se naj
človek izogne razpravam in prepirom, saj tako življenje ne vodi h
čistosti (Aṭṭ 8:7).

\subsection{Razvijanje dobrih kvalitet}

V \emph{Aṭṭhakavaggi} lahko
najdemo sto dvaindvajset vodil, kako naj človek živi svoje duhovno
življenje. Dvainosemdeset od teh je uvrščenih v zadnje tri sutte (Aṭṭ
14, 15 in 16) in ta vodila dajo poseben ton tem verzom. Vsa ta vodila
niso izbrana med ideologijami, ampak so osnovana na nenavezanju in
vodijo v mir (Aṭṭ 9:3). Poleg tega ta vodila niso namenjena za
poveličevanje samega sebe (Aṭṭ 7:9, Aṭṭ 9:12). Če človek sledi ošabnosti
in vzvišenosti, potem se bo še bolj utopil v eksistenco (Aṭṭ 13:4).

\section{Od nevednosti do nasprotovanja}

\emph{Mahānidāna sutta} (DN 15) lepo ponazarja, da z neznanjem Dhamme
postaja naše življenje polno nesoglasij, prepirov, nasilja in celo vojn.
Ta sutta pravi, da zaradi neznanja o pogojenosti kakršnegakoli obstoja
mislimo, da je naša eksistenca stabilna in enotna ter da so naši pogledi
vredni čaščenja. S tem neznanjem mislimo, da imamo kontrolo nad občutki,
da jih lahko spreminjamo s sledenjem čutnim užitkom ali z odrivanjem na
stran vsega, česar ne maramo. Z nerazumevanjem narave občutkov, t.j. da
so minljivi oz. med seboj odvisni ter da so neodvisni od sebstva,
uporabljamo to lažno moč kontrole, z mislijo, da lahko stvari (predvsem
občutke) spremenimo, kot je nam po godu. Tej »moči po kontroli« pravimo
želje (\emph{taṇhā}).

\emph{Mahānidāna sutta} od tu naprej poda spisek tega, kako je naše
vsakdanje življenje pogojeno od teh želja. Sutta pravi, da ko je
prisotna želja je tu tudi prisotno iskanje (\emph{pariyesana}); ko je
prisotno iskanje, je prisotno tudi pridobivanje koristi (\emph{lābha});
ko je tu korist, je tu teorija, oz. domneva ali neka enota izbranih
koristnih idej ali stvari (\emph{vinicchayo}); ko je tu teorija, so tu
želje in strasti (\emph{chandarāga}); ko so tu želje in strasti, so tu
tudi navezovanja (\emph{ajjhosāna}); ko je prisotno navezovanje, je tu
posedovanje (\emph{pariggaha}); ko je tu posedovanje, je tu strah pred
izgubo (\emph{macchariya}); ko je tu strah pred izgubo, je tu
zaščitništvo (\emph{ārakkho}); ko se svari branijo, so tu mnogi zli
fenomeni, kot npr. uporaba orožja za obrambo, prepiri, prerekanje,
nasilje, žaljivke, obrekovanje in laži.

Ta spisek ne pravi, da je med temi individualnimi točkami neka časovna
razlika, na primer, najprej iščem in kasneje se navezujem. Sutta pravi,
da ko je tu prvo, je hkrati tudi drugo. Zato se lahko ta spisek bere
tudi kot, na primer, ko je tu strast, je tu strah pred izgubo. To potem
tudi pomeni, da ko je tu nasilje, je to zaradi nevednosti, ne-vedenja
narave uma, ki je prisotno prav tu in zdaj. Dokler smo nevedni, tudi
iščemo, imamo strasti, smo zaščitniški, itd. ne glede ali smo na to
pozorni ali ne. Prav zato se mora \emph{Aṭṭhakavaggo} jemati kot nekaj
zelo osebnega, kot besede, ki govorijo prav \emph{tebi osebno} in ne kot
nekaj, kar se tiče drugih ljudi, češ, da je to problem nekoga drugega, a
ne moj. Poleg tega \emph{Aṭṭhakavagga} ni zapisana za nek določen čas
ali le za neko določeno versko skupino. Velja prav tu, zdaj in za vse.

Zdaj lahko pogledamo, kako \emph{Aṭṭhakavagga} razloži te ključne
faktorje, ki pogojujejo naš pogled na samega sebe in na svet. 

Ko so
prisotne želje, je to vedno \emph{po nečem,} po stvareh, ki jih čutimo s
petimi čuti. (V pesniški obliki sutte zajamejo vseh pet čutov s: »kar je
videl, slišal ali občutil.« Ostale čutne osnove, to so vonjanje,
okušanje in dotikanje, so pravzaprav združene v »občutenje«.) To čutenje
seveda potrebuje objekt. Toda objekt ni neka enotna stvar. Vsak zavesten
objekt je določen ali tvorjen od mnogih stvari (\emph{saṅkhāra}). 

Želje
neprestano iščejo enotnost med temi neskončnimi podobami. Tej
»enotnosti« lahko pravimo dobiček, oz. korist. \emph{Aṭṭhakavagga}
pravi, da iskanje koristi vodi le v prerekanja (Aṭṭ 8:5), razočaranja
(Aṭṭ 10:7) in do težav med ljudmi (Aṭṭ~14:15). Toda enotnost ni nič
drugega kot teorija. Tu ni jasnega videnja o tej enotnosti; ali z
drugimi besedami, ničesar ni \emph{dokazanega}. Zato tu govorimo o
\emph{obstoju} teorij, ki so nastale z našimi pogledi, ki podpirajo
določeno teorijo. Posledično, stvari, ki ne podpirajo te teorije,
odrinemo na stran (čeprav na to nismo vedno pozorni). Zato
\emph{Aṭṭhakavagga} pravi: »Z videnjem obstoja in neobstoja snovi, si
človek ustvarja teorije v svetu.« (Aṭṭ 11:16) 

To razlikovanje se pokaže
tudi v vsakdanjem odnosu do sočloveka: »Ko stoji v svoji teoriji, se
posmehuje drugim in pravi: »Ta je bedak in ne mojster«« (Aṭṭ 12:10) in
»Dokler se človek drži teorij, povzdiguje zgolj sebe in nadaljuje s
prepiri v svetu.« (Aṭṭ 12:17) 

Tu so želje do teh dveh skrajnosti: po
obstoju in po izničenju obstoja (Aṭṭ 2:7), ki so osnovane na tem, kar je
»prijetno« in »neprijetno« v svetu.« (Aṭṭ 11:6) Tako si ustvarjamo
stvari, ki so nam drage, in s tem obstajajo hrepenenja, upi in cilji
(Aṭṭ 11:4) in strast do pogledov (Aṭṭ 12:14). Tudi, ko opuščamo neke
določene poglede, si posledično hkratno že jemljemo nove poglede: »Ko
ljudje opustijo, kar je bilo, a se držijo tega, kar sledi, jih premaga
navezanost, ker sledijo svoji strasti.« (Aṭṭ 4:4) 

Očitno je želje zelo
težko premagati, ker so trdno navezane na čutna poželenja (Aṭṭ 15:14).
Ko je tu navezovanje, so tu posedovanja do vseh stvari, ki jih jemljemo
kot »moje« (Aṭṭ 6:2, 11:10-11) in kot nam drage (Aṭṭ 11:2). In tako
sledi žalost, tegoba in strah pred izgubo (Aṭṭ 6:6) in posledično tudi
pohlep (Aṭṭ 10:13). Tak človek je zaščitniški, spletkarski, drzen in
poln odpora (Aṭṭ 10:5). Prav zagotovo se kdaj ujamemo, ko začnemo
zagovarjati svoje misli ali dejanja. Takega človeka prevlada jeza,
strah, skrbi in njegov govor postane neobvladan (Aṭṭ 10:3). Tako se
človek zapleta v argumente, debate in širjenje svojih pogledov,
ideologij in verovanj. To lahko vodi v prepire, žaljivke, obrekovanje,
laži, ustrahovanje, nasilje, pretepe, ubijanje in mnoga druga škodljiva
dejanja.

\section{Z vednostjo pride končni mir}
\label{vednostjo}

In kako se lahko vse to tudi konča? »Z jasnim razumevanjem kontakta,«
(Aṭṭ 2:7) pravi Buddha, ko človek uvidi naravo odvisnosti ter tako
opusti domneve (\emph{upādāna}) o enotnem obstoju ali eksistenci: to je
prenehanje kontakta (\emph{phassa nirodha}). Seveda zadeva ni enostavna
in tega ni mogoče razumeti s samim logičnim razumevanjem. Potrebno je
odkrivanje teh stvari na fenomenološki način, t.j. s prepoznavanjem
stvari, ki so prisotne ravno tu in zdaj pred nami in ne kot nekaj
abstraktnega. Aṭṭ 14, 15 in 16 govori, kako naj praktikant vadi, da
lahko razume pravo naravo.

Najbolj zanimivi koncept, ki je omenjen v tej zbirki je
\emph{papañcasaṅkhā}. Te zloženke ni lahko prevesti. Mnogo prevajalcev
si je že belilo glavo, kako to spraviti v njihov domači jezik. Za to
slovensko zbirko sem uporabil prevod »uveljavljena obsedenost«.
\emph{Papañca} pokriva celotno naše delovanje: kako razmišljamo,
opravljamo svoje opravke in komuniciramo z drugimi ljudmi. Na primer, v
AN 6:14 se \emph{papañca} prepozna v veselju pri delu, veselju pri
govorjenju, veselju po spanju, veselju v družbi, veselju pri
medsebojnemu vezanju, skratka veselju po bivanju. Ne glede kakšne
interakcije imamo do sveta, se to dogaja pod vplivom naših pogledov. V
njih se zatopimo. Toda pogledi niso nekaj enotnega ali stabilnega, ampak
nekaj, kar raste in se razprostira. Ne glede, koliko se zaznava
razprostira, se na to uveljavi zavest. Ta »uveljavljenost« nam daje
občutek vrednosti in stabilnosti. In misel »Jaz sem« je najbolj
problematična uveljavljena obsedenost (Aṭṭ 14:2). Z nevednostjo potem
mislimo na način: »Moj svet, ki ga spreminjam in mi hkrati daje pomen.«
S takimi pogojenimi mislimi človek širi svoje ideje in prepričanja.

Ta »virus« vpliva tudi na odnose do sočloveka. Da lahko skupina ljudi
živi skupaj, si morajo najti neke skupne točke v katerih (mislijo) imajo
skupne ideje. Seveda skupna ideologija ne more obstajati, če se ne
identificira z nečim, kar ni »naša« ideologija. Potrebna je druga
miselnost, kljub temu da se ta imenujejo »nižja«: heretična, zla,
sovražna, itd. (Aṭṭ 12:11, 12:9). Na tej nevednosti je grajen ta svet,
ki ga mi poznamo. Rast religij raste z željami po moči in kontroli.
Takšne so tudi ideološke vojne, ki jih poznamo v stari zgodovini in ki
še vedno »obsedajo« svet in povečujejo napetost med »Vzhodom« in
»Zahodom« ali »levico« in »desnico«. Torej, \emph{karkoli} človek
zagovarja, zagovarja tisto, kar je izbral iz tistega, česar \emph{noče}
zagovarjati. Tu je ta dvojnost pojavnega: obstoj in neobstoj (Aṭṭ 11:6).

Kdor se drži pogleda, da nekaj za njega obstaja, se hkrati drži tudi
pogleda, da nekaj za njega ne obstaja; in tudi, če se drži pogleda, da
nekaj ne obstaja, se hkrati drži pogleda, da nekaj obstaja (DN 11). Tak
človek ne pobegne problemu, ampak se vrti okoli obstoja ali eksistence,
kot pes prevezan na kol z vrvjo. Teče in laja okoli tega istega kola (SN
22:99). Tako so tu »jeza, laž, negotovost in ideje, ko je prisotna ta
dvojnost.« (Aṭṭ 11:7) In seveda vsa pogojena razumevanja, znanja, ideje
in pogledi niso »brezmadežni« (Aṭṭ 12:4)

\emph{Papañca} preneha postopoma, ko človek nič več ne uživa v
obsedenosti oz. vznemirjenosti do vsega, kar svet človeku ponuja.
»Raziskovalec razume naravo odvisnosti. Z razumevanjem je osvobojen in
se ne prepira« (Aṭṭ 11:16) in opusti želje (Aṭṭ 2:7).

Človek razume, da je vse, kar prinaša »navdušenje«, »ne-navdušenje«,
»obstoj« in »neobstoj« v svetu (Aṭṭ 11:6) pogojeno s kontaktom (med
»jazom« in »svetom«) (Aṭṭ 11:9). Tudi kontakt je nekaj, kar je pogojeno:
to je od »imena in snovi«, oz. od pojavnega fenomena: »Če snov
izgine, se kontakt ne sklene.« (Aṭṭ 11:11). In kako snov pokaže
svojo odvisnost in s tem minljivost?
\label{vednostjo-end}

\begin{verse}
Ko nezaznava zaznavo; ko nezaznava nezaznavo;\\
ko ne nezaznava; ko ne zaznava »izginitve«.\\
Za tistega, ki to doseže, snov izgine,\\
saj je zaznava osnova za uveljavljeno obsedenost. (Aṭṭ 11:13)
\end{verse}

To se zgodi z direktnim spoznanjem, ko zaznava le zaznava, ko občutki le
čutijo. Domneva (\emph{upādāna}), da zaznava zaznava snov, je le
domneva, ne resnica. Domneva, da občutek čuti snov, je le domneva,
ne resnica. Domneva daje snovi obstojnost ali bitnost, toda v resnici
snov te bitnosti ne potrebuje. Snov se ne da na noben način
kontrolirati, saj se sama razporedi glede na štiri elemente.

\begin{verse}
Če ničesar ne jemlje in ne doživlja kot »moje«\\
v kakršnemkoli pogledu na ime in snov,\\
in ko ga ne žalosti to, česar ni,\\
ta resnično ne trpi posvetne izgube. (Aṭṭ 15:16)
\end{verse}

S tem tudi prenehajo občutki, ki so odvisni od sprejetja. Tako človek
spozna, da nima smisla domnevati o sebstvu, ki naj bi obvladoval
doživljanje. Doživljanje nam je dano, preden domnevamo to edinstveno
osebo in njeno moč kontrole. Tako izginejo želje in iskanje: »Z
opuščanjem želja po obeh skrajnostih\ldots{} se modri ne oklepa ničesar
kar vidi in sliši« (Aṭṭ 2:7) in »Menih s pozornostjo in s popolnoma
osvobojenim srcem, zavrne vsako željo po teh čutnih objektih.
Pravočasno, pravilno in temeljito preiskuje Dhammo, z poenotenim umom
naj izniči vso to temo.« (Aṭṭ 16:21) In kdor je prešel čutna poželenja,
je prešel navezovanja (Aṭṭ 15:14). Z izpuljenim trnom ni pogreznjen v
posedovanje (Aṭṭ 2:8), saj človek razume, da »resnično, trajno imetje ne
obstaja« in da »so ločitve neizogibne.« (Aṭṭ 6:2) Človek je popolnoma
neodvisen, se z nikomer in ničemer ne primerja, nima ničesar izgubiti,
brez strahu in žalovanja živi v miru (Aṭṭ 6:8, 15:20).

Kot smo verjetno do zdaj spoznali, v Buddhovem učenju človek ne
postane »svetnik« zaradi širjenja ali zagovarjanja tega nauka, pa
čeprav je ta človek pripravljen za to žrtvovati svoje življenje. Buddha
je tudi pravil: »Menihi, če bi kdorkoli govoril proti meni ali učenju
ali skupnosti, se ne odzovite na to z jezo, odporom ali sovraštvom. Če
si jezen in imaš občutek odpora in sovraštva, bo to zate le ovira.« (DN
1) »Četudi razbojnik z žago reže tvoje ude, pa v tvojih misli nastane
jeza do tega razbojnika, potem ne uresničuješ mojega učenja.« (MN 21)
Buddha pravi, da človek postane pravi svetnik le tako, da se osvobodi
odpora, čutnosti in nevednosti. Takemu človeku se ni potrebno
zagovarjati, prepirati niti iskati podpore in priznanja. Ta živi v
popolnem miru.

{\setlength{\stanzaskip}{0.5\baselineskip}
\begin{verse}

Tisti, ki ni z nasprotnega brega\\
niti ne s tega brega in niti ne z obeh,\\
ki je brez strahu in nenavezan,\\
ta je zame sveti človek. 

Kdor lahko mirno prenese\\
zmerjanje, udarce, okove,\\
čigar obramba je moč potrpežljivosti,\\
ta je zame sveti človek. 

Potrpežljiv med nepotrpežljivimi,\\
razorožen med oboroženimi,\\
brez hlepenja med pohlepneži,\\
ta je zame sveti človek. 

\emph{Dhammapada} 26:3,17,24\\
(Prevod: Bhikkhu Hiriko)
\end{verse}}

