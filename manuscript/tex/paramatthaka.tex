\cleartorecto
\chapterNote{Paramaṭṭhaka Sutta}
\chapter{Učenje o univerzalnosti}

%\verseref{1}
\dropCap{T}isti, ki živi v svojih pogledih:\\
»To je univerzalnost«,\\
ceni eno znanje iz sveta kot najboljše;\\
za vsa ostala znanja pravi, da so »nižja« znanja.\\
Ta zagotovo še ni šel preko prepirov.

%\verseref{2}
Kakršnokoli korist vidi za sebe v tem,\\
kar je videno, slišano, občuteno, v morali in običajih,\\
poveličuje in vidi le to,\\
vse ostalo vidi kot manjvredno.

\clearpage

%\verseref{3}
Ampak mojstri pravijo, da je v tem skrita past,\\
saj je tako videnje odvisno od »nižjih« videnj.\\
Torej, od vsega kar vidi, sliši ali občuti\\
in od vsega, kar izhaja iz običajev in morale,\\\vin naj menih ne bo odvisen.

%\verseref{4}
Poleg tega, se naj ne oklepa pogledov na svet,\\
ki so osnovani na znanju, kakor tudi na morali in običajih.\\
Sebe naj ne predstavlja kot enakega,\\
niti ne slabšega niti boljšega.

%\verseref{5}
Ko bo opustil pridobljeno in se ne bo oklepal na novo,\\
ne bo odvisen niti od znanja.\\
On resnično ni privrženec učenjakov\\
in ne popušča nobenemu pogledu.

%\verseref{6}
Brez težnje do obeh skrajnosti --\\
za ta ali drugi obstoj, za tukaj in zdaj ali v bodoče --\\
za njega ni nobenih utrjenih pogledov,\\
ki izhajajo iz vrednotenje idej.

%\verseref{7}
Za njega, tu, v vidnem, slišnem ali čutnem,\\
ni prirejena niti najmanjša zaznava.\\
Ta sveti človek si pogledov ne prisvaja --\\
le s čim na svetu bi se ga lahko merilo?

%\verseref{8}
Ničesar si ne izmišljuje, nič si ne izbira;\\
ne sprejema nobene miselnosti.\\
Sveti človek se ne pusti zapeljati morali in običajem,\\
ko stopi na drugi breg, se ne vrne več nazaj.

