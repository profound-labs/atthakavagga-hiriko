
% === Pali ===

\cleartoverso
\chapter*{Paramaṭṭhaka Sutta}

\verseref{1}
\dropCap{p}aramanti diṭṭhīsu paribbasāno\\
yaduttari kurute jantu loke\\
hīnāti aññe tato sabbamāha\\
tasmā vivādāni avītivatto

\verseref{2}
yadattanī passati ānisaṁsaṁ\\
diṭṭhe sute sīlavate mute vā\\
tadeva so tattha samuggahāya\\
nihīnato passati sabbamaññaṁ

\verseref{3}
taṁ vāpi ganthaṁ kusalā vadanti\\
yaṁ nissito passati hīnamaññaṁ\\
tasmā hi diṭṭhaṁ va sutaṁ mutaṁ vā\\
sīlabbataṁ bhikkhu na nissayeyya

\verseref{4}
diṭṭhimpi lokasmiṁ na kappayeyya\\
ñāṇena vā sīlavatena vāpi\\
samoti attānamanūpaneyya\\
hīno na maññetha visesi vāpi

% === Slovenian ===

\cleartorecto
\chapter{Osemverzno učenje o univerzalnosti}

%\verseref{1}
\dropCap{T}isti, ki živi v svojih pogledih:\\
»To je univerzalnost«,\\
ceni eno znanje iz sveta kot najboljše;\\
za vsa ostala znanja pravi, da so »nižja« znanja.\\
Ta zagotovo še ni šel preko prepirov.

%\verseref{2}
Kakršnokoli korist vidi za sebe v tem,\\
kar je videno, slišano, občuteno, v morali in običajih,\\
poveličuje in vidi le to,\\
vse ostalo vidi kot manjvredno.

%\verseref{3}
Ampak mojstri pravijo, da je v tem skrita past,\\
saj je tako videnje odvisno od »nižjih« videnj.\\
Torej, od vsega kar je videno, slišano ali občuteno,\\
in od vsega, kar izhaja iz morale in običajev,\\\vin naj menih ne bo odvisen.

%\verseref{4}
Poleg tega, se naj ne oklepa pogledov na svet,\\
ki so osnovani na znanju, kakor tudi na morali in običajih.\\
Sebe naj ne predstavlja kot enakega,\\
niti ne slabšega niti boljšega.

% === Pali ===

\clearpage

\verseref{5}
attaṁ pahāya anupādiyāno\\
ñāṇepi so nissayaṁ no karoti\\
sa ve viyattesu na vaggasārī\\
diṭṭhimpi so na pacceti kiñci

\verseref{6}
yassūbhayante paṇidhīdha natthi\\
bhavābhavāya idha vā huraṁ vā\\
nivesanā tassa na santi keci\\
dhammesu niccheyya samuggahītaṁ

\verseref{7}
tassīdha diṭṭhe va sute mute vā\\
pakappitā natthi aṇūpi saññā\\
taṁ brāhmaṇaṁ diṭṭhimanādiyānaṁ\\
kenīdha lokasmiṁ vikappayeyya

\verseref{8}
na kappayanti na purekkharonti\\
dhammāpi tesaṁ na paṭicchitāse\\
na brāhmaṇo sīlavatena neyyo\\
pāraṅgato na pacceti tādīti

% === Slovenian ===

\clearpage

%\verseref{5}
Ko bo opustil pridobljeno in se ne bo oklepal na novo,\\
ne bo odvisen niti od znanja.\\
On resnično ni privrženec učenjakov\\
in ne popušča nobenemu pogledu.

%\verseref{6}
Brez težnje do obeh skrajnosti --\\
za ta ali drugi obstoj, za tukaj in zdaj ali v bodoče --\\
za njega ni nobenih utrjenih pogledov,\\
ki izhajajo iz različnih ideologij.

%\verseref{7}
Za njega, tu, v vidnem, slišnem ali čutnem,\\
ni prirejena niti najmanjša zaznava.\\
Ta sveti človek si pogledov ne prisvaja --\\
le s čim na svetu bi se ga lahko sodilo?

%\verseref{8}
Ničesar si ne izmišljuje, nič si ne postavi na svojo pot;\\
ne sprejema nobene miselnosti.\\
Sveti človek se ne pusti zapeljati morali in običajem,\\
ko stopi na drugi breg, se ne vrne več nazaj.

