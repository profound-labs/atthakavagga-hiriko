
% === Pali ===

\cleartoverso
\chapter*{Pasūra Sutta}

\verseref{1}
\dropCap{i}dheva suddhī iti vādayanti\\
nāññesu dhammesu visuddhimāhu\\
yaṁ nissitā tattha subhaṁ vadānā\\
paccekasaccesu puthū niviṭṭhā

\verseref{2}
te vādakāmā parisaṁ vigayha\\
bālaṁ dahantī mithu aññamaññaṁ\\
vadanti te aññasitā kathojjaṁ\\
pasaṁsakāmā kusalā vadānā

\verseref{3}
yutto kathāyaṁ parisāya majjhe\\
pasaṁsamicchaṁ vinighāti hoti\\
apāhatasmiṁ pana maṅku hoti\\
nindāya so kuppati randhamesī

\verseref{4}
yamassa vādaṁ parihīnamāhu\\
apāhataṁ pañhavimaṁsakāse\\
paridevati socati hīnavādo\\
upaccagā manti anutthunāti

% === Slovenian ===

\cleartorecto
\chapter{Učenje Pasūri}

%\verseref{1}
\dropCap[»]{S}amo tukaj je čistost« -- tako trdijo.\\
Pravijo, da čistosti ni v drugih ideologijah,\\
da je Dobro le v tem, v kar oni zaupajo --\\
tako so prepričani vsak v svojo resnico.

%\verseref{2}
Z željami po prepirih se zapodijo v zborovanja,\\
nasprotniki, drug drugega imajo za bedaka.\\
Navezani na debate razpravljajo med seboj,\\
častihlepni trdijo, da so mojstri.

%\verseref{3}
V razpravi sredi zborovanja,\\
želeč pohval, skrbi ga, da bo poražen.\\
In ko je ovržen, je otožen,\\
razburjen zaradi graje, išče šibkosti.

%\verseref{4}
Ko sodniki povedo, da je njegov dokaz\\
pomanjkljiv in izpodbit,\\
s slabimi argumenti objokuje in se žalosti;\\
»Premagal me je,« se joče.

% === Pali ===

\clearpage

\verseref{5}
ete vivādā samaṇesu jātā\\
etesu ugghāti nighāti hoti\\
etampi disvā virame kathojjaṁ\\
na haññadatthatthi pasaṁsalābhā

\verseref{6}
pasaṁsito vā pana tattha hoti\\
akkhāya vādaṁ parisāya majjhe\\
so hassatī unnamatī ca tena\\
pappuyya tamatthaṁ yathā mano ahu

\verseref{7}
yā unnatī sāssa vighātabhūmi\\
mānātimānaṁ vadate paneso\\
etampi disvā na vivādayetha\\
na hi tena suddhiṁ kusalā vadanti

\verseref{8}
sūro yathā rājakhādāya puṭṭho\\
abhigajjameti paṭisūramicchaṁ\\
yeneva so tena palehi sūra\\
pubbeva natthi yadidaṁ yudhāya

\verseref{9}
ye diṭṭhimuggayha vivādayanti\\
idameva saccanti ca vādayanti\\
te tvaṁ vadassū na hi tedha atthi\\
vādamhi jāte paṭisenikattā

% === Slovenian ===

\clearpage

%\verseref{5}
Taki prepiri nastajajo med misleci,\\
med njimi so le zmagovalci in poraženci.\\
S tem razumevanjem naj se vsak izogiba prerekanjem,\\
saj v njih ni drugega namena, kot gonja za slavo.

%\verseref{6}
A če je pohvaljen sredi zborovanja\\
ob razglasitvi svojih argumentov,\\
je pri tem prešeren in samozadovoljen,\\
saj je dosegel svoj namen.

%\verseref{7}
Toda vsako samozadovoljstvo je vir lastne škode,\\
saj se prereka z domišljavostjo in zaničevanjem.\\
Ko človek to razume, naj se ne prepira --\\
mojstri pravijo, da v tem zagotovo ni čistosti.

%\verseref{8}
Kot junak, nahranjen s kraljevsko hrano,\\
človek kot grom vneto išče si nasprotnika.\\
Pobegni, kjerkoli se pojavi on, junak.\\
Tu ni ničesar zate, kar je vredno boja.

%\verseref{9}
Tisti, ki so si prisvojili poglede,\\
prepirajoč trdijo: »Le to je resnica.«\\
A ti jim reci, da zate ni nihče nasprotnik,\\
ne glede na to, do kakšnih prepirov je prišlo.

% === Pali ===

\clearpage

\verseref{10}
visenikatvā pana ye caranti\\
diṭṭhīhi diṭṭhiṁ avirujjhamānā\\
tesu tvaṁ kiṁ labhetho pasūra\\
yesīdha natthī paramuggahītaṁ

\verseref{11}
atha tvaṁ pavitakkamāgamā\\
manasā diṭṭhigatāni cintayanto\\
dhonena yugaṁ samāgamā\\
na hi tvaṁ sakkhasi sampayātaveti

% === Slovenian ===

\clearpage

%\verseref{10}
Tisti, ki so prenehali z bojem,\\
ne izpodbijajo pogleda s pogledom.\\
Kaj bi rad pridobil od njih, Pasūra,\\
ko pa si ti ničesar ne lastijo?

%\verseref{11}
In sedaj si prišel sem razpravljat\\
pod vplivom notranjega prepričanja.\\
A stopil si v kontakt s prečiščenim človekom\\
in na tak način res ne boš mogel napredovati.

